\chapter{Indledning}\label{Indledning}
Trafiksystemet i Danmark undergår ofte udbygninger og ændringer, hvilket kan påvirke nærliggende vejnet. Det kan være svært at se hvordan disse ændringer vil påvirke trafikken, og derfor er der blevet opstillet forskellige modeller til at forudsige hvordan trafikken på vejnettet i fremtiden vil afvikle sig. Disse trafikmodeller bliver ofte opbygget med et konkret formål i fokus, og bliver derfor svære at vedligeholde i fremtiden i de tilfælde hvor konteksten ændrer sig. Konsekvensen af dette er at under en tredjedel af modellerne er blevet vedligeholdt, og at der ikke længere findes en model der dækker hele Danmark \cite[s. 1-2]{dtfnotat}.

\section{Problemets Relevans}
Formålet med trafikmodellerne er at forhindre trafikpropper og at sænke rejsetiderne. Uden modellerne er det besværligt at bestemme hvor der er problemer i vejnettet, og hvilken effekt nye veje vil have på trafikstrømmen. Trafikpropper har en effekt på landets økonomiske vækst, forurening, livskvalitet og tiden det tager for beredskaber og politi at nå frem.

\vspace{5mm}

Er man uheldig, kan man risikere at sidde fast i trafikken på vej til arbejdet. For at finde ud af hvilken effekt denne spildtid har på Danmarks økonomi, har Michael Knørr Skov og Karsten Sten Pedersen, der arbejder for konsulent firmaet COWI, analyseret 3 vejprojekter \cite{trafikoekonomi}. Vejprojekterne inkludere en tredje Limfjordsforbindelse, en ny motorvejsstrækning ved København, og en Forbindelse mellem Fyn og Als. Udfra COWI’s beregninger vil disse tilføjelser spare danskere 25 tusinde timer dagligt, hvilket svarer omtrent til en værdi på 2500 millioner kroner årligt. Antager man at en fjerde del af denne tid bliver brugt på arbejde vil man opleve en BNP-vækst på 0,035\%. Et velfungerende vejnet er dermed et vigtigt aspekt i forhold til at forbedre Danmarks økonomiske vækst. 

\vspace{5mm}

Billister er en af de største kilder af CO\textsubscript{2} forurening. Mængden af CO\textsubscript{2} der bliver udsluppet, afhænger af hastigheden billisterne kører. Ved en lav hastighed kan CO\textsubscript{2} udslippet per kilomet blive fordoblet, i forhold til at køre en stabil 50-130 km/t. I den anden ende, hvis man kører over de 130 km/t vil udslippet igen øges, da bilen er mindre effektiv i udnyttelsen af brændstoffen \cite[s. 5-6]{forurening}.

\vspace{5mm}

En undersøgelse har vist at der er en sammenhæng mellem trafik densiteten, og stress niveauet på en individ der befærder sig i denne trafik. Udover at det kan være ubehageligt under kørslen, bliver stressen også ført med videre på arbejdet og til hjemmet. Stressen kan også føre til aggresiv kørsel og i værste tilfælde ender det med en ulykke \cite[s. 2-3]{stress}.

\vspace{5mm}

For den almindelige bilist, så kan trafikpropper være irriterende at skulle igennem, da det er tidskrævende. Men når det kommer til ambulancernes udrykning, og det kan have fatale konsekvenser for nogle patienter. Falck har oplyst at det koster ambulancerne 1-2 minutter i udrykningstid, når der er trafikprop. Konsekvenserne kan variere alt efter hvor alvorligt syg patienten er, og i værste tilfælde så er konsekvensen menneskeliv \cite{udrykning}.

\section{Initierende problemstilling}

\begin{center}

\textbf{Danske trafik modeller bliver ofte opbygget med et konkret formål i fokus og bliver svære at vedligeholde i fremtiden i de tilfælde hvor konteksten ændrer sig.}
\end{center}
	
\begin{center}
\textit{Arbejdsspørgsmål
\begin{itemize}
\item Hvilke variabler bruges til at simulere trafik?
\item Hvem har gavn af disse trafikmodellerings simulatorer?
\item Hvordan fungerer de forskellige eksisterende software, trafik modellerings simulatorer?
\end{itemize}
}
\end{center}

\section{Eksisterende Modeller}
De modeller der er vedligeholdt og stadig bliver brugt i dag, er meget forskellige i deres fokus. Der findes modeller som Senex, der analysere godstrafikken mellem Danmark og Tyskland, der er en meget advanceret model til trafikafviklingen i hovedstadsområdet, og en masse mindre regionale og kommunale modeller \cite[s. 2]{dtfnotat}. Forskellen på modellerne kan ses på detaljeringsgraden og hvor langt modellen kigger ud i fremtiden, hvor de mindre modeller har flere detaljer, men kun kigger få år ud i fremtiden, og vice versa for de større modeller. Trafikmodellerne er derfor delt op i 3 katagorier; strategiske, taktiske og operationelle modeller \cite[s. 1]{dtfnotat}.

\vspace{5mm}

Strategiske modeller er langsigtede modeller, men med færre detaljer. Manglen på detaljer er påkrævet, da det ellers vil blive for svært at anskaffe data’en, der skal bruges til at specificere alle forudsætningerne for modellens forudsigelser \cite[s. 1]{dtfnotat}. Modeller af denne slags danner et billede over den internationale situation \cite[s. 9]{dtfnotat}. Danmark benytter sig af en strategisk model, Trans-Tools, der blev udviklet i sammarbejde med EU-kommissionen. Formålet med denne model er at forstå konsekvenserne af ændringer i det europæiske vejnetværk. Trans-tools hører også ind under taktiske modeller da den inkorporere detaljer som for eksempel transportmiddelvalg \cite[s. 10]{dtfnotat}.

\vspace{5mm}

Taktiske modeller har i størstedelen af tilfældene et sigte mellem 3 og 20 år. I forhold til de strategiske modeller er detaljerings graden højere. Formålet med disse modeller kan for eksempelvis være at finde ud af hvilke veje er belastede eller hvor lang tid en rejse vil tage \cite[s1]{dtfnotat}. Taktiske modeller bliver brugt til at vise udviklingen i både internationale, nationale og regionale situationer \cite[s. 9]{dtfnotat}. Modellerne der hører herunder er Senex, Storebæltsmodellen og Ørestadstrafikmodellen. Senex bruges til at vurdere tyske lastbilafgifter. Storebæltsmodellen bliver brugt til at vurdere takster og hvordan færgeudbuddet kan påvirke taksterne. Ørestadstrafikmodellen beskriver trafikken i Ørestad, og giver prognoser på hvordan en fremtidig stigning af antal biler vil påvirke vejnettet.

\vspace{5mm}

Operationelle modeller er kortsigtede modeller, og området man undersøger er meget afgrænset. Fordelen ved disse modeller er at den høje detaljerings grad kan give et mere præcist billede over situationen, dog kræver det at der skal bruges en masse data for at resultatet bliver realistisk \cite[s. 1]{dtfnotat}. Herunder har vi kommunale og regionale modeller.

\vspace{5mm}

Udover disse er der mange andre modeller der blev opstillet, men ikke længere er vedligeholdt, herunder har vi eksempelvis Landstrafikmodellen, Hovedstadstrafikmodellen (HTM), national lastbilmodel, trafikafvilkningsmodeller og mange andre \cite[s. 8]{dtfnotat}. Gennem problemanalysen vil en af de vedligeholdte modeller og en af de ikke vedligeholdte modeller blive undersøgt ved hjælp af en teknologianalysen, for at finde ud af hvilke elementer er vigtige. Informationen fra teknologiananlysen kan derefter bruges til at lave en trafikmodel der er nem at vedligeholde.