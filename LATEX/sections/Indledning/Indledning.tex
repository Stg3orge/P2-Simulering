\chapter{Indledning}\label{Indledning}
Trafiksystemet i Danmark undergår ofte udbygninger og ændringer, hvilket kan påvirke nærliggende vejnet. Det kan være svært at se hvordan disse ændringer vil påvirke trafikken, og derfor er der blevet opstillet forskellige modeller til at forudsige hvordan trafikken på vejnettet i fremtiden vil afvikle sig. Disse trafikmodeller bliver ofte opbygget med et konkret formål i fokus, og bliver derfor svære at vedligeholde i fremtiden i de tilfælde hvor konteksten ændrer sig. Konsekvensen af dette er at under en tredjedel af modellerne er blevet vedligeholdt, og at der ikke længere findes en model der dækker hele Danmark \cite[s. 1-2]{dtfnotat}.

\section{Problemets Relevans}
Formålet med trafikmodellerne er at forhindre trafikpropper og at sænke rejsetiderne. Uden modellerne er det besværligt at bestemme hvor der er problemer i vejnettet, og hvilken effekt nye veje vil have på trafikstrømmen. Trafikpropper har en effekt på landets økonomiske vækst, forurening, livskvalitet og tiden det tager for beredskaber og politi at nå frem.

\vspace{5mm}

Man kan risikere at sidde fast i trafikken på vej til arbejdet. For at finde ud af hvilken effekt denne spildtid har på Danmarks økonomi, har Michael Knørr Skov og Karsten Sten Pedersen, der arbejder for konsulent firmaet COWI, analyseret 3 vejprojekter \cite{trafikoekonomi}. Vejprojekterne inkludere en tredje Limfjordsforbindelse, en ny motorvejsstrækning ved København, og en Forbindelse mellem Fyn og Als. Udfra COWI’s beregninger vil disse tilføjelser spare danskere 25 tusinde timer dagligt, hvilket svarer omtrent til en værdi på 2500 millioner kroner årligt. Antager man at en fjerde del af denne tid bliver brugt på arbejde vil man opleve en BNP-vækst på 0,035\%. Et velfungerende vejnet er dermed et vigtigt aspekt i forhold til at forbedre Danmarks økonomiske vækst. 

\vspace{5mm}

Billister er en af de største kilder af CO\textsubscript{2} forurening. Mængden af CO\textsubscript{2} der bliver udsluppet, afhænger af hastigheden billisterne kører. Ved en lav hastighed kan CO\textsubscript{2} udslippet per kilomet blive fordoblet, i forhold til at køre en stabil 50-130 km/t. I den anden ende, hvis man kører over de 130 km/t vil udslippet igen øges, da bilen er mindre effektiv i udnyttelsen af brændstoffen \cite[s. 5-6]{forurening}.

\vspace{5mm}

En undersøgelse har vist at der er en sammenhæng mellem trafik densiteten, og stress niveauet på en individ der befærder sig i denne trafik. Udover at det kan være ubehageligt under kørslen, bliver stressen også ført med videre på arbejdet og til hjemmet. Stressen kan også føre til aggresiv kørsel og i værste tilfælde ender det med en ulykke \cite[s. 2-3]{stress}.

\section{Initierende problemstilling}

\begin{center}

\textbf{Danske trafik modeller bliver ofte opbygget med et konkret formål i fokus og bliver svære at vedligeholde i fremtiden i de tilfælde hvor konteksten ændrer sig.}
\end{center}
	
\begin{center}
\textit{Arbejdsspørgsmål
\begin{itemize}
\item hvilke ændringer i konteksten skal der tages højde for ved trafik simulering
\item Hvordan fungerer de forskellige eksisterende software, trafik modellerings simulatorer?
\item Hvem har gavn af trafikmodellerings simulatorer?
\end{itemize}
}
\end{center}