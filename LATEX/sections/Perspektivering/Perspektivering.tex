\chapter{Perspektivering}\label{Perspektivering}
Vi har igennem programmet benyttet os af Window Forms for at kunne fremstille et GUI, som brugeren kan benytte sig af. Problemet er bare at når programmet skal tegne simuleringen, og når glitteret skal tegnes, og dette bliver gjort hver gang noget rykker sig på glitteret. Dette bliver gjort på cpu'en, hvilket bruger meget kraft og ressourcer. Istedet kunne man taget Windows Presentation Foundation (WPF) i brug, da WPF benytter sig af GPU'en når der skal tegnes, og så skal processeren kun tænke over det input der kommer fra brugeren. Dette vil kræve de samme resourcer at lave en simulering, da denne kører på cpuen. det vil dog gøre at opdateringer af grafik bliver hurtigere og ikke påvirker responstiden fra input lige så meget.


\vspace{5mm}

Vores program ville være bedre hvis vi benyttede bedre multithreading, da vores program kører på to tråde ved simuleringsdelen. Vi bruger kun 50 procent af en CPU på 4 kerner og 25 procent på en med 8 kerner. Hvis vi havde håndterert at kunne benytte 100 procent af en cpu, ved f.eks. bedre multithreading såsom et variabelt antal tråde der var igang, så vi fuldt ud benyttede af de ressourcer CPU'en har.

\vspace{5mm}

En anden måde kunne være at man kunne optimere processen bag simulering, ved at sætte GPU'en til at beregne simulering beregningerne.

\vspace{5mm}

Når vores program kører i simulationen, så bilerne som der er i simulationen kender de ikke noget til de sideliggende veje, ergo de kender ikke noget til de modkørende. Det er grunden til at bilerne ikke har muligheden for at overhale hinanden. Vejene er heller ikke sammensat, så der er plads til at kunne overhale. Så hvis der er en bil kørende bag en der kører langsommere, så vil han sætte farten ned og blive bag ved bilen.

\vspace{5mm}

Brug af protobuff?

\subsection{Analysering af simulationsoutput}
På nuværende tidspunkt er simuleringsprogrammet ikke i stand til at give et tekstbaseret output, hvor man kan se hvordan implantationen af en ny strækning vil optimere på trafikken, eller hvor det vil være mest optimalt at implementere en ny strækning, for at optimere trafikken mest muligt. Det næste skridt i udviklingen af programmet, vil være at implementere en beregningsalgoritme, som ikke blot kan beregne, men også analysere, hvor det er mest optimalt at placere en vej, for at effektivere trafikken. Outputtet skal selvfølgelig være tekstbaseret, eftersom det vil være besværligt, for ikke at sige umuligt, at beregne blot 