\chapter{Perspektivering}\label{Perspektivering}
\section{Analysering af simulationsoutput}
På nuværende tidspunkt er simuleringsprogrammet ikke i stand til at give et tekstbaseret output, hvor man kan se hvordan implantationen af en ny strækning vil optimere på trafikken, eller hvor det vil være mest optimalt at implementere en ny strækning, for at optimere trafikken mest muligt. Det næste skridt i udviklingen af programmet, vil være at implementere en beregningsalgoritme, som ikke blot kan beregne, men også analysere, hvor det er mest optimalt at placere en vej, for at effektivere trafikken.
\section{Testing}
Som gennemgået i tidligere afsnit vedrørende testing så implementerede vi kun den funktionelle testing metode Unit Testing. Programmet er ikke udviklet i et \textbf{TDD}(Test-driven development) format. Problematikken ved ikke at udføre denne udviklingsmetode giver mulig anledning til fejltagelser ved kodeimplementationen. \textbf{TDD} er en udviklingsmetode orienteret omkring udvikling gennem testing. Ideen ligger i at kode tests før selve kildekoden bliver implementeret. Fordelen ved dette er at man går ind i kode-delen med en præcis viden om hvad metoden skal gøre.\cite{unittestbenefits}
Der ville have været mulighed for mere gennemgående testingformater, havde programfasen været udviklet i dette format.

\vspace{5mm}

Efter nøjere overvejelser over andre eksisterende testing metoder såsom Usability Testing og System Testing er der vurderet, at i forhold til at assistere løsningsmodellen for problemformuleringen, kunne have været ideelt at udføre en Usability Test på programmet. Testen handler i kort forstand en ren brugerbaseret interaktionstest. Brugeren benytter programmet og deres respons bliver taget til behandling af udviklerne.\cite{usability} Da dette projekts program skulle være brugervenligt ville dette være en ideel testmetode at udføre.