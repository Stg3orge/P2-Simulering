\chapter{Abstract}\label{abstract}
This report is written for the 2nd semester project of group A319. The theme of the work done and described within this report is "Simulering" and shows the extensive work there has been done on analyzing and idenfitfying a problem within the field of simulations as well as describe the development of a solution to said problem. The report is split into two parts, the problem analysis in which the process of going from an initial problem-statement to a descriptive final problem-statement is documented. Throughout this part, the different aspects of the problem field is described, including the current existing solutions there exists today. Among those are VisSim, a solution that the group's own take of a solution was mostly inspired by. Key organizations and people who has a significant stake in the solution that has been developed has also been identified as part of this project. Ultimately, the group found that the problem that needed solving was that some existing solutions were made for very specific problems, however even when this was the case, sudden changes in the context of the field those solutions were for, could cause potential loss of data or invalid data for simulations. It was decided to work on a solution that could adapt to a change within a set context.

\vspace{5mm}

The second part of the report describes the developed solution and the the group's decisions that were taken when making the solution. Requirements and success-criteria were made according to the findings made in the problem analysis and the final problem-statement. Furthermore, this part of the report compares two shortest-path algorithms the group looked at for implementation of the program, A* and Dijkstra, and also details the decision made on this matter. Most importantly, this part also includes detailed descriptions of some of key parts of the developed program, as well as snippets of the sourcecode for these parts. A classdiagram is also shown and described to provide a overview of the program and it's different classes. Finally, this chapter is rounded off by a detailed explanation of some of the features that are lacking from the program. This is then take to a discussion about the program and whether or not it can be considered a  success, which then leads to a part about possible future changes to the solution as well as a conclusion to round off the report.
