\section{Testing}\label{Testing}

\textbf{Unit Test} \newline
En unit er det minste testdel af et program, altså det kan være funktioner, klasser, procedure eller interfaces. Unit testing er en metode man kan benytte når man vil teste hver inviduel del af programmet virker og om de er egnet til brug. Unit tests er skrevet og benyttet af programmører, for at være sikker på at ens kode opfylder de krav som er forventet af det. 

\vspace {5mm}

Formålet med denne test er at splitte programmet op i mindre dele, hvor derefter man tester en efter en at de forskellige dele fungere optimalt og som de skal. Det kan være en funktion, som man vil teste, hvor man har nogle input og derefter skulle funktionen have de rigtige output fra funktionen. På den måde kan man håndtere fejl når det forkerte input er givet[2].

\vspace {5mm}

Fordelene ved unit testing:
\begin{itemize}
\item Problemer/fejl findes tidligt, så det ikke påvirker senere kode.
\item Unit testing hjælper med at vedligeholde og nemt kan ændre koden.
\item Opdagelse af bugs tidligt, hvilket hjælper med at reducere omkostningerne når man skal fejlrette koden.
\item Unit test hjælper med at forenkle debugging processen, så hvis der sker en test fejl, så er det kun de seneste rettelser der skal rettes.
\end{itemize}

\vspace {5mm}

Vi har kigget på andre former for testing af et program, og fandt frem til en testing metode kaldt Blackbox testing. Grunden til at vi ikke benyttede os af denne metode er, fordi den kræver at vi har en person som skal  teste programmets bruger grænseflade og hvordan programmet fungerer. Den person kender ikke til kodens struktur og hvordan det er blevet programmeret. Vi i gruppen er kommet frem til at Unit testen er en bedre og mere effektiv løsning til at teste vores program igennem. Da vores program benytter sig meget af forskellige klasser, metoder og interfaces, så mente vi som gruppen at det rette fremgangsmåde er at teste de forskellige dele i vores program del efter del. Da vi kan give input til programmet, og så finde ud af om det rigtige output kommer ud.



