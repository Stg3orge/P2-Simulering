\chapter{Design af program}
Dette afsnit har til formål at udvikle et program, som imødekommer kravet om vedligeholdelse. Afsnittet vil sætte nogle kravspecifikationer og succeskriterier. Kravspecifikationerne er dem som skal forme programmet og samtidig gøre programmet realistisk. Formålet med programmet skal være at brugeren selv skal kunne indstille det efter sine behov. Programmet skal sættes op således det er TrafikTeknikere der kan anvende det og sammenligne to simuleringer, for at finde den mest optimale løsning til et vejnetværk i den realistiske verden. 


%\subsection{Implementation af vejsystemer}
% i programmet er at gøre det muligt, at lave et vejsystem der kan bruges til at simulere forskellige scenarier i trafikken og trafikmønstre, således at det er muligt at optimere trafikken for motoriseret køretøjer. Eftersom det kun er optimeringen af motoriseret køretøjer, med tilladelse til at køre på alle de danske veje (derfor ses der bort fra køretøjer som bl.a. scootere og traktorer, da de er begrænset til visse typer af veje), bliver der ikke taget højde for bl.a. fodgængere og cyklister. Dette giver dog en fejlmargin, da disse også påvirker trafikanterne når de eksempelvis skal passere hinanden. Men da det er begrænset i hvor stort et omfang de påvirkes, vælges der at se bort fra disse faktorer, primært fordi det hovedsageligt kun påvirkes under bykørsel. 

\subsection{Simulering af trafik}
Idéen ved programmets simulering, er at det skal være muligt for brugeren at se, hvor det vil være mest optimalt at implementere nye ruter for at optimere trafikken og undgå eventuelle trafikpropper og problemer som ofte skyldes ruter der er stærkt trafikeret. Brugeren vil altså få mulighed for at kunne ændre på variablerne såsom hvor vejen skal implementeres, men antallet af trafikanter der benytter en vilkårlig vej samt deres hastighed. På denne måde vil det altså være muligt for brugeren at se hvor meget trykket ville kunne lettes for problematiske ruter, ved at tilføje alternativer, således at det er muligt at vurdere om det ville være omkostningerne værd at implementere. 

\subsection{Opbygning af vejnet ved brug af grid}
Vejnettet opbygges i et grid, koordinatsystem, som gør det muligt at tilføje forbindelsespunkter, der ved hjælp af koordinaterne kan beregne afstanden imellem hvert punkt, node, således at det er muligt at klarlægge den totale afstand en vilkårlig rute har. Den anvendte algoritme der benyttes i programmet kan så efterfølgende beregne den kortest mulige rute, for en vilkårlig trafikants start destination og slut destination, som det forventes brugeren benytter. Det vil så efterfølgende være muligt for brugeren selv at ændre på denne rute, såfremt det ikke forventes at trafikanten benytter den mest korteste rute.

%\subsection{eventuelt perspektivering?}
%Som det blev nævnt er det op til brugeren at tegne eksempelvis et lyskryds hver gang det skulle implementeres, hvilket med større typer lyskryds kan værre et større opgave. Såfremt man ville have et identisk lyskryds ville brugeren altså selv skulle lave det hver gang, hvortil en copy-paste funktion ville gøre det lettere for brugeren at skabe større vejnet, uden at det vil kræve et stort arbejde, blot for at lave hvad der allerede havde været udført tidligere. Men en copy-paste funktionen vil også gøre fremkalde et nyt problem, når først brugeren har afsluttet det igangværende projekt. Derfor vil en optimal løsning være at gøre det muligt for brugeren at gemme dele af et projekt, og give dem muligheden for at gemme typer af vejnet til senere brug. 