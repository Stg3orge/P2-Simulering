\chapter{Diskussion}
\externaldocument{Teknologianalyse}
\externaldocument{konteksten}
\externaldocument{FejlOgMangler}
Dette afsnit diskuterer hvorvidt den udviklede løsning, er en løsning på den opstillede problemformulering, samt afklarer om succeskriterierne i design afsnittet er opfyldt. Til sidst i afsnittet vil der diskuteres om hvorvidt vores program er realistisk.

\subsection{Diskussion af problemformuleringen}\label{diskproblem}
Som det første der beskrives i problemformuleringen, har vi at nuværende simuleringsværktøjer enten ikke er brugervenlige, eller at de ikke er fleksible. Den udviklede løsning minder om værktøjet VisSim, der blev analyseret i afsnit \ref{Teknologianalyse}, i det at programmet giver brugeren mulighed for at opstille et vilkårligt vejnetværk, og ændre på en række indstillinger for simuleringen. Det der gør gruppens værktøj anderledes fra VisSim er at det ikke er muligt at opstille andre simuleringer end trafik afviklings simuleringer. Denne afgrænsning betyder at VisSim er langt mere fleksibel, men det er også denne fleksibilitet der gør at VisSim ikke er brugervenlig. Udover afgrænsningen, er den udviklede løsning gjort brugervenlig ved hjælp af værktøjstips og etiketter, hvor brugeren kan læse om de forskellige funktioner af knapper og indstillinger. På grund af fokuset på trafik afvikling og informationerne der befinder sig i programmet, vurderer vi at gruppens løsning er mere brugervenlig end VisSim.

\vspace{5mm}
Problemformuleringen spørger dernæst hvordan et mesosimuleringsværktøj kan optimeres i forhold til vedligeholdse. I afsnit \ref{AenderingerAfKonteksten}, der omhandlede ændringerne i konteksten, fandt vi at antallet af køretøjer var stigende, og at der skete ændringer i befolkningens valg af transportmidler. I gruppens program er det muligt at indstille antallet af køretøjer der skal simuleres, og programmet kan dermed tilpasses til stigningen af antallet af køretøjer. Det er også muligt for brugeren at opsætte forskellige køretøjs-typer, og indstille hvor mange af køretøjerne skal have de forskellige typer. Det er ikke muligt i programmet at simulere cyklister, hvilket betyder at i tilfælde hvor transportmiddelvalget for en del af befolkningen ændrer sig til cykler, vil man ikke kunne vise dette i programmet, udover at der vil være færre køretøjer på vejnettet. 

\vspace{5mm}
Problemformuleringen beskriver også at der skal tages hensyn til billisternes adfærd, hvilket ikke bliver opfyldt af programmet. I programmet køre køretøjerne den maksimale hastighed, med mindre at køretøjer foran køre langsommere. I den virkelige værden svinger den ønskede hastighed fra billist til billist, og en implementation af adfærden ville dermed gøre simuleringen mere virkelighedsnær.

\vspace{5mm}
Samlet set er det gruppens mening at programmet er en løsning på problemformuleringen, i det at programmet blev videreudviklet og funktionalitet som cyklister og menneskelig adfærd blev implementeret.

\subsection{Succeskriterier}\label{disksucces}
Alle succeskriterierne er blevet løst udover kriterien om køretøjernes bevælgse skal gøres realistisk med hensyn til acceleration og deceleration. I gruppens program er acceleration og deceleration ikke blevet implementeret. Hvis et køretøj eksempelvis støder på et rødt lys, vil køretøjet stoppe øjeblikkeligt. Ligeledes når lyset skifter til grønt igen, vil køretøjet nå sin maksimale hastighed øjeblikketligt. Grunden til at acceleration og deceleration ikke er blevet implementeret, er at vigepligt og trafiklys blev prioriteret højere. Konsekvensen af denne mangel, er at køretøjerne vil bevæge sig gennem vejnettet hurtigere end det burde være muligt.

\subsection{Realisme}
For at programmet skulle kunne bruges til at lave beslutninger, kræves det af programmmet, at resultatet af simuleringen er realistisk. Udover problemerne beskrevet herunder, er der også manglen på acceleration, deceleration og menneskelig adfærd som blev beskrevet i afsnit \ref{diskproblem} og \ref{disksucces}.

\begin{itemize}
\item Køretøjerne i simuleringen kan lave uvendiger i alle lyskryds, dette er  ikke realistisk, da det ikke er lovligt i alle lyskryds.
\item Vejnetværket er opstillet såleves en vej er et spor for en bil, dette gør også programmet urealistisk, da en vej kan indholde flere spor.
\item Køretøjerne kan ikke overhale andre køretøjer, hvilket også er betydelig del af programmet.
\item I programmet kan brugeren opstille et tidsrum hvor køretøjerne kører fra deres hjem til destinationen, og et tidsrum hvor der bliver kørt hjem igen. Dette fungerer ved at brugeren sætter et bestemt tidspunkt, og derefter indstiller hvor meget køretøjerne må afvige fra dette tidspunkt. Denne metode er urealistisk, da brugeren ikke kan specificere at der skal være meget trafik ved 8 tiden, og sammentidigt have en lav mængde trafik om natten og midt på dagen. Det ville være en forbedring, hvis brugeren kunne indstille procentvis antallet af køretøjer for mindre tidsrum gennem en dag, eksempelvis fra klokken 8 til klokken 9.
\end{itemize}

\vspace{5mm}
Dog er programmet realistisk på synspunkter så som, vigepligt, lyskryds, køretøjerne ikke køre ind i hinanden, køretøjerner holder en afstand til de andre køretøjer og hastighedsgrænser. 