\chapter{Diskussion}
Dette afsnit gennemgår de tanker gruppen har gjort sig i forhold til forskellige synspunkter af rapporten. Problemformuleringen er blevet løst ved hjælp af et software værktøj i form af meso simulering, i dette afsnit vil der diskuteres om denne problemformulering er blevet løst. 

\vspace{5mm}

\section{Diskutering af problemformuleringen}
Der er tidligere beskrevet i problemformuleringen at \textit{Nuværende simuleringsværktøjer til simulering af trafik er enten for svære at anvende eller mangler fleksibilitet}. Programmet som gruppen har lavet har løst dette ved at brugeren ikke selv skal programmere sit vejnetværk ligesom i VisSim, men brugeren skal indsætte veje, huse, destinationer osv. Dog skal brugeren også bestemme enkle variabler i form af køretøjets hastighed, procentvisdeling af destinationer osv. I modsætning til VisSim, så vil vores værktøj være mere overskueligt at anvende, dog er vores programet ikke er lige så præcis som VisSim, fordi værktøjet ikke anvender acceleration og deceleration, dette var en central del af programmet og er ikke blevet implementeret. Fleksibiliteten vil dog menes at være nogen lunde det samme, fordi brugeren selv skal opstille sit vejnetværk i begge programmer. Derudover er vores program mere fleksibelt end Altrans, fordi man ikke kan ændre på vejnetværket i Altrans. Altrans kan heller ikke visualisere simuleringen. Derimod kan Altrans vise en række data af simuleringen, som vores program ikke kan. Dette medføre at vores program er mere fleksibelt end Altrans, men brugeren har ikke noget data at sammenligne med.

\vspace{5mm}

Problemformuleringen nævner også optimering i forhold til vedligeholdelse, trods ændringerne i konteksten se afsnit REFHER. Dette er blevet løst ved at brugeren selv kan ændre konteksten, ved at indtaste forskellige variabler, til diverse funktioner i programmet, så som transportmiddelvalg. Her kan man ændre antallet af køretøjet, hastigher osv. I modsætning til Altrans, hvor Altrans ikke kan ændre konteksten, hvilket betyder at værketøjet ikke kan vedligeholdes i forhold til konteksten.

\vspace{5mm}

\section{Realitisk simulering}
En af programmets formål var at det skulle være realistisk, dog har programmet nogle fejl, som der er beskrevet i afsnit \ref{FejlOgMangler}. Programmet implementere ikke deceleration og acceleration, som var en central del af simuleringen, da dette ville gøre programmet realistisk. Derudover kan køretøjerne lave uvendiger i alle lyskryds, dette er heller ikke realistisk, da det ikke er livligt i alle lyskryds. Vejnetværket er opstillet såleves en vej er et spor for en bil, dette gør også programmet urealistisk, da en vej kan indholde flere spor. Køretøjerne kan heller ikke overhale andre køretøjet, hvilket også er betydelig del af programmet. 

\vspace{5mm}

Programmet tager heller ikke højde for menneskelig adfærd, fodgænger og cyklister. Dette er også centrale dele, som vil påvirke trafikken. Køretøjerne køre deres max hastighed på vejene, hvilket betyder at der ikke er menneskelig adfærd i programmet. Dog burde der være variation af bilisternes adfærd, det er feks. ikke alle bilister som overholder hastighedsgrænsen, eller alle bilister som køre præcis hvad de må køre. 

\vspace{5mm}

Dog er programmet realistisk på synspunkter så som, vigepligt, lyskryds, køretøjerne ikke køre ind i hinanden, køretøjerner holder en afstand til de andre køretøjer, hastighedsgrænser osv. Derudover kan brugeren opstille et tidsrum, som er meget trafikeret hviklet gør programmet realistisk, da disse tidsrum også forekommer i traffikken idag. 


