\chapter{Diskussion}\label{Diskussion}
Det udviklede program er udviklet med den tilstræbelse, at få opfyldt et eller flere af de opstillede succeskritierer inde for hvad gruppen har været i stand til givet de faglige og tidsmæssige resourcer der er blevet sat af til projektet. Dette afsnit belyser på de tanker gruppen har gjort sig i forhold til succeskritierene post-løsningsudvikling.

\subsection{x.xx Succeskriterier}

\textbf{1. Brugeren skal være i stand til at opsætte et vejnet, der indeholder objekter som trafiklys, huse, parkeringpladser og destinationer.}

Når programmet åbnes bliver brugeren præsenteret med et grid hvor på knudder og veje kan indtegnes samt andre objekter. Dette gør brugeren i stand til at opsætte deres eget vejnet, både simpelt eller avanceret efter deres eget behov. Det er muligt at forbinde veje til forskellige knuder og give disse veje og knuder forskellige egenskaber. Derudover kan der indsættes destinations objekter på gridded som kan laves til at modtage en hvis procentsats af alle bil objekter der opstår igennem simuleringen. Lightcontrollere kan tilkobles knuder til at simulerer et trafiklys opførsel.

\textbf{2. Det skal være muligt for brugeren at kunne gemme deres arbejde, lukke programmet og forsætte deres arbejde næste de åbner programmet.}

Når brugeren åbner programmet, kan de navigerer til dropdown menuen file hvor de kan vælge at gemme deres nuværende fil, åbne en fil, samt at oprette en ny fil. Dette gør brugeren i stand til at gemme deres arbejde og åbne det igen på et senere tidspunkt eller sende til en anden bruger med programmet.

\textbf{3. }
