\section{Eksisterende Modeller}
De modeller der er vedligeholdt og stadig bliver brugt i dag, er meget forskellige i deres fokus. Der findes modeller som Senex, der analysere godstrafikken mellem Danmark og Tyskland, der er en meget advanceret model til trafikafviklingen i hovedstadsområdet, og en masse mindre regionale og kommunale modeller [14, s. 2]. Forskellen på modellerne kan ses på detaljeringsgraden og hvor langt modellen kigger ud i fremtiden, hvor de mindre modeller har flere detaljer, men kun kigger få år ud i fremtiden, og vice versa for de større modeller. Trafikmodellerne er derfor delt op i 3 niveauer; makro-, meso- og mikroniveau.

\vspace{5mm}

Makroniveau
På makroskopiske niveau er de anvendte modeller langsigtede, men med færre detaljer. Manglen på detaljer er påkrævet, da det ellers vil blive for svært at anskaffe data’en, der skal bruges til at specificere alle forudsætningerne for de anvendte modellers forudsigelser [14, s. 1]. Modeller på makroniveau danner et billede over den internationale situation [14, s. 9]. En af de anvendte modelmetoder på dette niveau er prognosemodeller, disse benyttes til at beregne fremtidens trafik behov, således at der kan planlægges hvordan ressourcerne kan anvendes.(kilde speciale.)


\vspace{5mm}

Mesoniveau
På det mesoskopiske niveau er detaljerings graden højere i forhold til makroniveau. De anvendte modeller på dette niveau er 4-trinsmodeller. Disse modeller bliver anvendt til at forudse trafikkens bevægelsesmønster.(speciale) De bruges også til at finde ud af hvilke veje der er belastede eller hvor lang tid en rejse vil tage. Disse modeller bliver brugt til at vise udviklingen i både internationale, nationale og regionale situationer [14, s. 9].

\vspace{5mm}

Mikroniveau
På det mikroskopiske niveau er det kortsigtede modeller der bliver anvendt. Det område man undersøger er meget afgrænset. Fordelen ved disse modeller er at den høje detaljerings grad kan give et mere præcist billede over situationen, dog kræver det at der skal bruges en masse data for at resultatet bliver realistisk [14, s. 1]. En af de anvendte moddelleringsmetoder er den empiriske metode, som bygger på observationer. Her er det enkeltstående situationer der bliver observeret, for at se hvilket udfald simuleringen har. På det mikroskopiske niveau kan det evt. være rundkørsler, lyskryds, eller en enkel vej der bliver analyseret.(speciale)

\vspace{5mm}

Gennem problemanalysen vil en af de vedligeholdte modeller og en af de ikke vedligeholdte modeller blive undersøgt ved hjælp af en teknologianalysen, for at finde ud af hvilke elementer er vigtige. Informationen fra teknologiananlysen kan derefter bruges til at lave en trafikmodel der er nem at vedligeholde.