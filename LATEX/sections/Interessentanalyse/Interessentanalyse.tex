%!TEX root = ..\..\Main.tex
\chapter{Interessentanalyse}\label{Interessentanalyse}

Metode:
Beskrivelse
Analyse:
Brug af 2 modeller.


Transport- og Bygningsministeriets (trm) - (Vejdirektoratet):
Transport- og Bygningsministeriet(trm) er Danmarks øverste danske statslige myndighed på transportområdet og bygningsområdet. Trms hovedopgave er at sikre sig at de forskellige love bliver overholdt, ved opførelse af fx. en motorvej. Dog da trm er en sammensætning af mange underdelinger har vi valgt at fokusere på en af deres styrelser nemlig Vejdirektoratet.
 
Vejdirektoratet står nemlig bag statsvejnettet som hovedsageligt består af motorveje, hovedlandeveje, og mange af landets broer (i alt 3.801 km vej). Det udgør med andre ord ca. 5 \% af det offentlige vejnet. Men selvom Vejdirektoratet kun står får 5 \% af vejnettet, så udgør disse 5\% ca. halvdelen af Danmarks trafik.

Grunden til Vejdirektoratet kunne være interesseret i vores simulering ses på deres primære opgaver. Vejdirektoratet står nemlig for planlægningen af vejnettet så både privatbilismen og den kollektive trafik kan fungere sammen. Samt at de anlagte vej vil fungere i samspil med de allerede eksisterende veje.
Vejdirektoratet vil være interesseret i vores simulering hvis den kunne gå hen og hjælpe med at forudsige placering af nye veje, så den virker i sammenspil med det allerede eksisterende vejnet. Altså at simulering kunne vise hvordan trafikken vil blive påvirket hvis der blev opført en vej.

Kommunen:
Danmark har 98 kommuner som er delt over Danmark. Derfor står hver enkelt kommune for de faktorer som påvirker kommunens areal, det kunne f.eks. være vedligeholdelse af veje. Kommunerne er derfor også midtpunktet i trafikprojekter vedrørende deres areal. De er derfor en vigtig interessant, da de ofte er forbindelserne mellem de forskellige organisationer, f.eks. mellem Vejdirektoratet og Trafik- og Byggestyrelsen. Hvor Vejdirektoratet ofte står for planlægningen og Trafik- og Byggestyrelsen står for opførelsen. Kommunen er derfor ofte involveret i alle dele af projektet.

Vi har valgt at tage Aarhus kommune som eksempel, da de ligger centralt i Danmark og derfor samler mange vejnet sig omkring Aarhus, derudover har Vejdirektoratet målt trafikudviklingen over mange år, hvor det viser sig at Aarhus er blandt nogle af de værst ramte kommuner, og vil derfor have en af de højeste trafikudviklinger i de kommende år. 


Aarhus har i mange år prøvet at løse trafikproblemer i form af grøn trafik, altså de har prøvet at tilgodese cyklisterne med direkte og trygge ruter, da de mener det vil det være muligt at overflytte biltrafik til cykeltrafik ved at skabe tryghed og mobilitet. Derudover har de prøvet at ændre bilisternes adfærd, ved at få flere til at benytte kollektiv trafik. Det har bl.a. ført til opførelsen af Danmarks første letbane som bliver køreklar i 2017. 

Aarhus kommune kunne altså være interesseret i vores projekt hvis det kan hjælpe dem med at forudse de ændringer i opførelser som f.eks. en letbane har for trafikken. Så letbanerne kan placeres det mest optimale sted, Aarhus og Favrskov kommune har allerede planer om 2 mere letbaner.

Kollektiv Trafik - (Nordjyllands Trafikselskab):
Kollektiv trafik har i de sidste mange år været noget som Danmark har haft fokus på, da Danmark gerne vil fremstå som et grønt land. Bare i de seneste år har de offentlige buskørsel i Danmark haft en påstigning på ca. 350 mio. I 2015 er det blevet fordelt på ca. 13408 busser. 

Vi har derfor valgt Nordjyllands Trafikselskab (NT) som eksempel på en kollektiv trafik organisation. Da de står for størstedelen af kollektiv trafik i nordjylland, fx. styrer de alle offentlige busser, og toge i Nordjylland. NT ændre ofte deres bus ruter, hvor mindskelse af busstoppesteder ofte er en prioritet, så bussen kan komme fra A til B hurtigst muligt. Derfor ville organisationer som NT være interesseret i værktøjer som kunne hjælpe med at mindske trafikken. Herved vil NT kunne tilbyde hurtigere transport, samt bedre mobilitet.

Forsvarsministeriet(fmn) - (BeredskabsStyrelsen):
Forsvarsministeriet har mange under departementerne hvor vi har valgt BeredskabsStyrelsen som eksempel. BeredskabsStyrelsen er blevet valgt da de samarbejder bl.a. med de kommunale redningsberedskaber, politiet og andre myndigheder og står derfor også for mange af de udrykninger som forekommer. Hvor mange af disse udrykninger kan have svært ved at komme frem til destinationen, da der kan være opstået trafikpropper.

BeredskabsStyrelsen vil altså kunne have interesse i vores projekt, da BeredskabsStyrelsen overordnet mål er komme frem til destinationen hurtigst muligt, for at redde så mange liv som muligt. 
Transportfirmaer - (GLS):
I 2015 blev ca. 28628 lastbiler registreret i Danmark (ca. 1\%), samt blev der kørt ca. 988 mio. km i Danmark for de første 3 kvartaler af 2015. Der kører altså en del transportmidler som fx. lastbiler på de danske veje. Transportindustrien kunne altså derfor have interesse i forbedring af trafikproblemer. Da deres overordnet mål er at nå til destinationen til den aftalte tid, som eksempel har vi valgt GLS. GLS er en af Danmarks største pakkedistributører, og derfor er de naturligvis afhængig af vejnettet. GLS har som interesse at aflevere pakken til modtageren så hurtig som muligt. GLS vil derfor kunne se fordele i at vejnettet bliver mere forudset så de kan levere pakken inden for det aftalte tidsrum.

Links:
%https://www.dst.dk/da/Statistik/emner/transport/transportmidler?tab=nog
%https://www.dst.dk/da/Statistik/emner/transport/godstransport-med-lastbil
%https://gls-group.eu/DK/da/gls-group/vision-vaerdier

Alm borger:
Vi vil i denne sammenhæng se mindre virksomheder som fx. det lokale pizzabud som en del af kategorien alm. borger.

I 2015 blev der registreret 2.329.578 personbiler i Danmark. Det udgør altså ca. 55\% af køretøjerne på vejnettet.
Den alm. borger tager ofte bilen til/fra arbejde, da de ofte mener at bilen giver mere frihed i form af mindre transporttid sammenlignet med fx. kollektiv trafik. Den alm. borger vil derfor være interesseret i at mindske transport tiden, dette vil kunne gøres ved optimering af vejnettet.


Links:
%https://www.researchgate.net/publication/277074441_Det_er_de_andre_der_korer_beboernes_holdninger_til_trafik_og_levevilkar_i_Kgs_Enghave

%(GOD) http://samf.ku.dk/pkv/faerdige_projektopgaver/191/191_samlet_pdf_til_web.pdf
%https://www.dst.dk/da/Statistik/emner/transport/transportmidler?tab=nog


Taxi/privatkørsel - (TaxiNord 4x48):
I 2013 fandtes der ca. 4800 taxier i Danmark. Privatkørsel som fx. taxier har har altid skulle fremstå som en "bedre service" i forhold til kollektiv trafik. Derfor kunne privatkørseler også have interesse i optimeringen af trafik.
Vi har her valgt TaxiNord 4x48 da de som det eneste taxiselskab dækker hele Nordsjælland og Storkøbenhavn. 4x48 kunne være interesseret da de som virksomhed gerne vil kunne give privaten den bedste service, hvor taxaen kommer til tiden, samt med brug af mindst muligt brændstof så prisen er så lille som muligt.

Taxibranchen omsatte for 5,129 mia. kr. i 2007. I samme år omsatte trafikselskaberne, det vil sige eksempelvis busselskaber, for 5,566 mia. kr.

Links:
%http://www.taxinord.dk/nyttig-info/om-4x48.html
%http://www.amagerobrotaxi.dk/Downloads/Holdepladsen/Holdepladsenjuni2015.pdf
%http://www.taxi.dk/Default.aspx?ID=266&PID=1074&Action=1&NewsId=551




Modeller:
Model 1:
Til at hjælpe med at placere de forskellige interessenter i forhold til vigtighed for projektet har vi valgt at kategorisere de forskellige interessenter i 2 akser, fordelt i 4 grupper. Op af x-aksen har vi interesantes indflydelse på projektet, og y-aksen er interesantes egen påvirkelse af projektet.

De 4 grupper vil blive forklaret ud fra eksemplet “opførelse af Aarhus nye letbane i 2016”

Ekstern interessent: (Orienteres)
Denne gruppe beskriver de intressenter som har har meget lille/ingen indflydelse på projektet, samt dem som bliver påvirket af projektet i mindre grad. Den eksterne interessent skal altså bare orienteers, og vide at projektet eksisere. 

Det kunne f.eks. være den alm. borger som ikke befinder sig i Aarhus egen. Her skal borgeren orienteres omkring opførelsen af Aarhus nye letbane, så hvis borgeren engang kom til Aarhus vil de vide den er der til benyttelse. Borgeren har altså ikke indflydelse på projektet, men borgeren skal informers omkring projektet.

Gidsel: (Informeres)
Denne gruppe indeholder de intressenter som bliver påvirket af projektet, men kun har lidt/ingen indflydelse på projektet. De skal altså informeres.

Det kunne f.eks. være den alm. Borger som ofte/altid er i Aarhus omegen. De skal orienteres omkring opførelsen af Aarhus nye letbane, da de kan forvente trafik under opførelses perioden. Derudover vil letbane påvirke borgerne i Aarhus omegn, da det vil have stor indflydelse på de andre kollektive transportformer i forhold til afgangstider og ruter. Det kan være at busruter bliver ændret så der opstår et bedre sammenspil mellem letbanen og bussen.

Grå eminence: (Høres)
Gruppen her beskriver de intressenter som bliver påvirket af projektet i mindre/ingen form. De har dog indflydelse på projektet, og det er derfor vigtigt at de høres, og deres interesse forbliver positiv.

Her kunne det f.eks. Være Transport- og Bygningsministeriet (trm). Trm vil ikke blive påvirket af opførelse af en letbane, men de har en stor indflydelse på hvordan, og hvor den skal opføres i Aarhus, da de skal godkende opførelsen. Det er altså derfor vigtigt at man her inddrage trm under projektet, så når det skal godkendes vil det gå nemmere igennem.

Ressource person: (Involveres)
Denne gruppe er den vigtigste gruppe. Ressource personerne er dem som har stor indflydelse på projektet, samt bliver påvirket af det. Det er altså her vigtigt at de inddrages i alt, da de har en stor interesse i projektet, og deres interesser skal bibeholdes.

Det kunne f.eks. være den kollektive trafik (midttrafik), som skal informeres omkring Aarhus letbane, så de planlægge deres ruter, samt bustider til at passe sammen med letbanes tidere. Så brugeren skal vente så lidt som muligt. Derudover vil trafikselskabet Midttrafik skulle stå letbanens køreplan samt billet. Midttrafik vil altså i dette tilfælde skulle informeres, og vil derfor være en ressource person.


Intersanterne indført i modellen er vist nedenfor.


\iffalse \chapter{Interessentanalyse}\label{Interessentanalyse} %

Dette kapitel vil afdække nogle af de interessenter som ville have et incitament til at se et program, der kan simulering en optimering af trafikken udviklet, samt de interessenter som vil have et incitament til at hindre eller miskrediterer udviklingen af dette. Der vil blive udvalgt instanser til at repræsentere de relevante interessenter til at afspejle et generelt billede i samfundet.

\section{Kollektiv trafik}\label{Kollektiv-trafik}

Dette afsnit har til formål at give et generelt billede af hvordan et simuleringsprogram vil være behjælpelig eller påvirke den kollektive trafik samt om det har et incitament til at hindre eller støtte udviklingen af programmet.

\subsection{Nordjyllands Traffikselskab (NT)}\label{Nordjyllands-Traffikselskab}



\section{Erhvervskørsel}\label{Erhvervskoersel}

Dette afsnit har til formål at give et generelt billede af hvordan et simuleringsprogram vil være behjælpelig eller påvirke erhvervskørsel samt om det har et incitament til at hindre eller støtte udviklingen af programmet.

\subsection{Transport og fragt}\label{Transport-og-fragt}

Dette afsnit har til formål at give et generelt billede af hvordan et simuleringsprogram vil være behjælpelig eller påvirke den transport og fragt samt om det har et incitament til at hindre eller støtte udviklingen af programmet.

\subsection{Taxiselskaber}\label{Taxiselskaber}

hold

\section{Privatkørsel}\label{Privatkoersel}

Dette afsnit har til formål at give et generelt billede af hvordan et simuleringsprogram vil være behjælpelig eller påvirke privatkørsel samt om det har et incitament til at hindre eller støtte udviklingen af programmet.

\subsection{Myldretidstrafik}\label{Myldretidstrafik}

Der er ingen tvivl om hvorvidt at myldretiden er det tidspunkt på dagen, hvor en simulering vil være til stop hjælp, og kunne afvikle et stort tidsforbrug. Myldretidstrafik propper er ofte forskyldt ved opbremsning og accelerations problemer. En undersøgelse fortaget i Japan viste  hvordan trafikpropper opstod på trods af at alle forsøgspersoner, var blev bedt om at køre med en konstant hastighed i en cirkel. Det viser altså hvordan, at trafikpropper vil opstå i tæt trafikkeret områder, selvom der ikke vil opstå nogle problemer eller forhindringer i trafikken. En simulering der kan være med til at distribuere trafikken, således at alle ikke skal igennem de store knudepunkter i trafikken, kan være med til at hindre store trafikpropper.

https://www.youtube.com/watch?v=Suugn-p5C1M

https://www.newscientist.com/article/dn13402-shockwave-traffic-jam-recreated-for-first-time/

http://iopscience.iop.org/article/10.1088/1367-

2630/10/3/033001/meta;jsessionid=438C580C1C2C3B3FB10D9B11878FE7D3.c3.iopscience.cld.iop.org

\subsection{Enkelte borger}\label{Enkelte-borger}

Den enkelte borger i privatbiler er dem som udgøre den største del af trafikken, specielt omkring de tidspunkter hvor folk skal på arbejde, generelt om morgenstunden, samt når de skal hjem fra arbejde, generelt imellem eftermidaggen og aften, hvor det ofte er de tidspunkter hvor der er størst chance for at mængden af trafikulykker, uheld og trafikpropper opstår (find en kilde). eventuelt giv en sammenligning hvordan trafiktrykket kan sammenlignes med andre ting?

\section{Katastrofekørsel}\label{Katastrofekoersel}

Dette afsnit har til formål at give et generelt billede af hvordan et simuleringsprogram vil være behjælpelig eller påvirke katastrofekørsel samt om det har et incitament til at hindre eller støtte udviklingen af programmet.

\subsection{Alarmberedskab}\label{Alarmberedskab}

hold

\subsection{Politi}\label{Politi}

hold

\subsection{Ambulanceudrykning}\label{Ambulanceudrykning}

hold

\subsection{Trafikuheld}\label{Trafikuheld}

Når der sker et trafikuheld, standser det hele trafikken indtil ambulanceredere har været og givet behandling til de tilskadekommende og der har været et oprydningshold. Sker der et uheld på en stærkt trafikkeret vej, som motorvejen eller en hovedvej, kan det skabe store køer alt efter uhelds størrelsesomfang, samt hvor hurtig svartiden er på beredskabsfolket. Dette koster mange tusinder mennesker timer, selvom der bliver sagt over trafiksstyrelsesradioen, hvilke strækninger man bør undgå. Derfor vil en simulering der kan hjælpe med at distribuere trafikken efter ulykken således at man undgår store trafikpropper.

\section{Kommunale og statslig instanser}\label{Kommunale-og-statslige-instanser}

Dette afsnit har til formål at give et generelt billede af hvordan et simuleringsprogram vil være behjælpelig eller påvirke de kommunale og statslige instanser samt om det har et incitament til at hindre eller støtte udviklingen af programmet.

\subsection{}



\subsection{Vejarbejde}\label{Vejarbejde}

Når der er vejarbejde optager de for det meste et helt vejspor eller mere, i værste tilfælde lukker de strækningen der arbejdes med. De veje som ikke er præget af meget trafik, vil ikke opleve den store påvirkning i trafikken, men større trafikkeret veje, broer, tunneller og motorvejsstrækninger kan skabe trafikpropper når mængden af billister der skal passere forbliver forholdsvist uændret, mens vejpladsen er blevet reduceret med et spor eller mindre (find kilder på hvor meget trafikken kan påvirkes pga. vejarbejde eventuelt kig på holland fra sommerferien). En simulering som kan vise hvordan \fi