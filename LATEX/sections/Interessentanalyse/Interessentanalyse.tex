%!TEX root = ..\..\Main.tex
\chapter{Interessantanalyse}\label{Interessentanalyse}

I følgende afsnit vil der redegøres for hvem vi mener har en interesse i, at der bliver udviklet en softwareløsning som kan simulere forskellige instancer af trafikhændelser og hvordan nogle kunne se en interesse i at sågar modarbejde sådan et produkt. Dette er essentielt til at kunne opstille krav for sådan en løsning da det vil have konsekvenser for udviklingen af løsningen. Interessenterne beskrevet i følgende afsnit vil altså blive afgrænset således at softwareløsningen er passet til denne bestemte målgruppe om man vil.

\section {Transport- og Bygningsministerists (TRM) - Vejdirektoratet}\label{TRM}

Transport- og Bygningsministeriet(TRM) er Danmarks øverste danske statslige myndighed på transportområdet og bygningsområdet. TRM’s hovedopgave er at sikre sig at de forskellige love bliver overholdt, ved opførelse af fx. en motorvej. Da TRM er en sammensætning af mange underdelinger har vi valgt at fokusere på en af deres styrelser nemlig Vejdirektoratet.

\vspace{5mm}

Vejdirektoratet står bag statsvejnettet som primært består af motorveje, hovedlandeveje og mange af landets broer. Alt i alt dækker disse forskellige veje 3.801 km vej \cite{Vejdirektoratet}. Dette udgør i alt 5\% af det offentlige vejnet men på trods af dette så er disse veje samtidig de veje hvor godt 50\% af alt danmarks trafik forgår på. Vi mener at Vejdirektoratet er en væsentlig interessant netop da de er ansvarlige for planlægning af vejnettet i Danmark. Da vi agter at skabe en løsning der har funktionaliteten til at planlægge disse veje og skabe forskellige trafik scenarier til at simulere potentielle alternativer til at opbygge sådan en vejnet.

\vspace{5mm}

Med en interessant som Vejdirektoratet skal kvaliteten af softwareløsningen møde en hvis standard da løsningen gerne skulle konkurrere med allerede eksisterende værktøjer der anvendes af Vejdirektoratet.

\section {Den Kommunale Sektor} \label{DenKommunaleSektor}
Kommunerne er ansvarlige for at vedligeholde og oprette veje i de dele af Danmark der nu er afsat til dem. Kommunerne skal godkende oprettelse af nye veje i deres områder hvilket vil sige at der oftest er andre organisationer indblandet så som det førnævnte TRM. Planerne for disse veje er altså nogle som skal pitches til kommunen således man kan præsentere ens case for at der netop er brug for oprettelse af en vej. Til netop dette kunne kommunen have en interesse i at sådan en case bliver opstillet i et simuleringsprogram som vores hvori det vil fremgå hvordan ændringer/oprettelse af en vej ville udspille sig i teori. Det bliver nævnt i rapporten fra Danmarks TransportForskning at nogle enkelte af kommunernes modeller er blevet opdateret og vedligeholdt, f.eks. Odense er blevet opdateret i 2004. \cite{dtfnotat}

\vspace{5mm}

Dette betyder at vi anser den kommunale sektor som en mulig interessant i den kontekst at informationen fra vores løsning ville kunne argumenterer for en case om at nye veje skal oprettes. Dette betyder dog at den udviklede løsning skal kunne opnå en kvalitet hvori det bliver en anerkendt standard for præsenterbare, faktuelle simuleringer.

\subsection{Visual Solutions}
Visual solution er et blandt andre firmaer der har lavet forskellige simulerings værktøjer. VisSim er f.eks. et anerkendt værktøj som er anvendt af over 100.000 forskellige forskere verden rundt\cite{VisualSolutions}.
Vores simulerings program kan, med mere tid og udvikling potentielt blive en konkurrent til disse programmer og derfor menes det at dette firma er en modvirkende Interessant der kunne have en interesse i at modarbejde ideen.

\vspace{5mm}

Visual Solutions værktøj, VisSim er derfor bl.a. Også blevet analyseret i denne rapports Teknologianalyse. Som interessant kunne Visual Solutions prøve at modvirke løsningen udarbejdet som en del af dette projekt, potentielt kunne denne løsning blive en mulig konkurrent til VisSim hvilket potentielt kunne lede til at firmaet, alt efter teknologien udarbejdet på længere sigt, til at søge om at tilegne softwareløsningen.

\section{Uddannelsessektoren}
En løsning som den vi agter at lave i dette projekt kan også være et godt værktøj til uddannelse af folk der vil arbejde inde for trafik sektoren. Dette kunne bl.a. Være DTU Transport som forsker inde for transportområdet. DTU har før i tiden foretaget undersøgelse i sammenhæng med optimering af trængsel i trafik, miljøproblematikken og trafiksikkerhed\cite{DtuForskning}. I dette tilfælde ville værktøjet pivot mod en uddannelses kontekst hvilket på samme tid også kunne være en potentielt ide til videreudvikling. DTU er som sagt også ansvarlig for mange undersøgelse med anledning i trafik og kunne potentielt bidrage til udviklingen af softwareløsningen eller fremtidige iterationer af den. Studerende der læser til vej- og trafik teknik, kunne have interesse i at få et simuleringsværktøj, som kræver mindre viden i at bruge.

\vspace{5mm}

Uddannelsessektoren tilbyder en interessant mulighed til at udarbejde projektet i en anden retning. 

\section{Specificering af målgruppe}

Ud fra interresantanalysen og teknologianalysen kan der delkonkluderes hvilken interresant der har størst interesse for vores projekt. Denne interresant er vurderet til at være kommunen baseret på at kommunerne står for størstedelene af vejnetværket i Danmark \cite{benjasfar}. Derudover er den kommunale sektor anset for at være den mest realistiske interessant for gruppen, eftersom det er et skala vi har mulighed for at arbejde på. Det er blevet bestemt at et simulerings værktøj med netop dette fokus, ville gavne mest af at simulere i en mesoskopiske kontekst. Dette er yderlige uddybet i krav og specifikationer afsnittet i denne rapport. En god grund til at vi har valgt kommunerne som målgruppe, er at mange af kommunernes modeller ikke er blevet opdateret \ref{DenKommunaleSektor}. 
