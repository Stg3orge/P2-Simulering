\section{Læsevejledning}\label{Laesevejledning}

\textbf{Kapitel 1 - 2.} I kapitlerne her vil rapportens indledning blive præsenteret, hvor vi kommer ind på tema vores rapport handler om. Derefter vil der være en kort redegørelse om hvad simulering er.

\vspace{5mm}

\textbf{Kapitel 3.} Her vil der blive beskrevet hvilke anvendte metoder der er blevet brugt til udarbejdelsen af denne rapport.

\vspace{5mm}

\textbf{Kapitel 4 - 8.} I kapitlerne her vil rapportens problemanalyse være, hvor vi mere uddybende forklarer vores undren om det valgte område. Til sidst vil der være en endelig problemformulering, som er med til at afgrænse problemfeltet.

\vspace{5mm}

\textbf{Kapitel 9 - 15.} Her vil problemløsningen være, hvor teorien bag vores løsning vil blive forklaret. Så vil man komme ind på vores design af programmet, hvilke krav skal vores program kunne. Detnæst kommer vores beskrivelse af vores program, hvor vi beskriver de vigtigste dele i programmet. Til sidst vil rapporten slutte med en diskussion og konklusion samt perspektivering.


\section{Kildekritik}
For at finde frem til relevant information, er der blevet taget udgangspunkt i kilder med ophav fra statslige instanser eller anerkendte virksomheder og organisationer, for at sikre informationernes gyldighed og troværdighed. Dette vil det også være muligt at kunne kontakte kilderne for uddybende spørgsmål, skulle der opstå tvivl om dele af informationen eller indsamlet data i det anvendte information. Ydermere er der blevet forsøgt at finde den nyeste tilgængelige information, for at sikre at den indsamlede data stadigvæk er brugbart og gyldigt. Fordi flere af de anvendte kilder kunne have en tendens, er den præsenteret information blevet kritisk overvejet efter hvorvidt det forholder sig objektivt eller om det er blevet fremstillet til at opnå noget for egen vinding eller et specielt mål.