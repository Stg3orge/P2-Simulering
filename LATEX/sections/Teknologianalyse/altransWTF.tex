\section{Altrans}
Alternativ Transportsystemer (Altrans) er en trafikmodel, hvis formål er at belyse hvordan en øget brug af den kollektive transport vil påvirke miljøet \cite[s. 14]{dmumodelanalyser}. Trafikmodellen er blevet udviklet af Danmarks Miljøundersøgelser (DMU) i 1994. Altrans består af 3 hovedmodeller: en geografisk model, en adfærds model og en emissions model \cite[s. 14]{dmuadfaerdsmodel}. Emissions modellen bliver ikke nævnt i rapporten, da den ikke er relevant for dette projekt.
\subsection{Geografisk model}
Den geografiske model bruges til at beregne rejsetider, ventetider og skiftetider. For at udregne disse benyttes følgende undermodeller:

\begin{itemize}
\item Model af kollektiv transportnet
\item Model af serviceniveau
\item Model for bilrejser
\item Model for attraktion til byfunktioner
\end{itemize}

Modellen for attraktion til byfunktioner består af information over antallet af beboere og arbejdspladser i forskellige områder, og benyttes mest i forbindelse med adfærds modellen. Derudover bruger den geografiske model et geografisk informations system (GIS) til opbevaring af data og udregning af rejsetiderne \cite[s. 18-19]{dmumodelanalyser}.

\vspace{5mm}

For at lave realistiske simuleringer, bruger modellen for det kollektive trafik præcise data for ankomst- og afgangstider. Denne data kommer fra 11 trafikselskaber der bruger køreplansystemet TR-System, DSB’s data kommer i et andet format der bliver brugt til DSB’s egen rejseplanlægger. For at beregne rejsertiderne af kørestrækningerne, samt ankomst og afgangstiderne sat op i et tredimensional koordinatsystem, hvor tiden bliver indsat som den tredje akse Z. Stationernes placering bliver indsat som X og Y koordinaterne, og man kan dermed finde ud af hvilke ruter der kan rejses med ved en given station \cite[s. 20-22]{dmumodelanalyser}.

\vspace{5mm}

Modellen af serviceniveau udregner serviceniveauet med variablerne tid, omkostninger og tilgængelighed. Modellen undlader at inkluderer variabler som komfort, da DMU har lavet antagelsen, at komforten ikke ændre sig kraftigt over tid. Dette kan gøre prognoser der ser på den fjerne fremtid upræcise. Modellen spænder over både taktisk (meso) og operationel (mikro). På den taktiske plan kigger Altrans på buskilometer, afgangs frekvenser og tilgængelighed. På den operationelle plan kigges der på tiden man bruger i køretøjet, hvor lang tid man skal vente ved skift samt ventetiden i alt, og prisen på rejsen \cite[s. 36-37]{dmuadfaerdsmodel}.

\vspace{5mm}

Formålet med modellen for bilrejser er at udregne tiden det tager at rejse fra by til by. Dette gøres ved brug af vejnettet i GIS, og hastigheden bilen kører kommer an på vejtypen. Ruten der bliver kørt starter og slutter fra centrumet af byerne der bliver rejst mellem. Modellen tager ikke højde for anden trafik på vejene, så tiderne der udregnes vil være præcise hvis der ikke er andre bliver på vejnettet \cite[s. 25]{dmumodelanalyser}. Hastighederne modellen bruger til de forskellige veje, kan ses på tabel \ref{altranshastigheder}.

\begin{table}[h!]
\centering
\caption{Hastigheder på forskellige vejtyper}
\label{altranshastigheder}
\begin{tabular}{|l|l|}
\hline
Motorveje             & 110 \\ \hline
Motortrafikveje       & 90  \\ \hline
Hovedveje             & 80  \\ \hline
Øvrige veje på landet & 70  \\ \hline
Veje i byer           & 40  \\ \hline
\end{tabular}
\end{table}

\subsection{Adfærds model}
Adfærds modellens formål er at give et estimat på fordelingen af transportmiddelvalg, populariteten af destinationer, kørekort fordeling, og bilejerskab. For at udregne estimaterne, benytter adfærds modellen sig af 3 undermodller \cite[s. 25-26]{dmumodelanalyser}:

\begin{itemize}
\item Model for valg af transportmiddel og destinationer
\item Cohortmodel og model for kørekorthold
\item Model for bilejerskab
\end{itemize}

Modellen for valg af transportmiddel og destinationer estimerer og simulerer antallet af kilometer der bliver rejst i de 4 transportmiddelkategorier: kollektiv trafik, bilfører, bilpassager og let trafik. Derudover estimeres de rejsenes destinationer, hvilket gør det muligt at finde ud af hvordan trafikken bliver fordelt på vejnettet, så trafikkens påvirkning på miljøet kan analyseres. Hovedformålet med at finde destinationerne er dog at modellere indvider i samfundet. Modellen vægter nytten ved rejserne, for eksempel kan en rejse til den nærmest købmand være mere nyttig end en der ligger længere væk. For at finde ud af hvilket transportmiddel et individ vælger, kigges der på prisen og tiden af rejsen, samt individets socioøkonomiske baggrund \cite[s. 26-27]{dmumodelanalyser}.

\vspace{5mm}

Modellen for kørekorthold er en prognosemodel. Sandsynligheden for at et individ har et kørekort, er udregnet ud fra kørekortfordelingen over alle individer og en logitmodel med variablerne køn, alder, indkomst, stilling og urbaniseringsgrad. Dette inddrages i cohortmodellen der simulerer om individet har et kørekort i det år der bliver beregnet på \cite[s. 30]{dmumodelanalyser}.

\vspace{5mm}

Modellen for bilejerskab estimerer hvor mange biler en husstand har. Modellen består af en logitmodel der bestemmer hvor mange biler husstanden har, denne logitmodel er indlejret i anden logitmodel, der bestemmer hvorvidt husstanden har biler eller ej. Til at bestemme om husstanden har bil, kigges der på husstandens socioøkonomiske forhold, om individerne i husstanden har kørekort, og hvor individerne rejser til. Outputtet af denne model bruges efterfølgende I modellen for valg af transportmiddel \cite[s. 29-30]{dmumodelanalyser}.

%\subsection{Emissions model}
%Man kan beregne emissioner af biltrafikken, dette gøres ved følgene: Der tages hensyn til og beregnes efter bilens tilstand, varm motor og koldstart. Det er en udregning der består ved at finde summen af en varstarts-emissionskoefficient gange trafikarbejdet og et koldstartstillæg for hver tur.

%\vspace{5mm}

%Disse faktorer er ikke selvstændige og bliver lavet per bilens årgang og dens størrelse/brændstoftype. Hastigheden af fartøjet determinere varmstarts-emissionskoefficienten. Der er sågar fortaget undersøgelser, såkaldte årskørsels undersøgelse af Vejdirektoratet hvorfra det er konkluderet at årskørsel forudsat er uafhængigt af bilens størrelse selvom at dette er set som urealistisk. Det estimeret trafikarbejde bliver udregnet ifølge en adfærdsmodel hvori man kigger på årskørsel pr bil i alle aldersgrupper. Der forøges eller reduceres med en faktor i selve fremskrivningsåret sådan at summen af antal biler i hver gruppe ganges deres gennemsnitlige årskørsel bliv lig det førnævnte trafikarbejde.

%\subsubsection{Varmstart}
%For at beregne varmstarts-emissionkoefficienten tager man udgangspunkt i COPERT II for den pågældene årgang. COPERT II er et windows program som gør det muligt at udregne emission fra vej trafik. Ydermere er programmet i stand til at lave lignende beregninger ud fra en forbrændingsmotor til et off-road fartøj. De beregnede emissioner inkludere alle de store forurenende stoffer så som (CO, NO\textsubscript{x}, VOC, PM) og mange flere. Programmet er ydermere i stand til at beregne brændstofforbrug \cite[s 4]{COPERT II}.  For nye biler i alle fremtidsår anvendes der de emissionskoefficienter der er blevet udarbejdet som EU-normer pr den årgang. Dette er ikke gældende for ældre biler. Her ændre emissionskoefficienterne sig med tiden og dette er specielt pågældende for katalysatorbiler. Emissionskoefficienten bliver korrigeret fra årgang til årgang alt efter motorslid, da informationer om emissionskoefficienten udvikling afhænger af fartøjets samlede kørsel. Når den samlede årskørsel bliver udregnet er det også muligt at beregne den gennemsnitlige bils samlede kørsel for en givet alder. Motorslid korrigerede emissionskoefficienten ved varmstart kan således udregnes for hver bilårgang for et beregnings år.

%\vspace{5mm}

%Til at beregne emissionen fra varmstarts er det forudsat at man har kendskab til fordelingen af trafik pr by-, land- og motorvejskørsel. ALTRANS har forudsat dette til at være som i basisåret.
%\subsubsection{Koldstartstillæg}
%I ALTRANS udregnes Koldstartstillægget ved brug af den tysk-schweizisk emissionsdatabase per bytrafiksituation. ALTRANS differentierer koldstartstillægget efter følgende:

%\begin{itemize}
%\item Turlængden. Disse kan være på 0-1 km, 1-2, 2.3, 3-4 og 4km.
%\item Pause siden sidste tur disse kan være 0-1 timer til over 8 timer.
%\item Udetemperaturen i temperaturspring på 5 grader som beregnes som månedsmiddelværdien.
%\item Bilens alder.
%\end{itemize}

%Derudover tager bilmodellen i ALTRANS også motorslid med i beregningerne for koldstartstillæg.

\section{Vurdering}
Når der i den Geografiske model bliver udregnet rejsetider for biler, bliver der ikke overvejet hvordan trafikken er på vejene. Rejsetiderne bliver udregnet ved at finde ruten gennem vejnettet og derefter gange delafstandene med hastighederne på figur \ref{altranshastigheder}. At udelade trafikdensiteten i udregningen kan dermed gøre bilrejser mere attraktive når et individ skal vælge transportmiddel, hvilket kan føre til et upræcist resultat. Ved simulering kan man selvfølgelig ikke lave en model der passer 100\% på virkeligheden, men i dette tilfælde kunne modellerne muligvis have taget et andet advanceret simulerings program som for eksempel VisSim i brug til at udregne realistiske rejsetider. Et andet problem når der skal udregnes bilrejser, er at afstanden bliver udregnet fra centrum til centrum af byer. Dette kan give et urealistisk billede hvis individet egentlig kun skal fra udkanten af en zone til udkanten af en sidelæggende zone, specielt hvis rejsen foregår i kun et centrum, da modellen da vil tage gennemsnittet for rejser i det centrum. Derudover gør det, at individet bare skal køre til et centrum, at modellen ikke overvejer hvor langt individet skal gå fra en parkeringsplads til destinationen. 

\vspace{5mm}

Adfærds modellen finder destinationen et individ rejser til, og hvilket transportmiddel der bliver valgt, men der bliver ikke overvejet om individet vil rejse eller ej. Det vil sige at alle individerne i simuleringen rejser på en beregnings tidspunktet. Dette kan gøre at både vejnettet og den kollektive trafik virker til at være mere belastet end de i virkeligeheden vil være. I forhold til Altrans formål, at finde miljøpåvirkningen i skift fra bilrejser til kollektiv transport, vil dette ikke gøre at resultatet bliver upræcist, hvis man kigger på dataene procentvis, men det vil være svært at bruge Altrans resultater i sammenspil med andre simuleringsmodeller, der overvejer hvor mange individer der rejser på en dag.

\vspace{5mm}

I forhold til vedligeholdelse, har Altrans følgende ulemper. Vejnettet og destinationerne bliver indlæst i et GIS system fra Transportvaneundersøgelsens data, hvilket betyder at det ikke er muligt selv at styre hvordan vejnettet ser ud. Dette kan blive problematisk da disse undersøgelser bliver foretaget med et 3 års interval, og man kan dermed risikere at arbejde med forældet data. Derudover er det er ikke muligt at specificere væksten af antallet af biler på vejnettet, da antallet afhænger af adfærdsmodellen. Det at man ikke kan definere denne vækst, kan være en af grundene til at modellen ikke længere bliver vedligeholdt, da den i fremtiden vil blive mere og mere upræcis. Generelt er Altrans meget fokuseret på hovedformålet, at finde ud af udviklingen i fordelingen af individer mellem den kollektive trafik og bilrejser. Hvis man skulle få brug for at vide hvordan denne fordeling påvirker vejnettet, kan man blive nødt til at bruge et andet simuleringsværktøj. Havde Altrans været mere fleksibel og spillet bedre sammen med andre simulerings modeller, vil der sandsynligvis være en større interesse i at vedligeholde den.