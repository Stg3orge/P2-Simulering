%!TEX root = ..\..\Main.tex

\chapter{Interessentanalyse}

Dette kapitel vil afdække nogle af de interessenter som ville have et incitament til at se et program, der kan simulering en optimering af trafikken udviklet, samt de interessenter som vil have et incitament til at hindre eller miskrediterer udviklingen af dette. Der vil blive udvalgt instanser til at repræsentere de relevante interessenter til at afspejle et generelt billede i samfundet.

\section{Kollektiv trafik}

Dette afsnit har til formål at give et generelt billede af hvordan et simuleringsprogram vil være behjælpelig eller påvirke den kollektive trafik samt om det har et incitament til at hindre eller støtte udviklingen af programmet.

\subsection{Nordjyllands Traffikselskab (NT)}



\section{Erhvervskørsel}

Dette afsnit har til formål at give et generelt billede af hvordan et simuleringsprogram vil være behjælpelig eller påvirke erhvervskørsel samt om det har et incitament til at hindre eller støtte udviklingen af programmet.

\subsection{Transport og fragt}

Dette afsnit har til formål at give et generelt billede af hvordan et simuleringsprogram vil være behjælpelig eller påvirke den transport og fragt samt om det har et incitament til at hindre eller støtte udviklingen af programmet.

\subsection{Taxiselskaber}

hold

\section{Privatkørsel}

Dette afsnit har til formål at give et generelt billede af hvordan et simuleringsprogram vil være behjælpelig eller påvirke privatkørsel samt om det har et incitament til at hindre eller støtte udviklingen af programmet.

\subsection{myldretidstrafik}

Der er ingen tvivl om hvorvidt at myldretiden er det tidspunkt på dagen, hvor en simulering vil være til stop hjælp, og kunne afvikle et stort tidsforbrug. Myldretidstrafik propper er ofte forskyldt ved opbremsning og accelerations problemer. En undersøgelse fortaget i Japan viste  hvordan trafikpropper opstod på trods af at alle forsøgspersoner, var blev bedt om at køre med en konstant hastighed i en cirkel. Det viser altså hvordan, at trafikpropper vil opstå i tæt trafikkeret områder, selvom der ikke vil opstå nogle problemer eller forhindringer i trafikken. En simulering der kan være med til at distribuere trafikken, således at alle ikke skal igennem de store knudepunkter i trafikken, kan være med til at hindre store trafikpropper.

https://www.youtube.com/watch?v=Suugn-p5C1M

https://www.newscientist.com/article/dn13402-shockwave-traffic-jam-recreated-for-first-time/

http://iopscience.iop.org/article/10.1088/1367-

2630/10/3/033001/meta;jsessionid=438C580C1C2C3B3FB10D9B11878FE7D3.c3.iopscience.cld.iop.org

\subsection{Enkelte borger}

Den enkelte borger i privatbiler er dem som udgøre den største del af trafikken, specielt omkring de tidspunkter hvor folk skal på arbejde, generelt om morgenstunden, samt når de skal hjem fra arbejde, generelt imellem eftermidaggen og aften, hvor det ofte er de tidspunkter hvor der er størst chance for at mængden af trafikulykker, uheld og trafikpropper opstår (find en kilde). eventuelt giv en sammenligning hvordan trafiktrykket kan sammenlignes med andre ting?

\section{Katastrofekørsel}

Dette afsnit har til formål at give et generelt billede af hvordan et simuleringsprogram vil være behjælpelig eller påvirke katastrofekørsel samt om det har et incitament til at hindre eller støtte udviklingen af programmet.

\subsection{Alarmberedskab}

hold

\subsection{Politi}

hold

\subsection{Ambulanceudrykning}

hold

\subsection{Trafikuheld}

Når der sker et trafikuheld, standser det hele trafikken indtil ambulanceredere har været og givet behandling til de tilskadekommende og der har været et oprydningshold. Sker der et uheld på en stærkt trafikkeret vej, som motorvejen eller en hovedvej, kan det skabe store køer alt efter uhelds størrelsesomfang, samt hvor hurtig svartiden er på beredskabsfolket. Dette koster mange tusinder mennesker timer, selvom der bliver sagt over trafiksstyrelsesradioen, hvilke strækninger man bør undgå. Derfor vil en simulering der kan hjælpe med at distribuere trafikken efter ulykken således at man undgår store trafikpropper.

\section{kommunale og statslig instanser}

Dette afsnit har til formål at give et generelt billede af hvordan et simuleringsprogram vil være behjælpelig eller påvirke de kommunale og statslige instanser samt om det har et incitament til at hindre eller støtte udviklingen af programmet.

\subsection{}



\subsection{Vejarbejde}

Når der er vejarbejde optager de for det meste et helt vejspor eller mere, i værste tilfælde lukker de strækningen der arbejdes med. De veje som ikke er præget af meget trafik, vil ikke opleve den store påvirkning i trafikken, men større trafikkeret veje, broer, tunneller og motorvejsstrækninger kan skabe trafikpropper når mængden af billister der skal passere forbliver forholdsvist uændret, mens vejpladsen er blevet reduceret med et spor eller mindre (find kilder på hvor meget trafikken kan påvirkes pga. vejarbejde eventuelt kig på holland fra sommerferien). En simulering som kan vise hvordan