\chapter{Black Box Test - Unit Test}\label{Test}
\textbf{Black Box Test} \newline
Black box testing er en software metode som man benytter til at teste sit program for interne fejl. Denne metode er den måde de fleste tester deres program, og bliver brugt i størstedelen af det praktiske liv. 

\vspace {5mm}

Denne metode er kaldt ”Black Box”, da man behandler den som en sort box, hvor man ikke kender det interne struktur af den software man vil teste. Når så man vil teste programmet, så gør man det fra kundernes synspunkt. Når så man tester programmet, så kender kunden ikke til hvordan programmet virker, så han sidder med noget input som han vil putte ind i denne ”Black Box”, hvor outputtet er noget man forventer der har været igennem en proces i programmet og udarbejdet til noget man kan bruge[1]. Det vigtigste formål med dette er at kontrollere om softwaret arbejder som der forventes af det, altså detopfylder kravene som der er blevet sat og det opfylder kundernes forventninger til det.
Der er forskellige måder at teste der bliver benyttet i industrien. Hver testing måde har deres egen fordele og ulemper. De forskellige måder er:
\begin{itemize}
\item Boundary Value Analysis (BVA)
\item Equivalence Class Partitioning
\item Decision Table based testing
\item Cause-Effect Graphing Technique
\item Error Guessing
\end{itemize}

\vspace {5mm}

\textbf{Unit Test} \newline
En unit er det minste testdel af et program, altså det kan være funktioner, classer, procedure eller interfaces. Unit testing er en metode man kan benytte når man vil teste hver inviduel del af programmet virker og om de er egnet til brug. Unit tests er skrevet og benyttet af programmører, for at være sikker på at ens kode opfylder de krav som er forventet af det. 

\vspace {5mm}

Formålet med denne test er at splitte programmet op i mindre dele, hvor derefter man tester en efter en at de forskellige dele fungere optimalt og som de skal. Det kan være en funktion, som man vil teste, hvor man har nogle input og derefter skulle funktionen have de rigtige output fra funktionen. På den måde kan man håndtere fejl når det forkerte input er givet[2].

\vspace {5mm}

Fordelene ved unit testing:
\begin{itemize}
\item Problemer/fejl findes tidligt, så det ikke påvirker senere kode.
\item Unit testing hjælper med at vedligeholde og nemt kan ændre koden.
\item Opdagelse af bugs tidligt, hvilket hjælper med at reducere omkostningerne når man skal fejlrette koden.
\item Unit test hjælper med at forenkle debugging processen, så hvis der sker en test fejl, så er det kun de seneste rettelser der skal rettes.
\end{itemize}




