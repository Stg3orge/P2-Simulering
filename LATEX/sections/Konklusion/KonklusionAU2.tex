\chapter{Konklusion}\label{Konklusion}
Det daglige tidsspild på 25 tusinde timer på de danske veje, som koster den danske befolkning mere end 2500 millioner danske kroner om året, er et problem hvor vi, på baggrund af problemananlysen, kan konkluderer at en mulig løsning ligger i optimeringen af det danske vejnet. En optimering af vejnettet vil kræve en udbyggelse ved alle de knudepunkter, hvor trafikken ofte bliver hæmmet eller går i stå, for at mindske ruternes belastning. Derfor er det vigtigt at den tiltænkte udbyggelse der skal laves for at aflaste ruterne, faktisk også vil afhjælpe problemet før der investeres i at udvide vejnettet. Til dette formål anvendes der simulering for at undersøge hvorvidt disse udbyggelser faktisk også vil gøre en forskel og hvorvidt det kan betale sig. Men ud fra vores analyse af eksisterende løsninger kan vi konkluderer at disse løsninger ikke er fleksible nok til at kunne tilpasse sig ændringer i konteksten eller er ikke særlig brugervenlige. Derfor er der blevet udviklet et simuleringsprogram, hvor det skulle have været muligt at afgøre, hvorvidt en ny rute eller udvidelse af vejnettet vil være en investering som kan betale sig at implementere. 

\vspace{5mm}
 Dette simuleringsværktøj gør det muligt for dem som skal træffe disse beslutninger, at teste deres løsninger, før de bliver implementeret i virkeligheden og på en billig og realitisk måde. Trods simuleringsværktøjet ikke er færdigudviklet, har det nogle fordele som de nuværende og tilgængelige  simuleringsværktøjer ikke har, såsom at det netop kan tilpasse sig ændringer i konteksten i form af justering i vejnet, mængden af biler og mere, samt at det har den brugervenlighed der gør det nemmere at bruge.

\vspace{5mm} 
På grund af vores fokus på at kunne præsenterer en realistisk simulering af trafik, har vi dog ikke haft success med at udvikle de funktionaliteter som skulle anvendes til at sammenligne data fra to forskellige simuleringer og i det hele taget kunne retunerer data fra simulering som output. Der kan derfor konkluderes at løsningforslaget ikke lever fuldt op til problemformuleringen da det ikke løser hele problemet på grund af manglen på resultater, manglen på evnen til at justerer bilernes adfærd og på samme tid lever det ikke op til alle succeskriterier. 