\chapter{Konklusion} %Lavet af Benjamin
Det kan konkluderes at mange trafik moddeller ikke bliver vedligeholdt i dag, derfor lavede vi et simulations program på mesoskopisk niveau som kan vedligeholdes, da brugeren selv kan opstille forskellige scenarier. Vi valgte at lave programmet på et mesoskopisk niveau, da dette kan tilpasses kommunerne, samtidig, var der flere kommunemodeller, som ikke bliver vedligeholdte i dag.

\vspace{5mm}

Vi undersøgte nuværende simulations programmer se afsnit \ref{Teknologianalyse}, her fandt vi ud af at Altrans har svært ved at blive vedligeholdt da Altrans ikke kan tilpasse sig ændringer i konteksten. Derudover var det svært for nye brugere af VisSim, at kende til programmet, da det kan så meget. Samtidig var der nogle fordele og ulemper ved begge værktøjer, vi har derfor imøde kommet disse, og programmeret et værktøj selv. Her kom vi frem til at det ikke er nemt at lave et program som både er brugervenligt, men samtidig også fleksibelt. Det konkluderes også at vores program ikke har en konkret formål, men derimod kan det vise hvordan vejnetværket bliver påvirket af trafik, og brugeren selv kan opstille to simuleringer og kombinere dette. På samme tid fandt vi ud af, at vores program nærmere bruges til trafik afvikling.

Derudover undersøgte vi hvilke interessenter, som vil blive påvirket af dette projekt se afsnit \ref{Interessentanalyse}. Her kom vi frem til at kommunerne var de centrale, da de ejede største delen af vejnetværket og kommunemodellerne ikke er vedligeholdte. Det var netop også derfor vi valgte at lave et program på det mesoskopiske niveau, da dette niveau vil tilpasse sig mest til kommunerne. 