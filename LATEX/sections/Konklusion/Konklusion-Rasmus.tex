\chapter{Konklusion}
\externaldocument{Teknologianalyse}
I dette projekt er der blevet undersøgt om nuværende simuleringsværktøjer bliver vedligeholdt, så de kan tilpasse sig til de ændringer som brugeren sætter til vejnettet. Herudover hvordan man kan ved hjælp af værktøj, set fra mesoperspektiv, opstille et vejnet hvor brugeren kan ændre på hvordan vejnettet skal sættes op. Vejnettet skal også kunne være i stand til at blive ændret i, altså efter ændringer i konteksten.  

\vspace{5mm}

Ud fra problemanalyse delen, kan vi konkludere at virksomhederne som står bag nogle af de nuværende simuleringsværktøjer, har det svært ved enten at vedligeholde deres værktøjere, eller gøre det brugervenligt. Herudover blev der også undersøgt hvordan vi bliver påvirket af vores interessenter, og hvordan vi påvirker dem med vores produkt, se \ref{Teknologianalyse}. Hvor vi fandt frem til at kommunerne var dem der ikke vedligeholdte deres trafikmodeller, hvilket gjorde dem til nogle af de vigtigste interessenter. 

\vspace{5mm}

Gennem projektet og udviklingen af vores produkt, fandt vi ud af at det ikke var så let at lave et værktøj, som både var brugervenligt og var i stand til at vedligeholde vejnettet, alt efter hvilke ændringer der er i konteksten. Det konkluderes også at vores program ikke har en konkret formål, men derimod kan det vise hvordan vejnetværket bliver påvirket af trafik, og brugeren selv kan opstille to simuleringer og kombinere dette.
