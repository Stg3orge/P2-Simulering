\chapter{Konklusion}\label{Konklusion}
I starten af dette projekt satte gruppen sig for at identificerer og løse et problem inde for emnet "Simulering". Det blev obsereret at i nogle kontekster kan der fortage sig ændringer, både store og små, som kan have en effekt på den data man søger fra en simulering i den givne kontekst. Efter at have foretaget en analyse af problemet, dets omfang og de eksisterende løsninger inde for simulerings området, blev der opstillet en problemformulering \ref{problemformulering}. På baggrund af denne problemformulering samt de krav og successkritirer der blevet sat så kan vi konkluderer at den udviklede løsning ikke tilfredstiller problemformuleringen og heller ikke en del af kravene og succeskriterier.

Programmet er tilnærmelsesvist i stand til at simulerer et vejnet med biler der kører på dette vejnet, hvor bilrnes ruter bliver udregnet med den kortestvej algoritme som gruppen har valgt at implementerer i programmet. Med det sagt, så er programmet ikke istand til at sammenligne en simulering sammen med en anden, da programmet ikke har noget output i form af data der kan anvendes til netop sammenligning. Dette vil sige at udover ren observation er der ikke noget der fortæller brugeren om en ny, sekundær rute som de har anlagt i vejnettet, er mere eller mindre effektiv. Derudover har vi sat meget ambitøse krav til programmet hvilket, givet gruppens resourcer, ikke har været mulige at implementerer på nuværende tidspunkt \ref{fejlogmangler}. 