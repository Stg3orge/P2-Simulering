\chapter{Problemformulering}\label{Problemformulering}

I problemanalysen er der blevet foretaget teknologianalyser med fokus på vedligeholdelse af de undersøgte værktøjer, yderemere en analyse af forskellige interressanter har givet henblik på omfanget af løsningsmodellen. I teknologianalysen ses det at VisSim er ikke specifikt til trafiksimulering og er vanskeligt at opsætte for nyere brugere, derimod har Altrans ikke nok forskellige input til at kunne anvendes i flere kontekster. Samtidig er det blevet observeret, at VisSim er fleksibelt og tillader brugerne at specificere deres simulationer med deres egne parametre.

\vspace{5mm}

Altrans er modsat VisSim, opbygget til at finde fordelling af individer mellem den kollektive trafik og bilrejser, et specifikt formål. Der ses en mulighed for at lave en softwareløsning der har fleksibiliteten i stil med VisSim, men kun inde for et bestemt område, meso-trafik-simulering. Vores specificerede interessant, den kommunale sektor, lægger meget op til at opbygge netop et program med et meso omfang da der kan opsættes simulering af forskellige scenarier i netop dette mesosimulering. Vi formulere derfor følgende problemformulering:

\vspace{5mm}

\begin{center}
\textit{Nuværende simuleringsværktøjer til simulering af trafik er enten sværere for nye brugere at anvende eller mangler fleksibiliteten til at kunne tilpasse sig den kontekst brugeren ønsker at arbejde i. En ideal løsning er at udarbejde et mesosimuleringsværktøj som kan tilpasse sig efter brugerens behov i den givne kontekst.}
\end{center}

\section{Ny}
Nuværende simuleringsværktøjer til simulering af trafik er enten sværere for nye brugere at anvende eller mangler fleksibiliteten til at kunne tilpasse sig den kontekst brugeren ønsker at arbejde i. Hvordan kan et mesosimuleringsværktøj, hvori brugeren gennem en brugerflade kan opstille et vejnet, samt indstille variabler som eksempelvis antallet af biler, hastigheder og adfærd, optimeres i forhold til vedligeholdese, trods ændringerne i konteksten?