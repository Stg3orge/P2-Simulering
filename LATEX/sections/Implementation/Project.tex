\section{Project}

Project klassen har til formål at sætte de forskellige klasser op i lister, som skal bruges til , mens Project (string name) samtidig sætter default values når man laver en ny RoadType, DestinationType eller VehicleType i programmet. Project Clones gør det muligt at gemme det man har lavet i en ny fil, som så kan bruges senere såfremt det skal sættes ind andre steder i programmet ved eksempelvis at copy-paste det. 

\begin{figure}[H]
\begin{lstlisting} namespace A319TS
{ 
    [Serializable]
    public class Project : ICloneable
    {
        public string Name;
        public List<Node> Nodes = new List<Node>();
        public List<Destination> Destinations = new List<Destination>();
        public List<LightController> LightControllers = new List<LightController>();
        public List<RoadType> RoadTypes = new List<RoadType>();
        public List<DestinationType> DestinationTypes = new List<DestinationType>();
        public List<VehicleType> VehicleTypes = new List<VehicleType>();
        public List<SimulationData> Simulations = new List<SimulationData>();
        public SimulationSettings Settings = new SimulationSettings();
        
        public Project(string name)
        {
            Name = name;
            RoadTypes.Add(new RoadType("Default", 50));
            DestinationTypes.Add(new DestinationType("Default", Color.LightSlateGray) { Distribution = 100 });
            VehicleTypes.Add(new VehicleType("Default", 130, 5, 5, Color.LightSlateGray) { Distribution = 100 });
        }

        public object Clone()
        {
            MemoryStream memory = new MemoryStream();
            BinaryFormatter formatter = new BinaryFormatter();
            formatter.Serialize(memory, this);
            memory.Position = 0;
            return formatter.Deserialize(memory);
        }
    }
}
\end{lstlisting}
\caption{Project.cs}\label{project.cs}
\end{figure}

