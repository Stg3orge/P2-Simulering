\section{Fejl og mangler i programmet}\label{FejlOgMangler}
Ikke alle programmets kravspecifikationer er blevet implementeret med ønsket effekt. I et af kriterierne ønskes det, at simuleringen også kan foregå med cyklister og fodgængere. Dette er dog ikke tilfældet, programmet simulere kun med køretøjer, så derfor har vi valgt at tage dette med i videreudvikling. Programmet opfylder dog heller ikke ønsket om acceleration og deceleration dvs. programmet bliver mindre realistisk, da dette var en central del af kriterierne. Samtidig var dette også vigtigt fordi VisSims acceleration og deceleration udspillede sig lineær, meningen var at lave en acceleration og deceleration som udspillede sig eksponentielt, da dette var mere realistisk. I tabel \ref{Fejl og mangler} er de følgende fejl og mangler.

%\centering
%\setlength\LTleft{\fill}
%\setlength\LTright{\fill}
\setlength{\LTleft}{-20cm plus -1fill}
\setlength{\LTright}{\LTleft}
\begin{longtable}{| p{0.50\textwidth} | p{0.50\textwidth} |}
	\hline
\textbf{Funktioner} & \textbf{Løsningsforslag} \\
	\hline
Programmet implementere ikke Fodgængere & Skal implementeres på samme måde som køretøjerne, dog skal der lave en knude, der hedder fodgængerfelt. \\
	\hline
Programmet implementerer ikke cyklister & Skal implementeres således der laves en separat cykelsti, som cyklerne kan cykle på. derudover skal vejene ændres, således cyklister og bliver kan være på en vej. \\ 
	\hline
Programmet implementerer ikke acceleration & Step skal ændre sig på en bestemt måde, således at step forøges gradvist ud fra køretøjets acceleration. Derudover holde øje med om der findes et objekt foran den.\\ 
	\hline
Programmet implementerer ikke deceleration & Køretøjerne skal kunne se yield fra en bestemt distance, så de kan nå at decellerere på samme måde som acceleration, bortset fra at de skal kunne bremse i stedet. \\ 
	\hline
Programmets åbne og gemme funktionalitet bruger for meget tid &  Programmets åbne og gemme funktionalitet skal forbedres ved brug af \texttt{Protobuf}. \\ 
	\hline
TimeSpread, ToHomeTime, SetTime og ToDestinationtime skal ændres således brugeren ikke indtaster millisekunder & Det der står i tekstboksen skal ændres så der står klokkeslæt eller timer, derefter skal teksten konverteres til en integer og ganges op. \\ 
	\hline
Programmets skal implementere menneskelig adfærd så bilerne ikke altid kører deres maksimale hastighed på vejene. & Lave en ny klasse \texttt{Driver}, hvor man kan indstille menneskelig adfærd. \\ 
	\hline
Der skal implementeres UnitTest på alle klasser i kildekoden. & Implementer UnitTests. \\ 
	\hline
Viewport går ud over kanterne i højre side og bunden. & Når programmet genererer gitteret, skal den tjekke programvinduets bredde og højde. Samt sørge for at Viewport positionen ikke kommer til at være større end hele størrelsen minus bredden og højden. \\ 
	\hline
Samling og analyse af simuleringens data. & \texttt{SimulationData} skal gemme alt data som simuleringen har, ved at programmet skal kigge på de forskellige positioner. Hvis positionen er lig hinanden så står den stille, ellers skal den måle hastigheden.\\ 
	\hline
Implementere en hjælpeside til information omkring programmets dele. & Lave en hjælpe knappe, som forklarer med tekst, hvordan programmet fungere. \\ 
	\hline
I et lyskryds kan køretøjerne lave u-vendinger. & Der skal laves en ny vej, som man kun må køre ind på fra bestemte knude. \\ 
	\hline 
\caption{Fejl og mangler}\label{Fejl og mangler}
\end{longtable}



\subsection*{Optimering af performance}\label{OptimeringAfPerformance}

\textbf{Hastighed} \\
Som sagt har programmet en svaghed når det kommer til performance. For altefter tickraten på simulationen, samt antal af biler osv vil brug af CPU og hukommelse øge markant. Det betyder at hastigheden på simulationen, og åbning af simulationensfilen kan tage fra få sekunder til flere timer alt efter de valgte simulationindstillinger. \\
Der blev lavet en performance profile som vidste at serialize og deserialize af vores simulations fil som tog det meste af simulation/view tiden. Desuden kan programmet ikke åbne filer, hvis de er over 1.2 GB.(\textbf{fordi VS ikke kan håndtere filer over 1.2GB})

\vspace{5mm}

Der blev derfor også kigget på andre metoder at serialize og deserialize simulation filerne istedet for at bruge XML. Der blev fundet frem til Protocol Buffer, og BinaryFormatter som iføgle flere tests siges at være hurtigere end XML \cite{SerializationPerformance}\cite{ComparingThePerformance}\cite{PerformanceTest}. Der blev derfor implementeret BinartFormatter, da det som XML er en del af NET Framework hvilket gjorde det en del nemmere at implementere, samt det understøttede klasser som Point og Color, hvliket Protocol Buffer ikke gjode som startard(Brug surrogate klasser til at konvetere typer som Point, Color til at være kompitabel til Protocol Buffer). Men selvom at BinaryFormatter er hurtigere end XML bliver det stadig overgået af Protocol Buffer som i nogen tests er 12x så hurtig iforhold til BinaryFormatter \cite{PerformanceTest}\cite{Protobuf}, og siden programmet stadig er langtsomt nå det kommer til serialize og deserialize kunne Protocol Buffer midske simulation/view tiden, og afhjælpe med programmet ikke terminere fordi filen enten er for stor, eller tager forlang tid at åbne.

\vspace{5mm}

\textbf{Hukomelses Brug} \\
Dog vil dette ikke afhjælpe på hukommelses forbruget, for at løse dette kunne vi optimere brugen af datatyper f.eks. bruge floats isetdet for doubles, hvliket vil halvere brugen af hukomelse hvergang vi beytter en double. Vi bruger f.eks double ved vores grid hvor vi bruger points med doubles værdier. \\

En anden måde at løse hukomelses problemet på, er ved brug af Memory-Mapped Files klassen, da den vil kunne splite filen op i en bestemt mægnde bytes  både ved gemning/indlæsning. Herved vil programmet ikke gemme/indlæse hele simulations filen på hukomelsen på engang, men kun den mægnden bytes som Memory-Mapped streamen er sat til. Herved ungåes det at programmet alt efter størelsen på simulationen har et hukommelses forbrug på måske over 4000MB, hvliket i sidste ende vil terminere enten programmet eller computeren.

\vspace{5mm}

\textbf{CPU forbrug} \\
For at optimere CPU forbruget, vil en mulig løsnig være at få GPU'en til at tegne grafikken i programmet, istedet for CPU'en. Dette kan gøres ved brug af Windows Presentation Foundation (WPF). Fordelen ved dette er at den mægnde CPU som bliver sparet,  vil kunne bruges på simulationen istedet, samt er GPU'en hurtigere til at tegne grafik end CPU'en. Derudover vil implemitering af flere "threads" kunne udnytte flere kerner i computeren, og herved øge simulationshastigheden, da vores program kun indeholder 2. Hvliket vil sige at f.eks. på en core i7 maskine med 8 kerner vil kun 25\% kraft blive brugt.






