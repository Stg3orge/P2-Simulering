\section{Pathfinder}\label{Pathfinder}
For at forklare implementationen af A*, tages der udgangspunkt i tre metoder som befinder sig i \texttt{Pathfinder} klassen, altså \texttt{FindPath}, \texttt{EstimateNeighbors} og \texttt{TracePath}. Grunden til at vi udvælger disse metoder er fordi de er vigtige i forhold til hvordan programmet skal finde frem til den hurtigste vej i programmet. Derudover vil der også kigges på en metode som ligger i \texttt{Vertex} klassen, der udregner kosten og estimeringen af den resterne kost.

\begin{figure}[H]
\begin{lstlisting}
public static List<Road> FindPath(Node start, Node end)
{
  if (Vertices == null || start == null || end == null)
    throw new ArgumentNullException();

  InitLists();
  SetStartEnd(start, end);
  Start.Cost = 0;
  Open.Add(Start);

  Vertex current;
  while (Open.Count > 0)
  {
    current = Open.Min();
    if (current == End)
    {
      return TracePath();
    } 
    else
    {
      MoveToClosed(current);
      EstimateNeighbors(current);
    }
  }
  throw new Exception("No path found");
}
\end{lstlisting}
\caption{FindPath metoden}\label{FindPathCode}
\end{figure}

Den første metode er \texttt{FindPath}, der ses på figur \ref{FindPathCode}, hvor der startes med at overskrive listerne \texttt{Closed} og \texttt{Open} med tomme lister gennem metoden \texttt{InitLists}. \texttt{SetStartEnd} metoden som køres derefter, finder de \texttt{Vertex} der svarer til start og end noderne der bliver taget ind som parameter. I while løkken kigger den på \texttt{Open} listen og tjekker om der er nogle \texttt{Vertex} der ikke er evalueret endnu, hvis ikke kan ruten ikke findes og der kastes en exception. Inde i while loopet sættes det nuværende \texttt{Vertex} til at være den mindste på \texttt{Open} listen. Inde i \texttt{Vertex} klassen, er interfacet \texttt{IComparable} implementeret, således at \texttt{Min()} metoden returnerer den \texttt{Vertex} med den mindste \texttt{Estimate}. Hvis vi er ved enden returneres \texttt{Path} gennem metoden \texttt{TracePath}, ellers bliver vi ved med at vurdere naboerne til den nuværende \texttt{Vertex}, og sætter den nuværende over på \texttt{Closed} listen, så den ikke bliver vurderet igen senere.

\begin{figure}[H]
\begin{lstlisting}
private static void EstimateNeighbors(Vertex current)
{
  foreach (Edge edge in current.Edges)
  {
    Vertex neighbor = edge.VertexTo;
    if (!Open.Contains(neighbor) && !Closed.Contains(neighbor)) 
    // Skip evaluated
    {
      Open.Add(neighbor);
      double PossibleCost = current.Cost + edge.Cost;
      if (neighbor.Cost > PossibleCost)
      {
        neighbor.Cost = PossibleCost;
        neighbor.Previous = current;
        double heuristic = 
          MathExtension.Distance(neighbor.Position, End.Position) 
          / MaxSpeed;
        neighbor.Estimate = neighbor.Cost + heuristic;
      }
    }
  }
}
\end{lstlisting}
\caption{EstimateNeighbors metoden}\label{EstimateNeighborsCode}
\end{figure}

I \texttt{EstimateNeighbors} metoden, der kan ses på figur \ref{EstimateNeighborsCode}, kigger vi på naboerne, til den \texttt{Vertex} der bliver givet gennem parameteret. For hver nabo tjekkes der om naboen allerede er evalueret, hvis den er det bliver den sprunget over, ellers bliver den evalueret. Derefter tjekkes der om naboens mulige \texttt{Cost} fra den nuværende node, er bedre end den \texttt{Cost} der allerede er fundet. Hvis \texttt{PossibleCost} er bedre, bliver nabo knudens \texttt{Cost} og \texttt{Estimate} overskrevet, og \texttt{Previous} sættes til at være den nuværende \texttt{Node}, så ruten senere kan blive fundet igen.

\begin{figure}[H]
\begin{lstlisting}
private static List<Road> TracePath()
{
  List<Road> roads = new List<Road>();
  Vertex current = End;
  while (current.Previous != null)              
  { 
    roads.Add(current.Previous.Edges.Find(edge => 
              edge.VertexTo == current).Source);
    current = current.Previous;
  }
  roads.Reverse();
  return roads;
}
\end{lstlisting}
\caption{TracePath metoden}\label{TracePathCode}
\end{figure}

Sidst har vi på figur \ref{TracePathCode} metoden \texttt{TracePath}, der finder vejen tilbage, når algoritmen støder på slut punktet. Dette gøres ved at kigge på \texttt{Previous} referencen for den nuværende \texttt{Vertex}, og tilføje den vej der ligger mellem dem til en liste, ind til at \texttt{Previous} er lig \texttt{null}, hvilket vil sige at den er nået tilbage til start. Før ruten returneres bliver der kørt \texttt{Reverse()} på listen, så den står i den rigtige rækkefølge.