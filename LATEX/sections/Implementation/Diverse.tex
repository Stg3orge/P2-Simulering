\section{Diverse}\label{Diverse}

\section{Project}

Project klassen har til formål at sætte de forskellige types klasser op i lister, mens Project (string name) samtidig sætter default values når man laver en ny RoadType, DestinationType eller VehicleType i programmet, hver gang der bliver "addet" noget nyt til de forskellige lister under Types. Project Clones gør det muligt at gemme det man har lavet i en ny fil, som så kan bruges senere såfremt det skal sættes ind andre steder i programmet ved eksempelvis at copy-paste det.

\begin{figure}[H]
\begin{lstlisting} 

{ 
    [Serializable]
    public class Project : ICloneable
    {
        public string Name;
        public List<Node> Nodes = new List<Node>();
        public List<Destination> Destinations = new List<Destination>();
        public List<LightController> LightControllers = new List<LightController>();
        public List<RoadType> RoadTypes = new List<RoadType>();
        public List<DestinationType> DestinationTypes = new List<DestinationType>();
        public List<VehicleType> VehicleTypes = new List<VehicleType>();
        public List<SimulationData> Simulations = new List<SimulationData>();
        public SimulationSettings Settings = new SimulationSettings();
        
        public Project(string name)
        {
            Name = name;
            RoadTypes.Add(new RoadType("Default", 50));
            DestinationTypes.Add(new DestinationType("Default", Color.LightSlateGray) { Distribution = 100 });
            VehicleTypes.Add(new VehicleType("Default", 130, 5, 5, Color.LightSlateGray) { Distribution = 100 });
        }

        public object Clone()
        {
            MemoryStream memory = new MemoryStream();
            BinaryFormatter formatter = new BinaryFormatter();
            formatter.Serialize(memory, this);
            memory.Position = 0;
            return formatter.Deserialize(memory);
        }
    }
}
\end{lstlisting}
\caption{Project.cs}\label{Project}
\end{figure}


\section{SimulationSettings}

SimulationSettings sætter defaultværdierne til de forskellige variabler, og giver brugeren muligheden for at ændre dem og gemme dem med nye værdier i stedet for de predefineret values, skulle man fortryde de selvdefineret values der er blevet sat, kan man altid vælge SetDefault values, ændre ens settings tilbage til de originale default values.

\begin{figure}[H]
\begin{lstlisting} 

{
    [Serializable]
    public class SimulationSettings
    {
        // Shared
        public int StepSize { get; set; }
        public int VehicleSpace { get; set; }
        public int IncommingRange { get; set; }

        // Primary
        public int PrimaryCarCount { get; set; }
        public int PrimaryInbound { get; set; }
        public int PrimaryOutbound { get; set; }
        public int PrimaryToDestTime { get; set; }
        public int PrimaryToHomeTime { get; set; }
        public int PrimaryTimeSpread { get; set; }

        // Secondary
        public int SecondaryCarCount { get; set; }
        public int SecondaryInbound { get; set; }
        public int SecondaryOutbound { get; set; }
        public int SecondaryToDestTime { get; set; }
        public int SecondaryToHomeTime { get; set; }
        public int SecondaryTimeSpread { get; set; }

        // Defaults
        public SimulationSettings()
        {
            StepSize = 100;
            VehicleSpace = 2;
            IncommingRange = 10;
            PrimaryCarCount = 1000;
            PrimaryInbound = 100;
            PrimaryOutbound = 100;
            PrimaryToDestTime = 28800000; // 08:00
            PrimaryToHomeTime = 57600000; // 16:00
            PrimaryTimeSpread = 14400000; // 4h
            SecondaryCarCount = 1000;
            SecondaryInbound = 100;
            SecondaryOutbound = 100;
            SecondaryToDestTime = 28800000; // 08:00
            SecondaryToHomeTime = 57600000; // 16:00
            SecondaryTimeSpread = 14400000; // 4h
        }
    }
}
\end{lstlisting}
\caption{SimulationSettings}\label{SimulationSettings}
\end{figure}

\subsection{Data}
\subsection{MathExtension}
\subsection{Vector2D}