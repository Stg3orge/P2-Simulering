\section{Program Oversigt}
Denne sektion omhandler strukturen af programmet med udgangspunkt fra 'GUIMain', programmets start og hoved vindue. 

\begin{figure}[!h]
    \centering
    \includegraphics[width=\textwidth,height=\textheight,keepaspectratio]{Pictures/Implementation/program}
    \caption{Programmets hoved vindue 'GUIMain'}
    \label{a319program}
\end{figure}

På figur \ref{a319program} ses GUIMain. GUIMain består af en menulinje, værktøjslinje og en viewport. Vinduet fungerer udelukkende ved brug af events. Ved tryk på et menupunkt, vil der udløses en event, der åbner det tilsvarende vindue. Menupunkterne er delt op i 4 forskellige kategorier. File indeholder menupunkter der håndtrer fil åbning og gemning. Settings har indstillinger til det nuværende projekt, fordeling af destinationer og transportmiddelvalg, og sidst indstillinger for hvordan simuleringen skal udføres. Types har menupunkter til opsætning af forskellige destination, vej og køretøj typer. Sidst kan man igennem 'Simulation' køre og vise simuleringer.

\vspace{5mm}

Værktøjslinjen har en række knapper der kan tjekkes. Når en knap bliver trykket på, bliver der sendt en event der kalder metoden 'ToggleTool' på ToolControlleren. ToggleTool sørger for at der kun er en aktiv knap ad gangen. Derudover er der to lister på værktøjslinjen, i listen til venstre kan brugeren vælge destinations typen der bliver brugt, og i listen til højre kan brugeren vælge hvilken vejtype der skal bruges.

\vspace{5mm}

Viewporten er gitteret hvor der er muligt at opstille et vejnet. Viewporten abonnerer på 'Move' og 'Click' begivenhederne. Hver gang brugeren flytter musen, vil viewporten finde ud af hvor musen er, og tegne en cirkel, så brugeren kan være sikker på, hvilken gitter position der vil blive tilføjet til på forhånd. Ved tilfældet at brugeren trykket på viewporten, tjekker ToolControlleren efter hvilket værktøj der er aktivt, og kører metoden der er forbundet til værktøjet.

\vspace{5mm}

\section{GUI}
Der er valgt at opstille en GUI til brugeren igennem windowsforms aplication.
Dette gøres for at brugeren har et bedre overblik over programmet.
GUI er opdelt i flere forskellige klasser, hver klasse håndtere enkle områder af programmet.
Der er valgt at opstille det på denne måde, således at vi som programmøre får et overblik over kildekoden.
I hver klasse er der programmeret diverse værktøjer såsom knapper, tekstbokse osv.
Herinde er der valgt at opstille størrelser og positioner på de forskellige værktøjer.
Vi har valgt selv at programmere værktøjerne i stedet for at få dem auto generet af Visual Studio,
fordi det optimere koden, og skaber et overblik, således vi selv har styr på hvor værktøjerne ligger i kildekoden.

\begin {lstlisting}
\end {lstlisting}

GUIMain

GUIMain er hvor alle underklasserne bliver samlet.

Andre klasser