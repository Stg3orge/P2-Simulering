\section{Kerne Funktionalitet}\label{Kernefunktionalitet}
\externaldocument{sections/Implementation/Vehicle}

For at håndtere alle dele af programmet blev der implementeret en et styringscenter, som heri findes som klassen \texttt{ToolController}. Formålet med denne fremgangsmåde skulle give bedre overblik over programmets mange metoder og give mulighed for let at håndtere fejl og ændringer i funktionaliteten. Klasserne i programmet er til dette formål håndteret objektorienteret og generelt nok til at fremgangsmåden kan fungere i praksis.

%//////////////////////////////// ToolController ////////////////////////////////%
\subsection*{ToolController}
\texttt{ToolController} klassen har til formål at forbinde de forskellige værktøjer så når brugeren f.eks. trykker på et af værktøjerne vil \texttt{ToolControlleren} kalde de tilsvarende metoder til værktøjet. Samt at der kun at være valgt et værktøj af gangen. \texttt{ToolController} er altså klassen som står for funktionerne som f.eks. \texttt{AddNode()}, \texttt{AddRoad()}, \texttt{AddLightController()} osv.

\begin{figure}[H]
\begin{lstlisting}
public ToolController(ToolStripItemCollection collection, 
                      Viewport viewport, Project project)
{
   Tools = collection;
   Viewport = viewport;
   Viewport.Input.MouseClick += ViewportClick;
   Project = project;
}
\end{lstlisting}
\caption{ToolController metoden}\label{ToolControllerCode}
\end{figure}

På figur \ref{ToolControllerCode} ses constructoren til \texttt{ToolController}, der bliver kaldt via GUIMain som sender alle værktøjerne, nuværende viewport samt projekt. Herved har \texttt{ToolController} alle elementerne til f.eks at tilføje en \texttt{Node}.

\begin{figure}[H]
\begin{lstlisting}
private void ViewportClick(object sender, MouseEventArgs args)
{
  if (ActiveTool != null && args.Button == MouseButtons.Left)
  {
    switch (ActiveTool.Name)
    {
      ...
      case "ToolAddNode": Add(typeof(Node)); break;
      case "ToolLinkLight": LinkLight(); break;
      case "ToolAddDestination": Add(typeof(Destination)); break;
      case "ToolAddRoad": AddRoad(Partitions.Shared); break;
      case "ToolPrimaryRoad": AddRoad(Partitions.Primary); break;
      case "ToolEdit": Edit(); break;
      ...
    }
  }
}
\end{lstlisting}
\caption{ViewportClick() metoden}\label{ViewportClickCode}
\end{figure}

Derudover bliver click eventen på \texttt{Input} laget af \texttt{Viewporten} i constructoren sat til at blive håndteret af metoden \texttt{ViewportClick()}. \texttt{ViewportClick()} der kan ses på figur \ref{ViewportClickCode} tjekker hvilket værktøj der er aktivt og kalder den tilsvarende metode.

\begin{figure}[H]
\begin{lstlisting}
public void ToggleTool(ToolStripButton clickedTool)
{
  if (clickedTool.Checked)
  {
    clickedTool.Checked = false;
    ActiveTool = null;
  }
  else
  {
    foreach (ToolStripButton tool in 
      Tools.OfType<ToolStripButton>())
      tool.Checked = false;
    clickedTool.Checked = true;
    ActiveTool = clickedTool;
  }
  StopConnection();
}
\end{lstlisting}
\caption{ToggleTool metoden}\label{ToggleToolCode}
\end{figure}

For at sikre at der ikke er “valgt” flere værktøjer på samme tid, kaldes metoden på figur \ref{ToggleToolCode} \texttt{ToogleTool}, hvergang et værktøj bliver trykket på. Metoden fravælger alle \texttt{ToolStripButtons} i værktøjsListen, hvorefter det nuværende værktøj bliver sat til \texttt{true} (active). Derefter bliver det valgte værktøj sat over i \texttt{ActiveTool} variablen. Til sidst bliver \texttt{StopConnection()} kaldt, som er en metode til at nulstille værktøjets handling, så hvis man f.eks. har valgt \texttt{AddRoad} så vil \texttt{StopConnection()} sikre at det næste klik på gitteret vil tilføje vejens startpunkt og ikke slutpunkt.

\vspace{5mm}

Klassen indeholder som sagt alle værktøjerne og derfor indeholder klassen også en del metoder, derfor vil kun de mest væsentlige værktøjer blive beskrevet.

\begin{figure}[H]
\begin{lstlisting}
private void Add(Type type)
{
  object obj = Viewport.GetObjByGridPos();
  if (obj == null)
  {
    if (type == typeof(Node))
    {
      Project.Nodes.Add(new Node(Viewport.GridPos));
      Viewport.Nodes.Refresh();
    }
    else if (type == typeof(Destination))
    {
      Project.Destinations.Add(new Destination(Viewport.GridPos,  
                                   SelectedDestinationType));
      Viewport.Entities.Refresh();
    }
    else if (type == typeof(LightController))
    {
      Project.LightControllers.Add(new 
        LightController(Viewport.GridPos));
      Viewport.Entities.Refresh();
    }
  }
  else if (obj is Node)
  {
    ((Node)obj).Type = NodeTypes.None;
    Viewport.Nodes.Refresh();
  }
}
\end{lstlisting}
\caption{Add metoden}\label{AddCode}
\end{figure}

\texttt{Add} metoden der ses på figur \ref{AddCode} benyttes til flere værktøjer som f.eks. at tilføje en \texttt{Node}, \texttt{LightController} eller \texttt{Destination}. Udfra typen som bliver sendt fra metodekaldet bestemmes hvilket objekt som skal tilføjes. Hvis den nuværende position i gitteret er en \texttt{Node} bliver \texttt{NodeTypen} sat til \texttt{None} og gitteret vil blive opdateret med \texttt{Refresh()}.

\begin{figure}[H]
\begin{lstlisting}
private void SetNodeType(NodeTypes type)
{
  object obj = Viewport.GetObjByGridPos();
  if (obj is Node)
  {
    if (type == NodeTypes.Light && 
        ((Node)obj).Type == NodeTypes.Light)
      ((Node)obj).Green = !((Node)obj).Green;
    else
      ((Node)obj).Type = type;
    Viewport.Nodes.Refresh();
  }
}
\end{lstlisting}
\caption{SetNodeType metoden}\label{SetNodeTypeCode}
\end{figure}

\texttt{SetNodeType()} som er vist på \ref{SetNodeTypeCode}, benyttes til at give den enkelte \texttt{Node} en type som f.eks. \texttt{Light}, \texttt{Yield}, \texttt{Home}, \texttt{Parking} osv. Metoden modtager en \texttt{NodeType} som bliver bestemt fra \texttt{ViewportClick()}. Hvorefter den checker om objektet på den nuværende position i gitteret er en \texttt{Node}. Hvis det er en \texttt{Node} vil \texttt{NodeTypen} blive sat til den modtaget type. Til sidst vil gitteret blive opdateret med \texttt{Refresh()}.

\begin{figure}[H]
\begin{lstlisting}
private void AddRoad(Partitions partition)
{
 object obj = Viewport.GetObjByGridPos();
 if (obj != null && obj is Node)
 {
    if (_firstNodeConnection)
    {
      _firstNode = (Node)obj;
      _firstNodeConnection = false;
      Viewport.HoverConnection = ((Node)obj).Position;
    }
    else
    {
      _firstNode.Roads.Add(new Road(_firstNode, (Node)obj, 
                               SelectedRoadType, partition));
      if (Control.ModifierKeys == Keys.Shift)
      {
        _firstNode = (Node)obj;
        Viewport.HoverConnection = ((Node)obj).Position;
      }
      else
      {
        _firstNodeConnection = true;
        Viewport.HoverConnection = new Point(-1, -1);
      }
      Viewport.Connections.Refresh();
    }
  }
}
\end{lstlisting}
\caption{AddRoad metoden}\label{AddRoadCode}
\end{figure}

\texttt{AddRoad()} som kan ses på figur \ref{AddRoadCode}, bruges til at tilføje en vej mellem 2 noder, derfor checkes der først om object på den nuværrende positon i gitteret er en node. Hvis det er en node vil der blive checket om \texttt{\_firstNodeConnection} er sket, altså om startpunktet til vejen er blevet valgt. Hvis \texttt{\_firstNodeConnection} er \texttt{true}, betyder det at det ikke er sket, og noden på den nuværrendeposition i gitteret vil blive sat til \texttt{\_firstNode}, og \texttt{\_firstNodeConnection} vil blive \texttt{false}. Det betyder at næste gang brugeren trykker på en node i gitteret vil programmet vide at \texttt{\_firstNode} er blevet sat, og derfor tilføjes der en vej mellem \texttt{\_firstNode} og noden på den nuværrende position i gitteret.

\vspace{5mm}

Hvis brugern holder "Shift" nede imens, vil programmet sætte \texttt{\_firstNode} til den nuværrende \texttt{Node} efter at der er blevet tilføjet en vej, da den \texttt{Node} vil være startpunktet for den næste vej. Det er en implementation som gør det nemmere og hurtigere for brugeren at tilføje veje. 

%//////////////////////////////// FileHandler ////////////////////////////////%
\subsection*{FileHandler}
\texttt{FileHandleren} er sat op så at man har mulighed for at lave et ny projekt, åbne og gemme projektet. Der er blevet dannet tre metoder som håndtere de tre valg for brugeren, for at gøre det mest læsevenligt for dem der skal læse koden. \texttt{FileHandler} gør sig brug af \texttt{BinaryFormatter} for at gemme og åbne de forskellige objekter i binær form. Vi startede ud med at bruge \texttt{XMLSerializer}, da vi lavede \texttt{FileHandleren}. Vi stødte ind på nogle problemer da \texttt{XMLSerializer} skulle læse to objekter som har en reference til hinanden, og det skabte en circular reference som var årsagen til vores program crashede på daværende tidspunkt. Ved denne fejl skiftede vi til \texttt{BinaryFormatter}, da den er i stand til at håndtere en circular reference.

\vspace{5mm}

Metoden \texttt{NewProject} er meget simpel, den åbner et vindue med en \texttt{TextBox}, der beder om et navn til det nye projekt. Hvis et navn blev indtastet vil der så blive oprettet et nyt projekt med det navn, og det vil erstatte \texttt{CurrentProject} i \texttt{GUIMain}.

\begin{figure}[H]
\begin{lstlisting}
static public Project OpenProject()
{
  FileStream file = null;
  try
  {
    OpenFileDialog fileOpen = new OpenFileDialog();
    fileOpen.Filter = "TSP Files|*.tsp";
    if (fileOpen.ShowDialog() == DialogResult.OK)
    {
      BinaryFormatter formatter = new BinaryFormatter();
      file = new FileStream(fileOpen.FileName, FileMode.Open);
      return (Project)formatter.Deserialize(file);
    }
    return null;
  }
  catch (Exception e)
  {
    MessageBox.Show("Error: " + e.Message);
    return null;
  }
  finally
  {
    if (file != null)
    file.Close();
  }
}
\end{lstlisting}
\caption{OpenProject metoden}\label{OpenProjectCode}
\end{figure}

Metoden \texttt{OpenProject}, der vises på figur \ref{OpenProjectCode}, kan åbne et eksisterende projekt, når brugeren trykker på \texttt{Open} i \texttt{File} menuen. Denne metode benytter sig af \texttt{OpenFileDialog}, som ligger under \texttt{System.Windows.Forms}. Koden benytter sig af try-catch-finally, hvor den går ind i try fasen og filtrerer alle andre fil-typer væk som ikke er en TSP fil (traffic simulation project), hvis TSP filen er valgt så vil metoden deserialisere og åbne det gemte projekt op. Sidst vil finally lukke filen, så andre kan komme til.

\vspace{5mm}

I \texttt{FileHandler} klassen findes der også en \texttt{SaveProject()} metode som gemmer projektet, som brugeren har arbejdet på. \texttt{SaveProject()} metoden benytter den samme kode struktur som \texttt{OpenProject()} metoden, som ses på figur \ref{OpenProjectCode}. Metoderne minder meget om hinanden, hvor SaveProjekt gemmer istedet for at åbne et projekt. Til gemning og åbning af simuleringsdata er der metoderne \texttt{OpenSimulation()} og \texttt{SaveSimulation()}, der fungerer på samme måde som \texttt{OpenProject()} og \texttt{SaveProject()}.

%//////////////////////////////// Simulation ////////////////////////////////%
\subsection*{Simulation}\label{SimulationClass}
\texttt{Simulation} klassen indeholder funktionaliteten der bruges til at simulere. Klassen tager to \texttt{BackgroundWorker} instanser i brug til at køre simuleringerne, en til den primære simulering og en til den sekundære. \texttt{BackgroundWorker} er en klasse der kan køre en metode på en ny tråd, så programmet kan multitaske. Begge \texttt{Backgroundworker} instanser er sat til at køre \texttt{Simulate()}, og \texttt{SimulationCompleted()} når \texttt{Simulate()} er færdig. 

\begin{figure}[H]
\begin{lstlisting}
public void Run()
{
  // Find Inbound and Outbound Nodes 

  SetupVehicles();

  Tuple<List<Vehicle>, Project, Partitions> primaryArguments;
  primaryArguments = new Tuple<List<Vehicle>, Project, Partitions>
    (_primaryVehicles, PrimaryProject, Partitions.Primary);
  Tuple<List<Vehicle>, Project, Partitions> secondaryArguments;
  secondaryArguments = new Tuple<List<Vehicle>,Project,Partitions>
    (_secondaryVehicles, SecondaryProject, Partitions.Secondary);

  PrimaryWorker.RunWorkerAsync(primaryArguments);
  SecondaryWorker.RunWorkerAsync(secondaryArguments);
}
\end{lstlisting}
\caption{Run metoden}\label{RunCode}
\end{figure}

Den offentlige \texttt{Run()} metode, som set på figur \ref{RunCode}, bruges til at starte simuleringen. Først bliver \texttt{SetupVehicles} kørt der fylder \texttt{\_primaryVehicles} og \texttt{\_secondaryVehicles} listerne med Vehicle instanser ud fra indstillingerne brugeren har sat. Derefter startes de to \texttt{BackgroundWorker}, hvor de får en liste af køretøjerne der skal simuleres, selve projektet køretøjerne befinder sig i, og en enumerator der viser hvilken simulering der køres. Udover \texttt{Run()}, er \texttt{Cancel()} også offentlig. \texttt{Cancel()} metoden kalder \texttt{CancelAsync()} på de to \texttt{BackgroundWorker} instanser. 

\begin{figure}[H]
\begin{lstlisting}
for (int i = 0; i < MsInDay; i += Project.Settings.StepSize)
{
  if (worker.CancellationPending)
    break;
  if (i % onePercent == 0)
    worker.ReportProgress(i, i / onePercent + "% " + partition);
  foreach (LightController controller in project.LightControllers)
    controller.Update(project.Settings.StepSize);
  for (int j = 0; j < vehicleCount; j++)
    vehicles[j].Drive(i);
}
\end{lstlisting}
\caption{Forløkke i Simulate metoden}\label{SimulateCode}
\end{figure}

\texttt{Simulate()} metoden starter med at udpakke argumenterne, der blev sendt i en \texttt{Tuple}, hvorefter forløkken set på figur \ref{SimulateCode}, bliver startet. For løkken er indkapslet i et try-catch statement, der stopper begge simuleringer i tilfælde af fejl, og rapporterer det til brugerfladen. Inde i forløkken på figur \ref{SimulateCode} tjekkes der først om brugeren har stoppet simuleringen, hvor forløkken så bliver brudt. Derefter tjekkes der om det er tid til at rapporterer fremdriften af loopet, hvilket gøres ved hvert lige procenttal. Brugerfladen \texttt{GUIMenuSimulationRun} der viser fremdriften, abonnerer på \texttt{ProgressChanged} begivenheden på begge \newline \texttt{BackgroundWorker} instanser. Når \texttt{ReportProgress()} bliver brugt, sendes fremdriften samt en besked, der beskriver hvor langt simuleringen er nået, og om det er den primære eller sekundære simuleringen der er tale om. Efter dette bliver alle \texttt{LightController} instanser der findes i projektet opdateret, og sidst køres \texttt{Drive()} metoden på alle køretøjerne. \texttt{Drive()} metoden beskrives i afsnit \ref{Vehicle}.

\begin{figure}[H]
\begin{lstlisting}
private void SimulationCompleted(object sender, 
  RunWorkerCompletedEventArgs args)
{
  if (PrimaryWorker.IsBusy || SecondaryWorker.IsBusy 
      || args.Cancelled)
    return;
  else
  {
    List<VehicleData> primaryData = new List<VehicleData>();
    foreach (Vehicle vehicle in _primaryVehicles)
      primaryData.Add(vehicle.ExtractData());

    List<VehicleData> secondaryData = new List<VehicleData>();
      foreach (Vehicle vehicle in _secondaryVehicles)
      secondaryData.Add(vehicle.ExtractData());

    SimulationData data = new SimulationData(Project, primaryData, 
                                             secondaryData);
    FileHandler.SaveSimulation(data);
    Filename = data.Filename;
    OnSimulationDone();
  }
}
\end{lstlisting}
\caption{SimulationCompleted metoden}\label{SimulationCompletedCode}
\end{figure}

\texttt{SimulationCompleted()}, som kan ses på figur \ref{SimulationCompletedCode}, bliver kaldet en gang for hver simuleringen. Simuleringen bliver derfor først gemt når begge \texttt{BackgroundWorker} instanser ikke er igang med noget, og derudover bliver der heller ikke gemt, hvis simuleringen er blevet afbrudt. Når begge simuleringer er færdige, bliver dataen lagt i en \texttt{SimulationData} instans, der gennem \texttt{FileHandler} bliver konverteret til binær, som senere kan åbnes og ses gennem \texttt{SimulationViewport} klassen. Sidst bliver \texttt{OnSimulationDone()} kaldt, der informerer brugerfladen om at simuleringerne er færdige.