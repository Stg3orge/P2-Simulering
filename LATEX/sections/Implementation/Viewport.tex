\section{Viewport}\label{Viewport}
\texttt{Viewport} klassen har til formål at sætte rammerne for området hvori man kan tegne elementerne i vejnettet. \texttt{Viewport} er opsat til at fungere som en række af lag lagt oven på hinanden. 

\begin{figure}[H]
\begin{lstlisting}
public Project Project;
public Point HoverConnection = new Point(-1, -1);
public Point MousePos = new Point(0, 0);
public Point GridPos { get { return GetGridPos(); } }
\end{lstlisting}
\caption{}
\label{ViewportParameters}
\end{figure}

Klassen arver fra \texttt{Panel} klassen. Til dette formål er det nødvendigt for \texttt{Viewport} and være en del af et nyt projekt og derfor instansieres der en nyt \texttt{Project} og en række parametre set i figur \ref{ViewportParameters}. \texttt{Viewport} visualiserer alle enheder på panelet igennem de datasæt der eksisterer som en del af et \texttt{Project}.

\vspace{5mm}

\texttt{HoverConnection} visualiserer via en svævende streg, den forbindelse man prøver at lave mellem to objekter. \texttt{MousePos} indikerer det aktuelle punkt hvor musen befinder sig i gitter-systemet, selv ved en nedskalering vil den altid finde det samme koordinat. \texttt{GridPos} får data igennem en metode der indikerer alle mulige koordinater i gitteret. Dette er brugbart til at finde koordinaterne til objekterne der bliver tegnet i gitteret.

\begin{figure}[H]
\begin{lstlisting}
public object GetObjByGridPos()
{
  Node node = Project.Nodes.Find(n => 
              n.Position == GridPos);
  if (node != null)
      return node;
  LightController controller = Project.LightControllers.Find(l => 
                               l.Position == GridPos);
  if (controller != null)
    return controller;
  Destination dest = Project.Destinations.Find(d => 
                     d.Position == GridPos);
  if (dest != null)
    return dest;
  return null;
}
\end{lstlisting}
\caption{}
\label{ViewportGetObjByGridPos}
\end{figure}

Som sagt kan \texttt{Viewport} indikere koordinaterne til indsatte objekter og dette gør den igennem metoden \texttt{GetObjByGridPos()} som set i figur \ref{ViewportGetObjByGridPos}. Metoden Tjekker for alle \texttt{Node}, \texttt{LightController} og \texttt{Destination} om deres position er lig \texttt{GridPos}. Hvis den har fundet en, så retunere den det fundne objekt. Grunden til dette er for at være i stand til at arrangere de håndterede objekter så deres koordinat kan identificeres.

\vspace{5mm}

Inde i \texttt{partial} klassen \texttt{ViewportSetup} bliver der angivet en række af lag (\texttt{Layer()}) som set i figur \ref{ViewportSetupLayers}. \texttt{Layer()} er en klasse som arver fra \texttt{PictureBox} og har en constructor med \texttt{DoubleBuffer=true}. Grunden til at \texttt{PictureBox} ikke bare er benyttet alene er fordi måden disse \texttt{Layer()} er anvendt i bl.a. \texttt{ToolController} klassen afhænger af at de kan genindlæses når et objekt, såsom en \texttt{Node}, skal skal ændres eller tegnes en ny. Dette gøres som vist i figur \ref{ToolControllerSetNodeType} fra \texttt{ToolController} klassen. Heri kan der ses at metoden til sidst genindlæser på \texttt{Nodes}. Hvis \texttt{DoubleBuffer} sat til \texttt{true} ville programmet flimre visuelt, fordi objekter ikke bliver tegnet hurtigt nok til at virke omgående.

\begin{figure}[H]
\begin{lstlisting}
private void SetNodeType(NodeTypes type)
{
    object obj = Viewport.GetObjByGridPos();
    if (obj is Node)
    {
        if (type == NodeTypes.Light && ((Node)obj).Type == NodeTypes.Light)
            ((Node)obj).Green = !((Node)obj).Green;
        else
            ((Node)obj).Type = type;

        Viewport.Nodes.Refresh();
    }
}
\end{lstlisting}
\caption{Brug af Control.Refresh() i ToolController klassen}
\label{ToolControllerSetNodeType}
\end{figure}

Disse lag indikerer i hvilken rækkefølge objekterne på panelet skal tegnes. Dette kan ses ved brugen af \texttt{Controls.Add} hvori der gradvist bliver lagt lag på lag. Hvis programmet ikke tegnede objekter i lag, så ville de alle skulle tegnes hver gang man udførte en enkelt opdatering, ligesom set i figur \ref{ToolControllerSetNodeType}.

\begin{figure}[H]
\begin{lstlisting}
public Layer Grid = new Layer();
public Layer Connections = new Layer();
public Layer Nodes = new Layer();
public Layer Entities = new Layer();
public Layer Information = new Layer();
public Layer Input = new Layer();

Controls.Add(Grid);
Grid.Controls.Add(Connections);
Connections.Controls.Add(Nodes);
Nodes.Controls.Add(Entities);
Entities.Controls.Add(Information);
Information.Controls.Add(Input);
\end{lstlisting}
\caption{Lag for ViewportSetup, i deres kendetegnede rækkefølge}
\label{ViewportSetupLayers}
\end{figure}

\subsection{SimulationViewport}\label{SimulationViewport}

I \ref{ViewPortSimulation} bliver der instansieret et nyt lag \texttt{Vehicles} og en variabel for \texttt{SimulationData} hvori indeholdes nødvendige datasæt for at udføre en simulation. \texttt{SimulationViewport} tager derudover \texttt{SimulationData} som en parameter til dens egen constructor som set i figur \ref{ViewportSimulationConstructor}. Formålet med .. TBC

\begin{figure}[H]
\begin{lstlisting}
private const int VehicleSize = 16;
public Layer Vehicles = new Layer();
public SimulationData SimData;
public Partitions CurrentPartition = Partitions.Primary;
\end{lstlisting}
\caption{Variabler til SimulationViewport}
\label{ViewPortSimulation}
\end{figure}

\begin{figure}[H]
\begin{lstlisting}
public SimulationViewport(SimulationData data) : base(data.Project)
{
    SimData = data;
            
    Grid.Paint -= DrawGrid;
    Entities.Controls.Remove(Information);
    Entities.Controls.Add(Vehicles);
    Vehicles.Controls.Add(Input);
    Input.BringToFront();
}
\end{lstlisting}
\caption{Constructoren til SimulationViewport}
\label{ViewportSimulationConstructor}
\end{figure}

Til brug i selve simuleringen bliver der afsat en hel dag i millisekunder (86400000). I \texttt{SimulationSettings} bliver der instansieret en variabel \texttt{Step} på 100 millisekunder. Dette bliver der gået igennem i et \textbf{For-loop} som findes i klassen \texttt{Simulation}. Simuleringen opdateres efter det antal som \texttt{Step} dikterer. \texttt{SimulationViewport} har så til formål at visualisere dette step.

\begin{figure}[H]
\begin{lstlisting}
private int _time = 0;
public int Time
{
    get { return _time; }
    set
    {
        if (value > 86400000) _time = 86400000;
        else if (value < 0) _time = 0;
        else _time = value;
    }
}
\end{lstlisting}
\caption{Tids-afsætning for en simulering, 86400000ms eller en hel dag.}
\label{TimeForSim}
\end{figure}


