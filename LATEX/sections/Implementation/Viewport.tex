\section{Viewport}\label{Viewport}
\texttt{Viewport} klassen har til formål at sætte rammerne for området hvori man kan tegne elementerne i vejnettet.

\begin{figure}[H]
\begin{lstlisting}
public readonly int GridLength = 1000;
public readonly int GridSize = 16;
public readonly int EntitySize = 12;
public readonly int NodeSize = 8;

public Project Project;
public Point HoverConnection = new Point(-1, -1);
public Point MousePos = new Point(0, 0);
public Point GridPos { get { return GetGridPos(); } }
\end{lstlisting}
\caption{}
\label{ViewportParameters}
\end{figure}

Klassen arver fra \texttt{Panel} klassen. Til dette formål er det nødvendigt for \texttt{Viewport} and være en del af et nyt projekt og derfor instansieres der en nyt projekt og en række parametre set i figur \ref{ViewportParameters}. Readonly variablerne for \texttt{GridLength}, \texttt{GridSize}, \texttt{EntitySize} og \texttt{NodeSize} er størrelserne for de visuelle objekter på Viewporten.

\vspace{5mm}

Der bliver instansieret en variabel af typen \texttt{Project} da Viewporten skal kunne repræsentere projektet. \texttt{HoverConnection} visualiserer den forbindelse man prøver at lave mellem to objekter. \texttt{MousePos} indikerer det aktuelle punkt hvor musen befinder sig i gitteret, selv ved en nedskalering vil den altid finde det samme koordinat. \texttt{GridPos} får data igennem en metode der indikerer alle mulige koordinater i gitteret. Dette er brugbart til at finde koordinaterne til objekterne der bliver tegnet i gitteret.

\begin{figure}[H]
\begin{lstlisting}
public object GetObjByGridPos()
{
  Node node = Project.Nodes.Find(n => 
              n.Position == GridPos);
  if (node != null)
      return node;
  LightController controller = Project.LightControllers.Find(l => 
                               l.Position == GridPos);
  if (controller != null)
    return controller;
  Destination dest = Project.Destinations.Find(d => 
                     d.Position == GridPos);
  if (dest != null)
    return dest;
  return null;
}
\end{lstlisting}
\caption{}
\label{ViewportGetObjByGridPos}
\end{figure}

Som sagt kan \texttt{Viewport} indikere koordinaterne til indsatte objekter og dette gør den igennem metoden \texttt{GetObjByGridPos} som set i figur \ref{ViewportGetObjByGridPos}. Metoden Tjekker for alle \texttt{Node}, \texttt{LightController} og \texttt{Destination} om deres position er lig \texttt{GridPos}. Hvis den har fundet en, så retunere den det fundne objekt.