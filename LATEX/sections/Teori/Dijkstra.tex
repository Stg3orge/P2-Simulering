\section{Dijkstras Algoritme}\label{DijkstraTeori}
\begin{wrapfigure}{r}{0.45\textwidth}
    \centering
  \includegraphics[width=0.45\textwidth]{Pictures/Teoriafsnit/Figurfiler/DijkstraGraf.png}
\caption{En vægtet ikke-orienteret graf}\label{dijkstrasgraf}
\end{wrapfigure}

Dijkstra's er en grådig algoritme der kan finde en rute imellem to noder i en vægtet graf. Grafen ruten kan findes i kan både være orienteret og ikke orienteret. Algoritmen starter med at sætte afstanden til alle punkter lig uendelig, udover start punktet da det er det eneste der kendes til på tidspunktet. Når algoritmen kører en graf igennem, kigger den på den node hvor kosten er lavest, og udregner kosten til de naboliggende noder. Når kosten udregnes for en nabo node, tages kosten af ruten til den nuværende node og afstanden til nabo noden bliver adderet. Da algoritmen altid tager noden med den lavest kost, vil den have fundet den korteste rute, når den nuværende node er slutnoden \cite[s. 681-684]{DMATBOGEN}. Som et eksempel kan vejen fra A til Z på figur \ref{dijkstrasgraf} findes:

\begin{enumerate}
\item Algoritmen starter ved A, og kigger på naboerne B og C. Deres kost bliver noteret til at være 4 og 2.
\item Nu kigges der på noden C, da det er den nuværende korteste rute. Her ses det at kosten til B er 3 (2+1), og det bliver overskrevet da det er mindre end den tidligere værdi 4. Deruodver noteres det at kosten for D og E er 10 og 12.
\item Så kigges der på noden B, hvor det ses at kosten til D er 8 (2+1+5), og den overskriver den tidligere højere kost.
\item Derefter bliver noden D kigget på, hvor den noterer kosten for E og Z til at være 10 og 14. Selvom der er fundet en vej til slutnoden, forstættes der, da kosten til noden E er mindre end kosten til Z.
\item Så kigges der på noden E, hvor algoritmen ser at kosten til Z er 13, og dermed bliver den tidligere kost 14 overskrevet.
\item Der er ikke flere noder der har en lavere kost en slutnoden Z, og den korteste rute er fundet.
\end{enumerate}

Kostene for hver node kan gemmes i en tabel, som set på tabel \ref{dijktabel}.
  
\begin{table}[H]
\centering
  \begin{tabular}{|l|l|l|l|l|l|l|}
  \hline
                   & A & B & C & D  & E  & Z  \\ \hline
                   & 0 & $\infty$ & $\infty$ & $\infty$  & $\infty$  & $\infty$  \\ \hline
  A                & 0 & 4 & 2 & $\infty$  & $\infty$  & $\infty$  \\ \hline
  A, C             & 0 & 3 & 2 & 10 & 12 & $\infty$  \\ \hline
  A, C, B          & 0 & 3 & 2 & 8  & 12 & $\infty$  \\ \hline
  A, C, B, D       & 0 & 3 & 2 & 8  & 10 & 14 \\ \hline
  A, C, B, D, E    & 0 & 3 & 2 & 8  & 10 & 13 \\ \hline
  A, C, B, D, E, Z & 0 & 3 & 2 & 8  & 10 & 13 \\ \hline
  \end{tabular}
\caption{Dijkstra Kost Tabel} \label{dijktabel}
\end{table}

\begin{figure}[H]
\begin{lstlisting}
object Node
  Previous = null
  Cost = PositiveInfinity
  List NeighborNodes

procedure FindPath(AllNodes, StartNode, EndNode)
  StartNode.Cost = 0
  List Queue.Add(StartNode)
    while Queue > 0 do
      CurrentNode = Node with smallest Cost in Queue
      if CurrentNode is EndNode
        return TracePath(CurrentNode)
      else
        Queue.Remove(CurrentNode)
        EvaluateNeighbors(CurrentNode)
    end
    return null // No path found
end
\end{lstlisting}
\caption{Dijkstra pseudo-kode: FindPath og Node}\label{DijkstraCodeFindPathNode}
\end{figure}

På figur \ref{DijkstraCodeFindPathNode} ses pseudokode for kernen af Dijkstras algoritme. En \texttt{Node} er defineret først, hvor \texttt{Previous} til at starte med er \texttt{null} eller 'ikke eksisterende', og \texttt{Cost} er uendelig. \texttt{FindPath} proceduren tager en graph \texttt{AllNodes}, startnoden \texttt{StartNode} og slutnoden \texttt{EndNode}. Listen \texttt{Queue} laves og \texttt{StartNode} indsættes, da den skal findes i første iteration af while løkken. While løkken køres mens der stadig er noder der ikke er undersøgt, altså noder der ligger i \texttt{Queue}, og den starter med at sætte \texttt{CurrentNode} til noden der har den laveste \texttt{Cost}. Hvis \texttt{CurrentNode} er lig \texttt{EndNode} så har algoritmen fundet ruten og den bliver returneret ved brug af proceduren \texttt{TracePath}, som beskrives sidst i dette afsnit. Hvis ikke \texttt{CurrentNode} er lig \texttt{EndNode}, så fjernes noden fra \texttt{Queue} listen og naboerne til noden bliver evalueret gennem proceduren \texttt{EvaluateNeighbors} som beskrives herunder.

\begin{figure}[H]
\begin{lstlisting}
procedure EvaluateNeighbors(CurrentNode)
  for each NeighborNode in CurrentNode.NeighborNodes
    if NeighborNode is not in Queue
      Queue.Add(NeighborNode)
      PossibleCost = CurrentNode.Cost 
                     + distance from CurrentNode
                       to NeighborNode
      if NeighborNode.Cost > PossibleCost
        NeighborNode.Cost = PossibleCost
        NeighborNode.Previous = CurrentNode
  end
end
\end{lstlisting}
\caption{Dijkstra pseudo-kode: EvaluateNeighbors}\label{DijkstraCodeEvaluateNeighbors}
\end{figure}

\texttt{EvaluateNeighbors} proceduren som set på figur \ref{DijkstraCodeEvaluateNeighbors}, evaluerer naboerne af en \texttt{Node}, og hvis naboen ikker er i \texttt{Queue}, vil den blive tilføjet til listen. Mens evalueringen er igang så overskriver \texttt{Cost} og \texttt{Previous}, hvis \texttt{Cost} er lavere fra den nuværende node \texttt{CurrentNode}. 

\begin{figure}[H]
\begin{lstlisting}
procedure TracePath(CurrentNode)
  List Path
  while CurrentNode is not null
    Path.Add(CurrentNode)
    CurrentNode = CurrentNode.Previous
  end
  return Reverse(Path)
end
\end{lstlisting}
\caption{Dijkstra pseudo-kode: TracePath}\label{DijkstraCodeTracePath}
\end{figure}

Figur \ref{DijkstraCodeTracePath} viser proceduren \texttt{TracePath}, der finder vejen tilbage fra slutnoden. Dette gøres gennem en while løkke, der køres indtil at den forrige \texttt{Node} er lig \texttt{null}, hvilket vil sige at startnoden er nået. Noderne overføres til en liste, der bliver invereteret og returneret til brugeren af proceduren.