\section{Tidskompleksitet}
Ved brug af tidskompleksitet kan man effektivisere tiden der bliver brugt af en computer til, at løse et problem ved brug af en algoritme. Det er en analyse af hvor lang tid det tager, at løse et problem med en bestemt størrelse. Tidskompleksiteten af en algoritme kan blive udtrykt i en bestemt mængde operationer, det tager for algoritmen af udføre. kompleksiteten kan være beregninger med heltal, dvs. addition, division, multiplikation og andre operationer. Tidskompleksitet er antallet af operationer der kræves i algoritmen og ikke computer tiden det tager at udføre algoritmen. Dette er pga. forskelling i tid det tager for forskellige computere at udføre disse grundlæggende operationer.\cite{Tidskompleksitet}

\subsection{Tidskompleksitet af iterative kontrolstruktur}
På nedenstående pseudo kode er der en funktion A(), denne funktion indeholder en nested for løkke. Først initialiseret to int variabler i og j, derefter bruges de i for løkken til at tælle op med. i den indre for løkke er der et funktions kald til en print funktion, som udskriver "test". For hver gang i tælles op til n gange, så tælles j også op n gange i den indre løkke.  Dvs at print ("test") bliver udført n\^2 gange. Detter betyder også at kompleksiteten er O(n\^2).

\begin{figure}[H]
\begin{lstlisting}
     A()
     {
         int i, int j
         for (i = 1 to n)
             for (j = 1 to n)
                 print("test") 
     }
\end{lstlisting}
\caption{Tidskompleksitet iterativ}\label{Tidskompleksitet}
\end{figure}

