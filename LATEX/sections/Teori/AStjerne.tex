\section{A* Algoritmen}\label{AStjerneTeori}
A* (A stjerne) er en udvidelse af Dijkstras algoritme. Forskellen ved A* algoritmen er at den estimerer hvor langt noderne i graphen er fra slutnoden, og dermed findes den optimale rute hurtigere da den kun kigger på noder der ligger i retningen af slutnoden. Dette gøres når nabo noderne skal undersøges, ligesom i Dijkstra, vil algoritmen udregne kosten til nabo noderne af det nuværende punkt, og derudover vil der benyttes en heuristik til at estimere kosten fra nabo noden til slutnoden. Det vil sige at algoritmen arbejder med kosten til en node, betegnet \texttt{G}, en heuristik der estimerer kosten fra noden til slutnoden, betegnet \texttt{H}, og den samlede vurdering \texttt{F}, der udregnes som vist på formlen \ref{eq:A*}

\begin{equation} \label{eq:A*}
F(n) = G(n) + H(n)
\end{equation}
\vspace{5mm}

Hvis man ønsker at A* skal finde den absolut korteste rute, kræver det at heuristikken er optimistisk, altså at den aldrig overestimerer kosten til slutpunktet. Som et eksempel på en heuristik der kunne man definere heuristikken som værende afstanden i en lige linje til slutpunktet, da der aldrig ville være en kortere vej end dette. Yderligere, hvis man har informationen, kan der inddrages hastighedsgrænsen på vejene ved at dividere afstanden i en lige linje med den højste hastighedsgrænse der findes på vejnettet.

\vspace{5mm}

Estimationerne der bliver udregnet benyttes hver gang algoritmen skal vælge den næste \texttt{Node} der skal undersøges. Ligesom Dijkstra tager den node der har den nuværende laveste \texttt{Cost}, tager A* den Node der har den laveste \texttt{Estimation} \texttt{F}.

\begin{figure}[H]
\begin{lstlisting}
object Node
  Previous = null
  Cost = PositiveInfinity
  Estimate = PositiveInfinity
  List NeighborNodes
\end{lstlisting}
\caption{A* pseudo-kode: Node}\label{AStarCodeNode}
\end{figure}

Forskellen på pseudokoden fra Dijkstras algoritme er lille. Til \texttt{Node} objektet, der ses på figur \ref{AStarCodeNode}, er der tilføjet \texttt{Estimate} variablen, der senere bliver udfyldt af \texttt{EvaluateNeighbors}.

\begin{figure}[H]
\begin{lstlisting}
CurrentNode = Node with smallest Estimate in Queue
\end{lstlisting}
\caption{A* pseudo-kode: Find smallest}\label{AStarCodeFindSmallest}
\end{figure}

I stedet for at sortere køen på \texttt{Cost} som der gøres i Dijkstra, bliver køen nu sorteret på \texttt{Estimate}, som set på figur \ref{AStarCodeFindSmallest}.

\begin{figure}[H]
\begin{lstlisting}
procedure EvaluateNeighbors(CurrentNode)
  for each NeighborNode in CurrentNode.NeighborNodes
    if NeighborNode is not in Queue
      Queue.Add(NeighborNode)
      PossibleCost = CurrentNode.Cost 
                     + distance from CurrentNode
                       to NeighborNode
      if NeighborNode.Cost > PossibleCost
        NeighborNode.Cost = PossibleCost
        NeighborNode.Previous = CurrentNode
        NeighborNode.Estimate = NeighborNode.Cost
                              + Heuristic(NeighborNode)
  end
end
\end{lstlisting}
\caption{A* pseudo-kode: EvaluateNeighbors}\label{AStarCodeEvaluateNeighbors}
\end{figure}

Den sidste ændring er fortaget i \texttt{EvaluateNeighbors}, set på figur \ref{AStarCodeEvaluateNeighbors}. Efter at en nabo \texttt{Node} er blevet fundet til at have en lavere \texttt{Cost} end den tidligere havde, bliver \texttt{Estimate} udregnet ved brug af nodens \texttt{Cost} eller \texttt{G}, plus heuristikken som eksempelvis kunne være distancen fra noden til slutpunktet, som ses på linje 11 og 12 på figur \ref{AStarCodeEvaluateNeighbors}.