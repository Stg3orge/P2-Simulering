\section{A* Algoritmen}
A* (A stjerne) er en udvidelse af Dijkstras algoritme. Forskellen ved A* algoritmen er at den estimerer hvor langt noderne i graphen er fra slutnoden, og dermed findes den optimale rute hurtigere da den kun kigger på noder der ligger i retningen af slutnoden. Dette gøres når nabo noderne skal undersøges, ligesom i Dijkstra, vil algoritmen udregne kosten til nabo noderne af det nuværende punkt, og derudover vil der benyttes en heuristik til at estimere kosten fra nabo noden til slutnoden. Det vil sige at algoritmen arbejder med kosten til en node, betegnet \texttt{G}, en heuristik der estimerer kosten fra noden til slutnoden, betegnet \texttt{H}, og den samlede vurdering \texttt{F}, der udregnes som vist på formlen \ref{eq:A*}

\begin{equation} \label{eq:A*}
F(n) = G(n) + H(n)
\end{equation}
\vspace{5mm}

Hvis man ønsker at A* skal finde den absolut korteste rute, kræver det at heuristikken er optimistisk, altså at den aldrig overestimerer kosten til slutpunktet. Som et eksempel på en heuristik der kunne man definere heuristikken som værende afstanden i en lige linje til slutpunktet, da der aldrig ville være en kortere vej end dette. Yderligere, hvis man har informationen, kan der inddrages hastighedsgrænsen på vejene ved at dividere afstanden i en lige linje med den højste hastighedsgrænse der findes på vejnettet.

\vspace{5mm}

Estimationerne der bliver udregnet benyttes hver gang algoritmen skal vælge den næste node der skal undersøges. Ligesom Dijkstra tager den node der har den nuværende laveste kost, tager A* den node der har den laveste estimation \texttt{F}.

\begin{figure}[H]
\begin{lstlisting}
object Node
  Previous = null
  Cost = PositiveInfinity
  Estimate = PositiveInfinity
\end{lstlisting}
\caption{A* pseudo-kode: Node}\label{AStarCodeNode}
\end{figure}

\begin{figure}[H]
\begin{lstlisting}
CurrentNode = Find Node with smallest Estimate in Open
\end{lstlisting}
\caption{A* pseudo-kode: Find smallest}\label{AStarCodeFindSmallest}
\end{figure}

\begin{figure}[H]
\begin{lstlisting}
procedure EvaluateNeighbors(CurrentNode)
  for each NeighborNode in CurrentNode
      if NeighborNode is not in Open
      and NeighborNode is not in Closed
        Open.Add(NeighborNode)
        NeighborNode.Cost = CurrentNode.Cost 
                          + DistanceTo(NeighborNode)
        NeighborNode.Estimate = NeighborNode.Cost
                              + Heuristic(NeighborNode)
      if NeighborNode.Cost < CurrentNode.Cost
                             + DistanceTo(NeighborNode)
        NeighborNode.Previous = CurrentNode
  end
end
\end{lstlisting}
\caption{A* pseudo-kode: EvaluateNeighbors}\label{AStarCodeEvaluateNeighbors}
\end{figure}