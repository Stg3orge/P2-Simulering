\chapter{Syntese}
Dette kapital har til formål at komprimere alle procesanalysens kapitler ned til hvordan gruppen har udviklet sig igennem projektet, og hvad de mener kan tages med til fremtidige projekter, samt hvordan projektet er forløbet som helhed. Vi har benyttet Start-Stop-Fortsæt metode på de forskellige dele af processen. 

\setlength{\LTleft}{-20cm plus -1fill}
\setlength{\LTright}{\LTleft}
\begin{longtable}{| p{0.3\textwidth} | p{0.3\textwidth} | p{0.3\textwidth} |}
\hline
\textbf{Start}                                                                    & \textbf{Stop}                                                                     & \textbf{Fortsæt}                                                                                                                                            \\ \hline
Ajourføring af planlægningsredskaber på ugentligt basis.                          & Ikke benytte så mange planlægningsredskaber/programmer. Anledning til forvirring. & Trello, da det kan mere end f.eks. grantt-diagram hvis bare det sættes rigtigt op, dog forbedre forståelsen af programmet og hvordan det benyttes fuldt ud. \\ \hline
Sikre sig at alle gruppe medlemmer har forståelse for at anvende disse redskaber. &                                                                                   & Gruppe diskussioner omkring projektet struktur, og forholdsvis ofte.                                                                                        \\ \hline
Strengere håndhævelse af deadlines, til det formål for at holde tidsplanen.       &                                                                                   &                                                                                                                                                             \\ \hline
\caption{Planlægning}\label{PlanlaegningSSF}
\end{longtable}

\newpage

\setlength{\LTleft}{-20cm plus -1fill}
\setlength{\LTright}{\LTleft}
\begin{longtable}{| p{0.3\textwidth} | p{0.3\textwidth} | p{0.3\textwidth} |}
\hline
\textbf{Start}		& \textbf{Stop}		& \textbf{Fortsæt} \\ \hline
Gennemgange af hvad planen er for den kommende dag. Så alle i gruppen har styr på hvilke arbejdsmetode som skal benyttes. & Ikke grupperelateret aktiviteter i grupperummet uden for pauser. & Fortsæt brugen af Discord til kommunikation hjemmefra. \\ \hline
Mere struktur og gruppeplanlægning på programmet, således alle gruppemedlemmer kan være med til at programmere. &  & Fortsæt med at arbejde hjemmefra. Justér efter at nogle ikke altid arbejder bere i grupperum. \\ \hline
Flere codereviews, der var ikke nok i dette projektet.  &  & Fortsæt med gennemgang af det arbejde de enkelte gruppemedlemmer havde lavet, efter en aftalt periode. \\ \hline
Start med faste pauser.  &  & Fortsæt med at bruge Github.  \\ \hline
Gruppemedlemmer skal tage mere initiativ i arbejdet.  &  &  \\ \hline
Begynd at bruge ShareLaTeX til rapportskrivning.  &  &  \\ \hline
\caption{Arbejdsproces}\label{ArbejdsprocesSSF}
\end{longtable}

\setlength{\LTleft}{-20cm plus -1fill}
\setlength{\LTright}{\LTleft}
\begin{longtable}{| p{0.3\textwidth} | p{0.3\textwidth} | p{0.3\textwidth} |}
\hline
\textbf{Start}                                                                                                     & \textbf{Stop} & \textbf{Fortsæt}       \\ \hline
Flere codereviews, der var ikke nok i dette projektet.                                                             &               & Peer-learning          \\ \hline
Gruppen skal møde op til opgaveregning til forlæsningerne, så at gruppen kan hjælpe hinanden og samarbejde om det. &               & Brug af “IO” værktøjer \\ \hline
\caption{Læringsproces}\label{LaeringsprocesSSF}
\end{longtable}

\newpage

\setlength{\LTleft}{-20cm plus -1fill}
\setlength{\LTright}{\LTleft}
\begin{longtable}{| p{0.3\textwidth} | p{0.3\textwidth} | p{0.3\textwidth} |}
\hline
\textbf{Start}                                                                                                                                                 & \textbf{Stop}                                                                              & \textbf{Fortsæt}                                                        \\ \hline
Acceptere at ikke alle arbejder bedst på samme måde og har samme styrker og svagheder.                                                                         & Med at bruge mange værktøjer, i de tilfælde hvor et værktøj kan udføre flere af opgaverne. & Fortsæt brugen af kommunikation værktøjer f.eks. facebook, discord osv. \\ \hline
Større disciplin når der skal arbejdes som gruppe.                                                                                                             &                                                                                            & Individuelt arbejde, og derefter gennemgå det på gruppen.               \\ \hline
Rollefordelingen skal enten være mere konsistent med deres funktion, eller også bør alle gruppemedlemmer forbedrer sig således at alle kan udføre alle roller. &                                                                                            &                                                                         \\ \hline
Med at inddele i større programmerings grupper, så enklte personer ikke sidder med sværer opgaver som de ikke kan løse selv.                                   &                                                                                            &                                                                         \\ \hline
\caption{Gruppesamarbejde}\label{GruppesamarbejdeSSF}
\end{longtable}

\setlength{\LTleft}{-20cm plus -1fill}
\setlength{\LTright}{\LTleft}
\begin{longtable}{| p{0.3\textwidth} | p{0.3\textwidth} | p{0.3\textwidth} |}
\hline
\textbf{Start}                                                                  & \textbf{Stop}                                              & \textbf{Fortsæt}                     \\ \hline
Bedre kommunikation mellem gruppe og vejleder, sende flere spørgsmål over mail. & Stop med at sende materiale efter overskredet aftalt dato. & Med at sende materiale 24 timer før. \\ \hline
Bedre forberedelse til vejledermødet.                                           &                                                            &                                      \\ \hline
Med at bruge vejlederen til programmering, teori, eller finde kilder.           &                                                            &                                      \\ \hline
Lav bedre dagsorden til vejledermøder.                                          &                                                            &                                      \\ \hline
Med at rapportens formatering er mere læselig.                                  &                                                            &                                      \\ \hline
\caption{Vejledersamarbejde}\label{VejledersamarbejdeSSF}
\end{longtable}