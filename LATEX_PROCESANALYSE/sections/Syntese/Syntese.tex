\chapter{Syntese}

Dette kapital har til formål at komprimere alle procesanalysens kapitler ned til hvordan gruppen har udviklet sig igennem projektet, og hvad de mener kan tages med til fremtidige projekter, samt hvordan projektet er forløbet som helhed.

\section{Planlægning}
Tankegangen og idé med hvordan gruppen valgte at håndtere planlægning igennem projektet, var både effektiv og gav gruppen en konstant overblik over projektets fremgang, såfremt at det var blevet ajourført til hvert fredagsmøde som det oprindeligt var tiltænkt. Dette var desværre ikke tilfældet, eftersom gruppen sjældent holdte fredagsmøder og begyndte at arbejde under princippet; en dag af gang og uden en systematisk struktur. Derfor vil det kræve en større disciplin fra alle gruppens medlemmer, hvis denne metode at planlægge projektet på skal fungere i praksis.

Derfor er de vigtigste punkter der kan viderebringes til fremtidige projekter, eller som bør arbejdes på at forbedre som følgende:

\begin{itemize}
\item Ajourføring af planlægningsredskaber på ugentligt basis.
\item Sikre sig at alle gruppe medlemmer har forståelse for at anvende disse redskaber.
\end{itemize}

\section{Arbejdsproces}

Derfor er de vigtigste punkter der kan viderebringes til fremtidige projekter, eller som bør arbejdes på at forbedre som følgende:

\begin{itemize}
\item
\end{itemize}

\section{Læringsproces}

Derfor er de vigtigste punkter der kan viderebringes til fremtidige projekter, eller som bør arbejdes på at forbedre som følgende:

\begin{itemize}
\item
\end{itemize}

\section{Gruppesamarbejde}
Det mest væsentlige problem i gruppe har været disciplinen og strukturen, som har præget gruppen igennem mere end halvdelen af forløbet. En større portion af den tid der var afsat til projektarbejde blev ikke udnyttet når gruppen var samlet, og blev brugt på unødvendige diskussioner, underholdning og fingerpegning som både har været med til at skabe et stærkere sammenhold mellem gruppen, men også splitte og opdele gruppen.

Derfor er de vigtigste punkter der kan viderebringes til fremtidige projekter, eller som bør arbejdes på at forbedre som følgende:

\begin{itemize}
\item Acceptere at ikke alle arbejder bedst på samme måde og har samme styrker og svagheder.
\item Større disciplin når der skal arbejdes som gruppe.
\item Rollefordelingen skal enten være mere konsistent med deres funktion, eller også bør alle gruppemedlemer forbedrer sig således at alle kan udføre alle roller.
\item Det er bedre at benytte sig af få programmer som alle forstår at anvende, end at anvende flere programmer end nødvendigt som kun nogle har sat sig ind i.
\item 
\end{itemize}

\section{Vejledersamarbejde}

Gruppen har nogen gange følt at vejlederen ikke helt forstod hv

Derfor er de vigtigste punkter der kan viderebringes til fremtidige projekter, eller som bør arbejdes på at forbedre som følgende:

\begin{itemize}
\item Bedre kommunikation mellem gruppe og vejleder
\item Over
\end{itemize}

\section{Samarbejdsaftale}

Derfor er de vigtigste punkter der kan viderebringes til fremtidige projekter, eller som bør arbejdes på at forbedre som følgende:

\begin{itemize}
\item
\end{itemize}