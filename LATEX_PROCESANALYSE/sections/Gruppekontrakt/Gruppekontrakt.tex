%!TEX root = ..\..\Main.tex
\chapter{Gruppe Kontrakt\\A319}
\section{Mødetider og pauser}
\begin{enumerate}
\item Der er fastlagte mødetider for gruppen. Såfremt at der ikke er forelæsninger, mødes alle medlemmer i gruppe rummet kl 8.15 og tager hjem igen 16.00. Dette er gældene for alle hverdage pånær fredag, her er det 8.15 til 15.30.
\begin{enumerate}
\item Derudover er der ikke fastlagte pauser, derimod kan der tages pauser efter behov dog inde for rimmelighedens grænser. Disse pauser vare 5 - 10 minutter. Derudover er der en større spisepause i tidsrummet 11.50 til 12.30 såfremt at der ikke er lektioner.
\end{enumerate}
\item Såfremt at der er forelæsning fra 8:15 ændres mødetidspunktet til det tilsvarende tidspunkt for forelæsningens afslutning, og dermed ses der bort fra ovenstående regel
\item Såfremt at et medlem har en lægeaftale eller et andet vigtigt ærinde som gør de bliver nød til at komme senere eller gå før tid, gøres dette klart for resten af gruppen og gerne så snart dette medlem ved besked omkring tidspunkter på hvornår de er fraværende.
\item Der kan i særtilfælde arbejdes hjemmefra, f.eks, hvis en opgave kan laves selv og gruppen umiddelbart ikke har brug for at diskuterer opgavens omfang før et senere tidspunkt.
\begin{enumerate}
\item Isåfald at et medlem arbejder hjemme fra men ikke for løst eller arbejdet tilstrækkeligt på opgaven/opgaverne tildelt dem, så vil medlemmet miste privilegiet til at arbejde hjemmefra.
\end{enumerate}
\end{enumerate}

\section{Gruppemøde og Code Review}
Gruppemøderne bliver afhold en gang i slutningen af hver uge. Derudover holdes der Code Reviews (Så fremt der er kode at fremvise og gå igennem) hver mandag.
\begin{enumerate}
\item Ved dagens start laves der en dagsorden, dette bruges der 10 til 15 minutter på for at opliste dagens arbejde.
\item Ved ugens afslutning afholdes der et gruppemøde hvor i der gennemgås hvad der er blevet lavet igennem ugen.
\end{enumerate}

\section{Gruppens ansvar}
\begin{enumerate}
\item Hvis et gruppemedlem smider deres nøgle væk til grupperummet, hæfter det pågældende medlem for de resterende gruppemedlemmers nøgler, og skal derfor tilbagebetale gruppemedlemmernes depositum inden udgangen af projekforløbet.
\item Opstår der uenighed eller konflikter der kræver at der skal træffes et valg, vil resultatet blive bestemt ved håndsoprækning.
\item Der er mødepligt i grupperummet hver dag med mindre andet aftales på om tirsdagen.
\item I tilfælde af splid i gruppen hvad beslutninger angår, gennemføres dette med valg.
\end{enumerate}

\section{Organisering}
\begin{enumerate}
\item Google Drive bruges til fildeling.
\item Facebook bruges til at melde forsinkelser.
\item Git/Github bliver brugt til version kontrol af rapporten og koden.
\item Latex bliver brugt til at skrive rapporten.
\begin{enumerate}
\item Der laves et .tex document for hver sektion af rapporten, for at undgå version konflikter.
\end{enumerate}
\item Trello til opgave opstilling og fordelling af disse opgaver.
\item Tomsplanner bliver brugt til at lægge tidsplan over projektforløbet.
\item Fremmøde skema laves i regneark gennem Google Drive.
\item Tirsdag laves der status af hvad folk har skrevet, samt der aftales hvornår vi arbejder hjemme. Derudover skal der sættes deadlines på Trello.
\item Ved hjemmearbejde skal der holdes møder over discord 9.30, 12.00 og 14.15. Hvis ikke hjemmearbejde bliver færdig gjort, får man fravær.
\end{enumerate}

\section{Konflikthåndtering}
\begin{enumerate}
\item Hvis et gruppemedlem ikke overholder gruppekontrakten, da løses situationen på følgende vis:
\begin{enumerate}
\item Den involverede person gøres bekendt med situationen ved et ekstraordinært møde.
\item Gruppemedlemmet har en uge til at rette op på problemet.
\item I tilfælde af, at ovenstående punkt ikke er en løsning, da inddrages vejleder.
\item Hvis der ikke kan findes og opstilles et fælles grundlag for den involverede person og resten af gruppen på grundlag af gruppekontrakten, da medfører dette en ekskludering af gruppen.
\end{enumerate}
\item Hvis man har et problem skal man sige det hurtigst muligt, før problemet vokser.
\item Hvis man vil bringe noget op anonymt, skal man fortælle det til Alexander. Det vil derefter blive bragt op til diskussion ved et fredagsmøde.
\end{enumerate}

\chapter{Vejlederkontrakt}
\section{Vejlederen kan forvente at}
\begin{enumerate}
\item Gruppen møder velforberedt op til mødet.
\item Agenda og hele rapporten sendes 24 timer inden mødet sammen med en læsevejledning.
\item Alle i gruppen har læst det materiale, som vi sender til vejleder.
\item Vi svarer på mails indenfor 24 timer.
\item Vi holder vejlederen opdateret på vores projektarbejde via mail og møder.
\end{enumerate}

\section{Vi forventer at vejlederen}
\begin{enumerate}
\item Møder til det aftalte tidspunkt.
\begin{enumerate}
\item Hvis vejlederen bliver forhindret i at møde til det aftalte tidspunkt, skal vejlederen give besked hurtigst muligt.
\end{enumerate}
\item Er forberedt til møderne, det vil sige at have læst det der er blevet sendt.
\item Svarer på mails hurtigst muligt.
\item Har styr på de faglige mål, og dermed kan sige hvis projektet går i en forkert retning.
\end{enumerate}