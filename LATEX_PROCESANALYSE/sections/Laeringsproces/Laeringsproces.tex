\chapter{Læringsproces}\label{Laeringsproces}
Dette kapitel har til formål at give et indblik i hvordan gruppen lært igennem projektet, både fra gruppearbejde såvel som forelæsninger og opgaveregning igennem hele projektet. Selvom hvert gruppemedlem har haft hver deres måde at arbejde på, har alle gruppens medlemmer været flittig til at hjælpe hinanden, såfremt at der var nogen som var i stand til at hjælpe med at afvikle problemet. Hvis ingen i gruppen kunne løse et problem, blev det som regel til et gruppeproblem, som blev forsøgt løst ved, at alle gav deres input til hvordan det givne problem muligvis kunne løses. Til slut vil der blive reflekteret over læringsprocessen som helhed har fungeret.

\section{Vidensdeling}\label{Vidensdeling}
Der blev primært arbejdet med peer-learning igennem hele projektet, både til forelæsninger og projektarbejdet. Peer-learning blev anvendt ved at tage de opgaver på tavlen, som ikke alle vidste hvordan skulle løses, og gennemgå det på forskellige måder indtil alle mente at de havde forstået hvad der var foregået.

\section{Reflektion over læringsprocessen}\label{Reflektion-over-laeringsprocessen}
I starten af projektet var alle gruppens medlemmer gode til at møde op til forelæsninger, samt til at hjælpe hinanden med at løse de opgaver som blev stillet til den enkelte forelæsning og de ting som de forskellige medlemmer havde problemer med at forstå. Længere inde i perioden begyndt det at variere meget med hvem der mødte op til forelæsningerne, og hvor meget gruppen arbejdede sammen under opgaveregningen. Dette skyldtes at gruppen måtte acceptere, at det ikke var alle medlemmer som fik det samme udbytte af forelæsningerne og opgaveregningen og fik mere ud af at bruge tiden anderledes. Derfor blev det også op til det enkelte medlem om, hvorvidt de ville være en del af det, som ikke var direkte knyttet til projektarbejdet, så længe man stadigvæk var en del af projektet. Til at starte med blev dette set som et problem i gruppen, men det endte med at der blev en større respekt for gruppemedlemmernes forskellighed, og derfor ikke virkede logisk at tvinge de medlemmer som ikke fik det store udbytte af forelæsning eller opgaveregning, til at være en del af det.
	I projektarbejdet har det varieret meget, med hvordan det har været muligt med at hjælpe gruppens medlemmer. Det har ikke altid været muligt for gruppens medlemmer at hjælpe direkte med det gældende problemet, eftersom ikke alle har siddet og arbejdet direkte med det samme. Derfor fungerede det primært ved at man stillede et spørgsmål efterfulgt af en forklaring på hvad man havde gjort indtil videre, indtil det var muligt  opgaven på egen hånd, eftersom det ikke 