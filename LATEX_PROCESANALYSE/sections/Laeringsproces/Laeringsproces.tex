\chapter{Læringsproces}\label{Laeringsproces}
Dette kapitel har til giver et indblik i hvordan gruppen lært igennem projektet, både fra gruppearbejde såvel som forelæsninger og opgaveregning igennem hele projektet

\section{Vidensdeling}\label{Vidensdeling}
Der blev primært arbejdet med peer-learning igennem hele projektet både til forelæsninger og projektarbejdet, hvor gruppemedlemmerne hjalp hinanden når det var
Hovedsageligt blev viden

\section{Reflektion over læringsprocessen}\label{Reflektion-over-laeringsprocessen}
I starten af projektet var alle gruppens medlemmer gode til at møde op til forelæsninger, samt til at hjælpe hinanden med at løse de opgaver som blev stillet til den enkelte forelæsning og de ting som de forskellige medlemmer havde problemer med at forstå. Længere inde i perioden begyndt det at variere meget med hvem der mødte op til forelæsningerne, og hvor meget gruppen arbejdede sammen under opgaveregningen. Dette skyldtes at gruppen måtte acceptere, at det ikke var alle medlemmer som fik det samme udbytte af forelæsningerne og opgaveregningen og fik mere ud af at bruge tiden anderledes. Derfor blev det også op til det enkelte medlem om, hvorvidt de ville være en del af det som ikke var direkte knyttet til projektarbejdet, så længe man stadigvæk var en del af projektet.