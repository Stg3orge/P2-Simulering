\chapter{Samarbejdsaftale}\label{Samarbejdesaftale}
I dette kapitel bliver samarbejdsaftalen skulle bruges i det tilfælde at der opstod konflikter eller problemer, hvor gruppen ikke kunne nå til enighed om en løsning eller et kompromis i fællesskab. Dette gav også alle gruppemedlemmer mulighed for at kunne bringe et problem eller der opstod en strid om hvordan. Den afsluttende samarbejdsaftale er vedlagt som bilag, efter at den blev redigeret sidste gang. Til slut vil der blive reflekteret over samarbejdsaftalen som helhed har fungeret.

\section{Samarbejdsaftale}

Den første samarbejdsaftale blev udviklet med bagtanken om at den skulle tages op til revidering efter den første måned var overstået, således at gruppens medlemmer havde mulighed for at protestere imod nogle af de mente ikke fungerede, skulle udbygges eller fjernes fra aftalen. Dette skyldtes mest at der var en større uenighed i gruppen om hvilke måde der skulle arbejdes projekt-relateret. Nogle mente at der skulle være mødepligt hver dag fra 08:15 til 16:15 inklusiv forelæsninger, mens andre mente at forelæsninger var op til det enkelte medlem, hvor andre slet ikke mente at der skulle mødes hver dag, men benyttes stemme-kommunikationsprogrammer. 
Efter at gruppen havde udformet den første gruppekontrakt, blev den ikke anvendt, før der begyndt at blive diskuterede hvordan gruppen skulle arbejde, da flere af gruppens medlemmer viste utilfredshed med den daværende måde at arbejde på, hvorfor gruppen besluttede at redigere gruppekontraktten således at den blev tilpasset en arbejdsmetode som alle gruppens medlemmer kunne arbejde videre ud fra. Aftalen ændrede sig til at der var mulighed for at aftale om tirsdagen om man skulle møde om torsdagen eller om dagen kunne blive arbejdet igennem hjemmefra. Til dette formål blev der benyttet stemme-kommunikationsprogrammer gennem hele dagen så der var rig mulighed for at stille spørgsmål og gennemgå arbejde med de andre medlemmer.

\section{Reflektion over samarbejdsaftalen}\label{Reflektion-over-samarbejdsaftale}
Da gruppekontrakten var blev udformet, blev den ændret undervejs over den første måneds tid. Dette skyldtes at gruppen havde svært ved at tilpasse en arbejdsform som tilgodeså alle gruppemedlemmers ønsker vedrørende ting såsom mødetider, mødepligt, samt hvordan folk lærte og arbejder bedst og hvor de følte at de fik mest udbytte for deres tid. Derudover havde vi problemer med at gruppemedlemmer blev ofte distraheret af hinanden af både sociale såsom projektorienterede samtaler og andre forstyrrelser, hvilket ledte til at vi en gang imellem havde nogle meget uproduktive dage. Eftersom vi nåede frem til en ændring halvvejs inde i projektet vedrørende samarbejdsaftalen ledte det til nogle mere produktive dage. Dog gav dette også anledning til at flere medlemmer ikke mødte til en del forelæsninger herefter, og til tider var der kune 2 til 3 personer til opgaveregning. Redigeringen af samarbejdsaftalen skete kun én gang i projektet af en række grunde. Den første grund var at der fra start var et ønske fra mange medlemmer at mødepligten blev holdt fra morgen til eftermiddag, da de mente det var den bedste måde at føre gruppearbejdet på. Revideringen skete som sagt, på baggrund af den diskussion der blev taget om arbejdsformen. Arbejdsindsatsen gik derefter op efter aftalen blev indført og der var ingen grund til videre revidering, følte gruppen. Dog hen ad slutningen af projektet blev samarbejdsaftalen i forhold til hvilke dage man mødte osv. tilsidesat. Dette betød at der løbende blev aftalt dage, hvor man arbejdede hjemme og hvordan arbejdet blev uddelegeret. Medlemmer aftalte indbyrdes om hvad de kunne foretage sig arbejdsmæssigt fremadrettet. Denne fremgangsmåde virkede overraskende godt da projektets tidsramme nåede sit slutpunkt.