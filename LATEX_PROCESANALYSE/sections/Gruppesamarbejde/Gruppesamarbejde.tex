\chapter{Gruppesamarbejde}\label{Gruppesamarbejde}
Dette kapitel har til at give et indblik i hvordan gruppen har har samarbejdet under projektet, samt hvordan gruppens har 

\section{Rollefordeling}\label{Rollefordeling-1}

Noget af det først der blev vedtaget efter gruppedannelsen, var at blive enige om hvorvidt om deres skulle indføres grupperoller og hvordan vores gruppedemokrati skulle foregå igennem hele forløbet. Det blev vedtaget at indføre en rollefordeling, såfremt at der var nogle som mente at de var kvalificeret til det, eller havde lyst til at prøve at være det. Såfremt der var flere som gerne vil udfylde samme rolle, eller skulle man undervejs i forløbet få lyst til at prøve det, ville der blive indført en form for rotationsskift, således at dem som gerne ville prøve at udfylde en rolle, kunne få lov til at prøve det.
	Det viste sig dog at det ikke var alle som havde lyst til at udføre en rolle, og dem som gerne ville prøve at udfylde en rolle, ikke havde lyst til at have de samme roller. Det var derfor muligt for alle gruppens medlemmer at få deres ønske opfyldt til rollefordelingen, som endte med at blive som følgende:

\begin{enumerate}
\item Tidsplanlægger: Niclas
\item Opgavefordeler: Jacob
\item Referent: Lasse
\item Ordstyrer: Alexander
\end{enumerate}

Disse roller havde hvert deres formål igennem forløbet, og deres funktioner vil blive uddybet yderligere. Igennem forløbet blev disse rollers funktioner dog udført af andre eller helt udeladt, hvis der eksempelvis ikke var det store behov for det, eller hvis vedkommende som eksempelvis referenten var syg, og det var essentielt med et referat den følgende dag. På trods af at det i starten af projektet blev inddelt roller, så blev det mere flydende over tiden, og dem som ikke ville have roller til at starte med, påtog sig 
	
\subsection{Ordstyrer}\label{Ordstyrer-1}
Ordstyren udførte flere funktioner i forløbet. Den primære funktionen denne rolle udgjorde var at starte ud med at lave en dagsorden, og sikre sig at alle ikke snakkede i munden på hinanden når der blev diskuteret. Selvom dette job tilfaldt ordstyren, var det op til hele gruppen at sikre sig at alle i sad og snakkede i munden på hinanden, og at diskussioner ikke tog overhånd eller begyndte at brede sig til irrelevant snak og fokuserede på emnet.  og agere som mellemmand mellem gruppen og det enkelte medlemmer skulle der opstå situationer, hvor et medlem gerne ville have noget diskuteret i gruppen anonymt. 

\subsection{Referent}\label{Referent}
Referentens funktion var at notere alle relevante detaljer og diskussioner omkring projektet ned hver dag og til vejledermøder. Dette gjorde det muligt at gå tilbage og se  at skulle der opstå tvivl, om hvad der var blevet tidligere og hvad vi var nået frem til v. Det gjorde det også muligt for gruppemedlemmer der havde været fraværende de pågældende dage, at tjekke hvad der havde foregået og hvad der var blevet lavet. Derudover var det referents job at notere ned hvis nogen kom for sent, ikke mødte op eller var syg samt om der var en grund til fraværet, såsom arbejde eller private forhold. Det var dog op til det enkelte medlem selv at skrive ind såfremt referenten ikke selv var tilstede, eller hvis man vidste at der var dage man ikke kunne møde.  

\subsection{Tidsplanlægger}\label{Tidsplanlaegger}
Tidsplanlæggeren havde til formål at udforme en plan over hvor lang tid gruppen skulle forvente at bruge på forskellige dele af rapporten, men også tilpasse den efterhånden som projektet skred frem. Såfremt opgaver blev forsinket eller tog længere tid end forventet skulle tidsplanen revideres, således at det gav gruppen et overblik hvor hvor meget tid

\subsection{Opgavefordeler}\label{Opgavefordeler}
Opgavefordeleren havde til opgave at uddele opgaver til hvert gruppemedlem, samt at sikre sig at alle havde noget at lave og at opgaverne var struktureret så der ikke altid var en der sad med en større arbejdsbyrde end andre. Alle gruppens medlemmer havde dog mulighed for selv at gå ind og sætte sig på opgaver i Trello eller tilføje nye opgaver, såfremt der skulle opstå nye problemer der skulle løses. Dette gav et overblik for alle gruppens medlemmer, eftersom det var muligt for alle at se hvilke opgaver som var løst, og hvilke stadigvæk var under udarbejdelse. Det gjorde det muligt for alle gruppemedlemmer at til 

\section{Fredagsmøder}\label{Fredagsmoeder}
Hver fredag blev brugt på at holde et fredagsmøde enten i starten eller slutningen af dagen, alt efter hvordan det passede gruppens medlemmer bedst den pågældende fredag. Mødet blev brugt til at give alle gruppemedlemmer muligheden for at give positiv såvel som negativ feedback til gruppen eller enkelte medlemmer, men også til at afvikle spørgsmål der var opstået igennem ugen, som man ikke følte var blevet afklaret. Det var også en mulighed for at fortælle hvad man synes der skulle gøres anderledes i de kommende uger, som kunne være med til at forbedre gruppens præstation og effektivitet.

\section{Gruppekommunikation}\label{Gruppekommunikation}
Der blev under projektet brugt flere former for kommunikation der havde hvert deres formål. Udover de tiltænkte programmer som det blev besluttet at anvende under gruppedannelsen, som 

\subsection{Facebook}\label{Facebook}
Facebook startede ud med udelukkende at blive brugt til det sociale aspekt i gruppen, samt ting som forsinkelser og til at få fat i gruppemedlemmer. Den første måneds tid overtog det meget det som slack var tiltænkt,  

\subsection{Slack}\label{Slack}
Slack var tiltænkt at det skulle som en faglig form for Facebook, hvor det udelukkende var meningen at projekt- og forelæsningsrelateret materiale og beskeder skulle sendes igennem. Det vil således spare gruppens medlemmer en masse tid at skulle gennemrode hele Facebook korrespondancen igennem for et link eller besked der tidligere var blevet postet   

\subsection{Google Drev}\label{GoogleDrev}
GoogleDrev blev brugt som gruppens lager for alt materiale der blev brugt og lavet igennem forløbet, med undtagelse af LaTeX filer og simulationsprogrammet der var forbeholdt GitHub, dog blev pdf versionen af projektrapporten fra LaTeX og det kompilet simulationsprogram lagt op, således at man altid hurtigt kunne finde den nyeste version som var blev lavet.   

\subsection{GitHub}\label{GitHub}
GitHub blev brugt som versionskontrol til både LaTeX og programmering, selvom der blev eksperimenterede med programmer som bl.a. ShareLatex og VS Anywhere endte det altid med at gruppen valgte at gå tilbage til GitHub, som det blev besluttede at benytte allerede startfasen af gruppearbejdet, eftersom der altid endte med at opstå problemer med de andre programmer, der kostede gruppen mange timer at fikse hver gang en fejl skete, hvilket ofte var tilfældet.

\subsection{Skype og Discord}\label{SkypeOgDiscord}
Når der skulle arbejdes hjemmefra eller hvis et gruppemedlem ikke havde mulighed for at møde, men stadigvæk gerne ville være en del af dagens arbejde, blev der brugt en form for voice-kommunikation program såsom Skype eller Discord. Til at starte med brugte vi Skype hvilket virkede fint, men valgte senere at skifte over til Discord, eftersom der begyndte at blive problemer der gjorde at vi kunne bruge op til en time på at få alle med i opkaldet. Discord og Skype var en god hjælp til at gøre gruppen i stand til at arbejde på afstand, da man stadigvæk havde mulighed for at snakke samme, dele screenshots og andet materiale igennem. Selvom det først var mere i den sidst halvdel af projektet at kommunikationsprogrammerne blev anvendt, fortrød vi at det ikke blev brugt mere ved eventuelt sygdom eller manglende fremmøde, så man stadigvæk kunne være up-to-date med hvad der skete og hvilke beslutninger der blev truffet i deres proces.  

\section{Reflektion over gruppesamarbejdet}\label{Reflektion-over-gruppesamarbejdet}

Over perioden var der både op og nedture i gruppen, både socialt såvel som projektrelateret, men har været med til at udvikle gruppen samt at give gruppens medlemmer erfaringer til fremtidens projekter og grupper. Et af de store problemer i gruppe har været manglende disciplin omkring mange af de forskellige dele. Blandt andet blev alle de forskellige redskaber ikke altid holdt up to date, eller benyttet som de var tiltænkt. Dette gav anledning til spildtid der blev brugt på at 