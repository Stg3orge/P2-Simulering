\chapter{Gruppesamarbejde}\label{Gruppesamarbejde}
Dette kapitel har til formål at give et indblik i hvordan gruppen har samarbejdet under projektet, samt hvilke redskaber de har valgt at benytte sig af og hvordan disse blev benyttet. Til slut vil der blive reflekteret over gruppesamarbejdet som helhed har fungeret.

\section{Rollefordeling}\label{Rollefordeling-1}

Noget af det først der blev vedtaget efter gruppedannelsen, var at blive enige om hvorvidt om deres skulle indføres grupperoller og hvordan gruppens gruppedemokrati skulle foregå igennem hele forløbet. Det blev vedtaget at indføre en rollefordeling, såfremt at der var nogle som mente at de var kvalificeret til det, eller havde lyst til at prøve at være det. Såfremt der var flere som gerne vil udfylde samme rolle, eller skulle man undervejs i forløbet få lyst til at prøve det. Det var derfor muligt for alle gruppens medlemmer at få deres ønske opfyldt til rollefordelingen, som endte med at blive som følgende:

\begin{itemize}
\item Tidsplanlægger: Niclas
\item Opgavefordeler: Jacob
\item Referent: Lasse
\item Ordstyrer: Alexander
\end{itemize}

Disse roller havde hvert deres formål igennem forløbet, og deres funktioner vil blive uddybet yderligere. Igennem forløbet blev disse rollers funktioner dog udført af andre eller helt udeladt, hvis der eksempelvis ikke var det store behov for det, eller hvis vedkommende som eksempelvis referenten var syg, og det var essentielt med et referat den følgende dag. På trods af at det i starten af projektet blev inddelt roller, så blev det mere flydende over tiden, og dem som ikke ville have roller til at starte med, overtog en rolle for en dag, hvis en af dem der havde en rolle var syg.
	
\subsection{Ordstyrer}\label{Ordstyrer-1}
Ordstyren udførte flere funktioner i forløbet. Den primære opgave denne rolle havde, var at starte ud med at lave en dagsorden, og sikre sig at alle ikke snakkede i munden på hinanden når der blev diskuteret. Selvom dette job tilfaldt ordstyren, var det op til hele gruppen at sikre sig at alle i sad og snakkede i munden på hinanden, og at diskussioner ikke tog overhånd eller begyndte at brede sig til irrelevant snak og fokuserede på emnet. Ordstyreren agere som mellemmand mellem gruppen, hvis der skulle opstå situationer, hvor et medlem gerne ville have noget diskuteret i gruppen anonymt. 

\subsection{Referent}\label{Referent}
Referentens funktion var at notere alle relevante detaljer og diskussioner omkring projektet ned til ved hver gruppe- og vejleder-møde. Dette gjorde det muligt at gå tilbage og se, hvis der skulle opstå tvivl, om hvad der var blevet tidligere og hvad vi var nået frem til. Det gjorde det også muligt for gruppemedlemmer der havde været fraværende de pågældende dage, at tjekke hvad der havde foregået og hvad der var blevet lavet. Derudover var det referents job at notere ned hvis nogen kom for sent, ikke mødte op eller var syg samt om der var en grund til fraværet, såsom arbejde eller private forhold. Det var dog op til det enkelte medlem selv at skrive ind såfremt referenten ikke selv var tilstede, eller hvis man vidste at der var dage man ikke kunne møde.  

\subsection{Tidsplanlægger}\label{Tidsplanlaegger}
Tidsplanlæggeren havde til formål at udforme en plan over hvor lang tid gruppen skulle forvente at bruge på forskellige dele af rapporten, men også tilpasse den efterhånden som projektet skred frem. Såfremt opgaver blev forsinket eller tog længere tid end forventet skulle tidsplanen revideres, således at det gav gruppen et overblik hvor meget tid man havde til at blive færdig.

\subsection{Opgavefordeler}\label{Opgavefordeler}
Opgavefordeleren havde til opgave at oprette opgaver til projektet, hvor så gruppen gik ind og sikre sig at man var sat på noget at lave. Alle gruppens medlemmer havde dog mulighed for selv at gå ind og sætte sig på opgaver i Trello eller tilføje nye opgaver, såfremt der skulle opstå nye problemer der skulle løses. Dette gav et overblik for alle gruppens medlemmer, eftersom det var muligt for alle at se hvilke opgaver som var løst, og hvilke stadigvæk var under udarbejdelse.

\section{Fredagsmøder}\label{Fredagsmoeder}
Hver fredag blev brugt på at holde et fredagsmøde enten i starten eller slutningen af dagen, alt efter hvordan det passede gruppens medlemmer bedst den pågældende fredag. Mødet blev brugt til at give alle gruppemedlemmer muligheden for at give konstruktivt feedback til gruppen eller enkelte medlemmer, men også til at afvikle spørgsmål der var opstået igennem ugen, som man ikke følte var blevet afklaret. Hvis der var nogle der havde lavet noget kode, som de ville vise til gruppen, så benyttede man fredagsmødet til dette. Det var også en mulighed for at fortælle hvad man synes der skulle gøres anderledes i de kommende uger, som kunne være med til at forbedre gruppens præstation og effektivitet. Møderne var også til at holde status om hvor langt folk i gruppen var, og hvad der skulle laves til dagen efter eller i weekenden.

\section{Gruppekommunikation}\label{Gruppekommunikation}
Der blev under projektet brugt flere former for kommunikation der havde hvert deres formål. Udover de tiltænkte programmer som det blev besluttet at anvende under gruppedannelsen, blev der også eksperimenterede med andre programmer, for at vurdere om det var bedre egnet til det formål gruppen havde tiltænkt det. Det endte dog altid med at gruppen valgte at gå tilbage til de oprindelige udvalgte programmer, eftersom de programmer der blev eksperimenterede med, altid endte med at være til mere gene end gavn for gruppen. Derudover var det tidskrævende at skulle sætte alle gruppens medlemmer ind i et nyt program, hver gang det skulle anvendes, hvor der allerede havde været brugt tid, til at lære gruppens medlemmer at anvende de programmer, som det var besluttet skulle anvendes i starten.

\subsection{Slack}\label{Slack}
Slack var tiltænkt at det skulle som en faglig form for Facebook, hvor det udelukkende var meningen at projekt- og forelæsningsrelateret materiale og beskeder skulle sendes igennem. Det vil således spare gruppens medlemmer en masse tid at skulle gennemrode hele Facebook korrespondancen igennem for et link eller besked der tidligere var blevet postet, imellem alle de socialt relateret beskeder. I begyndelsen af projektet blev slack anvendt som tiltænkt, hvor fagligt og socialt var adskilt imellem Facebook og Slack, men efter idéudviklingsfasen ophørte brugen af Slack til Facebook, hvilket ikke var en bevidst beslutningen der blev taget i gruppen, men blot var mere belejligt for alle gruppens medlemmer, eftersom alle altid havde Facebook ved hånden, og det sociale og faglige aspekt af projektet blev mere blandet. 

\subsection{Facebook}\label{Facebook}
Facebook startede ud med udelukkende at skulle bruges til det sociale aspekt i gruppen, samt ting som at gøre gruppen opmærksomme på forsinkelser og til at få fat i gruppemedlemmer. Da idéudviklingsfasen var overstået efter den første måneds tid overtog det meget det som slack var tiltænkt. Næsten alt kørte primært igennem Facebook, indtil gruppen blev introduceret for Discord\ref{SkypeOgDiscord}, som blev det sted hvor alt det projektrelateret materiale blev delt.      

\subsection{Google Drev}\label{GoogleDrev}
GoogleDrev blev brugt som gruppens lager for alt materiale der blev brugt og lavet igennem forløbet, med undtagelse af LaTeX filer og simulationsprogrammet der var forbeholdt GitHub, dog blev pdf-versionen af projektrapporten fra LaTeX og det kompilet simulationsprogram lagt op, således at man altid hurtigt kunne finde den nyeste version som var blev lavet. Dette gav også muligheden for alle gruppemedlemmer at finde hinandens materiale, arbejdspapir og hvad man ellers havde benyttet under projektet, som man mente kunne være en hjælpe til andre.

\subsection{GitHub}\label{GitHub}
GitHub blev brugt som versionskontrol til både LaTeX og programmering, selvom der blev eksperimenterede med programmer som bl.a. ShareLatex og VS Anywhere endte det altid med at gruppen valgte at gå tilbage til GitHub, som det blev besluttede at benytte allerede startfasen af gruppearbejdet, eftersom der altid endte med at opstå problemer med de andre programmer, der kostede gruppen mange timer at fikse hver gang en fejl skete, hvilket ofte var tilfældet. GitHub havde også den fordel at det var muligt at gå tilbage i hvad andre tidligere havde skrevet, når først det var blevet uploadet til gruppens server. Skulle et gruppemedlem være uheldig at overskrive et andet gruppemedlems arbejde, eller besluttede at noget der tidligere var skrevet var bedre en det nuværende, var det muligt at gå tilbage og finde det frem igen. Funktionen til at gå tilbage til en tidligere version, reddet flere gange gruppen fra fejltagelser, som overskrev flere dages arbejde, så gruppen ikke skulle gendanne det tabte arbejde en gang til. 

\subsection{Skype og Discord}\label{SkypeOgDiscord}
Når der skulle arbejdes hjemmefra eller hvis et gruppemedlem ikke havde mulighed for at møde, men stadigvæk gerne ville være en del af dagens arbejde, blev der brugt en form for voice-kommunikation program såsom Skype eller Discord. Til at starte med brugte vi Skype hvilket virkede fint, men valgte senere at skifte over til Discord, eftersom der begyndte at blive problemer der gjorde, at vi kunne bruge op til en time på at få alle med i opkaldet. Discord og Skype var en god hjælp til at gøre gruppen i stand til at arbejde på afstand, da man stadigvæk havde mulighed for at snakke samme, dele screenshots og andet materiale igennem. Selvom det først var mere i den sidst halvdel af projektet at kommunikationsprogrammerne blev anvendt, fortrød vi at det ikke blev brugt mere ved eventuelt sygdom eller manglende fremmøde, så man stadigvæk kunne være up-to-date med hvad der skete og hvilke beslutninger der blev truffet i deres proces.

\section{Reflektion over gruppesamarbejdet}\label{Reflektion-over-gruppesamarbejdet}
Over perioden var der både op og nedture i gruppen, både socialt såvel som projektrelateret, men har været med til at udvikle gruppen samt at give gruppens medlemmer erfaringer til fremtidens projekter og grupper. Det største problem i gruppen har været manglende disciplin. Disciplinen dalede i takt med at forløbet skred frem, og gruppens medlemmer begyndte at være mere tilbøjelig til at møde for sent op og lave andet end projekt, indtil vi nåede udgangen af projektet, hvor motivationen var på samme højde som da gruppen startede ud. Rollefordelingerne som blev mere flydende som projektet skred frem, var dog en form for modsvar på den manglende disciplin og ikke blot fordi, at dem som havde påtaget sig de forskellige roller ikke ville udfører dem. Det skyldtes at gruppens medlemmer valgte at prøve at skubbe gruppen i gang, ved at gå forrest og prøve at motivere resten ved at gå i gang med gruppearbejde. Det kom også af at gruppens medlemmer begyndte at føle et større ansvar for gruppens fremdrift, og ikke blot kunne give dem som havde påtaget sig en rolle ansvaret for at gruppen rykkede sig i den rigtige retning.
	Specielt når det kom til de anvendte redskaber kontra de planlagte redskaber, blev der tegnet et tydeligt billede af manglende disciplin. Alle de planlagte redskaber blev ikke altid holdt up-to-date, eller benyttet som de var tiltænkt med undtagelse af GitHub, eftersom det var det som indeholdte hele projektet. Dette kan både ses som en fordel og ulempe. Det var en ulempe eftersom der blev spenderet meget tid på at benytte, opsætte og lære programmer og redskaber, som alligevel blev skrottet til fordel for de planlagte redskaber og programmer. Det var en fordel i de tilfælde, hvor de redskaber og programmer som blev eksperimenterede med erstattede de programmer og redskaber som var planlagt skulle benyttes. Derfor vil det for fremtiden være smart at undersøge hvilke redskaber og programmer, som kan udfører mere end en enkelt funktion og som medlemmer allerede har kendskab til. Formålet med dette er at sikre, at der ikke benyttes flere programmer end nødvendigt, i stedet for at have et program og redskab med hvert sit gøremål, hvis det alligevel ikke bliver anvendt men forsømt i stedet for. 
	Det sociale aspekt af gruppen var til tider presset, fordi at der var så stor forskellighed i gruppen og ikke alle gjorde alt på samme måde som andre. Gruppen var tilgengæld gode til at bonde, fordi at alle gruppens medlemmer havde en fælles fritidsinteresse tilfælles: videospil. Dette er dog noget som ikke er noget der kan forbedres, men mere et individuelt problem at folk skal lære at acceptere at vi alle er forskellige. Men det var en fordel at finde ting som gruppens medlemmer havde tilfælles, som kunne bruges til at samle gruppen igen, når der opstod splid.