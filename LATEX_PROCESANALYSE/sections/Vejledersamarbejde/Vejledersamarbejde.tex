\chapter{Vejledersamarbejde}\label{Vejldersamarbejde}
I forhold til 1. semester projektet, hvor der blev tildelt grupperne både en hoved- og bivejleder, er der på 2. semester kun blevet tildelt en enkelt vejleder til gruppen, hvortil gruppe A319 blev tildelt Anders Mariegaard som vejleder. Igennem projektet har gruppens vejleder været behjælpelig ikke kun med at guide gruppen igennem forløbet, men også til at finde materiale samt problemer med programmering og generalle spørgsmål omkring studiet med mere. 

\section{Vejledermøder}\label{Vejledermoeder}
Gruppen har hovedsagelig haft et vejledermøde mindst en gang om ugen, såfremt det kunne lade sig gøre for både vejleder og gruppe, men har dog undladt det i perioder hvor gruppen gruppen havde andet at fokusere på såsom eksamensopgaver og ferier, hvor projektrelateret arbejde blev sat i bero. Dagen forinden blev vejledermøderne blev det materiale gruppen ville have gennemgået sendt til vejlederen med og der ikke var 

\section{Vejlederkommunikation}\label{Vejlederkommunikation}
Både op til og efter vejledermøder har gruppen diskuteret og gennemgået hvad gruppen mener at der skulle diskuteres til vejledermøderne hvortil der blev udformet et agenda, og tilsendt vejleder. Når et vejledermøde var overstået var der mulighed for at både gruppemedlemmer og vejleder kunne stille spørgsmål der havde rejst sig under mødet. Efter mødets afslutning vedlagde vejlederen det tilsendte materiale til gruppen, med de notater og rettelser han havde til projektet. 


\subsection{Mailkorrespondance}\label{Mailkorrespondance}
Primært 

\section{Reflektion over vejledersamarbejdet}\label{Reflektion-over-vejledersamarbejdet}
Samlet set har gruppen været gode til at benytte vejlederen når vi følte det var en nødvendighed, med et vejledermøde o. 