\chapter{Vejledersamarbejde}\label{Vejldersamarbejde}
I forhold til 1. semester projektet, hvor der blev tildelt grupperne både en hoved- og bivejleder, er der på 2. semester kun blevet tildelt en enkelt vejleder til gruppen, hvortil gruppe A319 blev tildelt Anders Mariegaard som vejleder. Gennem projektet har gruppens vejleder været behjælpelig ikke kun med at guide gruppen igennem forløbet, men også til at finde materiale og generalle spørgsmål omkring studiet med mere. Til slut vil der blive reflekteret over vejledersamarbejdet som helhed har fungeret.

\section{Vejledermøder}\label{Vejledermoeder}
Gruppen har hovedsagelig haft et vejledermøde en gang om ugen, såfremt det kunne lade sig gøre for både vejleder og gruppe, men har dog undladt det i perioder, hvor gruppen havde andet at fokusere på såsom OOP eksamensopgaven, hvor projektrelateret arbejde blev sat i bero. Dagen forinden vejledermøderne blev det materiale gruppen ville have gennemgået sendt til vejlederen, samt en læsevejledning til materialet. Ved selve mødet, blev der skrevet referat, som efter mødet ville blive sendt til vejlederen.

\section{Vejlederkommunikation}\label{Vejlederkommunikation}
Både op til og efter vejledermøder har gruppen diskuteret og gennemgået hvad gruppen mener at der skulle diskuteres til vejledermøderne hvortil der blev udformet et agenda, som blev tilsendt til vejlederen. Når et vejledermøde var overstået var der mulighed for at både gruppemedlemmer og vejleder kunne stille spørgsmål der havde rejst sig under mødet. Efter mødets afslutning vedlagde vejlederen det tilsendte materiale til gruppen, med de notater og rettelser han havde til projektet. 

\subsection{Mailkorrespondance}\label{Mailkorrespondance}
Brugen af mailkorrespondancer imellem gruppe og vejleder var meget begrænset, eftersom det næsten udelukkende blev brugt til at sende materiale, agendas og referater afsted til gruppens vejleder, eller såfremt et mødetidspunkt skulle ændres. I løbet af projektet er mailen kun blevet brugt en enkelt til at stille spørgsmål, hvilket blev gjort i slutningen af projekt perioden. Ifølge vejlederkontrakt skulle alt materiale der skulle anvendes til det efterfølgende vejledermøde sendes senest 24 timer inden mødet, hvilket i nogle tilfælde blev ikke blev overholdt, da gruppen blandt andet har været nødt til at rette fejl i LateX.

\section{Reflektion over vejledersamarbejdet}\label{Reflektion-over-vejledersamarbejdet}
Samlet set har gruppen været gode til at benytte vejlederen når vi følte det var en nødvendighed, med et vejledermøde o. Desværre har der nogen gange været svært for gruppen at forstå hvor vejlederen ville hen med projektet, hvilket formentligt skyldes at det tog lang tid før gruppen selv forstod hvor gruppen som helhed ville hen med projektet, samt hvad det var de ville opnå. Derudover vil det eventuelt have været en fordel, hvis gruppen havde benytte mailkorrespondancer oftere når et spørgsmål rejste sig i gruppen, som kunne have fremskudt arbejdsprocessen ved at blive afviklet omgående.