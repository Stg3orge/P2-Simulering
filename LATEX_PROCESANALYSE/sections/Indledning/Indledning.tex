\chapter{Indledning}\label{Indledning}

Procesanalysen er udformet med det formål at reflektere over projektet, gruppen som helhed samt gruppens måde at arbejde på igennem hele forløbet. Denne refleksion over processen vil dermed også afspejle hvilke erfaringer gruppens medlemmer har fået igennem projektet og hvilke redskaber de kan tage med til senere projekter, både på universitet, arbejdsmarkedet og andre steder. Samtidig giver den også et indblik i hvilke fejl, problemer og konflikter gruppen er stødt på undervejs, men også alternativer til forbedringer eller hvordan disse ting kan undgås og afvikles i fremtiden. Projektet og procesanalysen blev udført af gruppe A319 der bestod af 7 medlemmer dannet ud fra 3 forskellige grupper fra 1. semester projekt, og alle havde arbejdet sammen med mindst et andet gruppemedlem i et tidligere projekt. Projektet har både udfordret gruppen fagligt men også socialt. Dette vil kunne ses undervejs igennem procesanalysen, hvor der også vil blive givet forslag til hvordan disse udfordringer kan afhjælpes i fremtiden, samt hvorfor disse udfordringer opstod til at starte med.